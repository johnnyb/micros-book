\chapter*{Postscript: Going Beyond Microsecession}

Microsecession is a great way of getting past the idiocy of our current
governance. It insulates practitioners from the impact of stupidity on
a number of levels. It gives an assurance that even if the world has a
bad day tomorrow, we{\textquotesingle}ll be just fine.
It{\textquotesingle}s kind of like the Boy Scout motto—“Be
prepared”—plus a new way of looking at what is valuable and what is
not.

However, if microsecession allows us to be independent of the idiocies
of government, I would also like to share a deeper secret that
will allow you to go
beyond all idiocy, all theft, and all despair.  By storing up food and
making a steady stream of production, we can cushion ourselves against
the things in this world that take away wealth. However, our lives are
short. Everything we make here and that we store here will eventually
turn to dust. Even our long-term assets will, in the long run, come to
nothing.

We should not look to microsecession as an end point. This is not the
goal. Having your home become a storehouse for your family and
community is a good goal, but it is not the ultimate goal. In the
chapters on value, I urged you to rethink what you think is valuable. I
pointed out that value is deeper than money. However, value is deeper
than even what I suggested. Value happens when our lives match the
model that God gave for us. In the Sermon on the Mount, Jesus said, “Do
not store up for yourselves treasures on earth, where moths and vermin
destroy, and where thieves break in and steal. But store up for
yourselves treasures in heaven, where moths and vermin do not destroy,
and where thieves do not break in and steal” (Matthew 6:19-20, NIV). In
other words, there is deeper value still to be had, and with more
permanence, too.

Just as microsecession is a means to live outside and beyond the
idiocies and immoralities of government, Jesus calls us to live beyond
the idiocies and immoralities of living itself. Just like
microsecession suggests that we tie invented value systems to more real
asset, Jesus says that these real assets aren{\textquotesingle}t as
real as we thought, and we should tie these to the greater asset of His
kingdom. Just as microsecession doesn{\textquotesingle}t mean that we
shouldn{\textquotesingle}t have any cash, following Christ
doesn{\textquotesingle}t mean that we shouldn{\textquotesingle}t have
any stored possessions—it simply means that their value is transitory
and illusory in comparison with the value attained in following
Him and being in His
Kingdom. Just like our paper money is only there to serve the harder
assets, so our hard assets are only there to serve His kingdom.

It is hard, when you are entrenched in a system, to see outside of it.
It is difficult for a business to look outside of itself and see the
way in which the larger society is going to affect its long-term
health. It often takes an outsider to point out the problems within a
system, which are too ingrained in the system for its members to see.
Christians, however, have the advantage of seeing the world from the
outside. As citizens of heaven, rather than the world, there is a
deeper view of the world that helps take a critical eye to everything
that the world does and measure it in terms of
eternity{\textquotesingle}s values.

This can be immensely practical (I believe that the microsecession
system in this book is the result of this outlook) or immensely
impractical (just ask the martyrs who gave their life for the sake of
Christ{\textquotesingle}s message). But the point is that it leads you
to evaluate your situation in deeper, truer terms. It tells you when
practicality is at the expense of eternity or in support of it. 

If the idea of value deeper than money resonates with you, or if
discussions of dependence and independence make you want to be more
independent of this world{\textquotesingle}s crazy systems, or if,
having looked at the amazing things that God has already put in this
world, you want to know the Creator of the world better, there is good
news. God has made a way for you to know Him, and to be a part of His
kingdom rather than the world{\textquotesingle}s system. If you place
your trust in God, and believe in Jesus, His Son whom He raised from
the dead, He will make you a part of His kingdom, and you will have
truly seceded from the crazy, immoral world in which we find
ourselves.

If you are already a Christian, keep in mind that, while I believe that
this book will be timely and important, and will help us be a blessing
to others during the coming trials, that our true treasure is not here.
 If we increase our value and hoard it to ourselves, this will be no
gain for us at all.  Only by using it for the benefit of others will it
be valuable in God{\textquotesingle}s kingdom.
