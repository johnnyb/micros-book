
\clearpage\setcounter{page}{1}\pagestyle{Standard}
{\centering\selectlanguage{english}\bfseries\color{black}
Introduction
\par}


\bigskip

\section{What is Microsecession?}

\bigskip

{\selectlanguage{english}\color{black}
In the aftermath of the 2012 election, there was a short-lived movement
of individuals petitioning for their states to secede from the
Union\textcolor[rgb]{0.32941177,0.5529412,0.83137256}{. This movement
garnered} a little over half a million signatures. Like the previous
attempts at secession, I believe such actions are, at best, completely
misguided. I do understand the
sentiment\textcolor[rgb]{0.32941177,0.5529412,0.83137256}{, however}. }

{\selectlanguage{english}\color{black}
The federal government is simply not working out for the average person.
Republicans and Democrats pretend to be on different sides of issues,
but really \textcolor[rgb]{0.32941177,0.5529412,0.83137256}{both
parties are} driving us to the same place. Modern politics
\textcolor[rgb]{0.32941177,0.5529412,0.83137256}{may make} for funny
movies, but unfortunately, our lives and welfare are bound up within
the decisions that lawmakers make on our behalf.}

{\selectlanguage{english}\color{black}
It isn’t good enough to simply vote for better lawmakers. There is a
limit to how much impact any of us can have on the national scale. No
matter how great a congressman you elect, Congress as a whole is still
run by self-aggrandizing idiots. Do you think that winning the
presidency will help?  First of all, the presidency doesn’t create laws
or agencies or even waste money. All of that is handled by one of two
entities: an out-of-control Congress and the bureaucracy that surrounds
them.}

{\selectlanguage{english}\color{black}
Therefore, even though seceding from the union is unwise, we do need a
way to insulate ourselves from the tragedy of modern politics. Imagine
that the government is a large limousine with a drunk driver. The
question is, “How do I get home safely?”  When a drunk is driving the
car, the answer is, “Walk.”  What the average person needs is a way to
shield himself from the coming waves of economic and political turmoil
that are the inevitable result of horrifically misguided ideas. We need
to put distance between us and the actions of the federal government. }

{\selectlanguage{english}\color{black}
I call this microsecession.}

{\selectlanguage{english}\color{black}
Microsecession is not about profiting off of the fall of the economy.
Many books that have predicted the economy’s future problems have the
goal of telling us how to get rich off of the decimation of the
economy. This book
\textcolor[rgb]{0.32941177,0.5529412,0.83137256}{offers} neither wealth
nor power over others. What it will do is help you make
you\textcolor[rgb]{0.32941177,0.5529412,0.83137256}{rself}, your
family, and your community more independent of what happens on the
national scale. It will allow you to sleep at night knowing that
\textcolor[rgb]{0.32941177,0.5529412,0.83137256}{the government’s
}stupidity is not your problem.}

{\selectlanguage{english}\color{black}
This book is not only about economics. The same drive to over-rely on
the federal government for economic support has spilled over into other
areas. We focus our political attention on national leaders, national
spokespeople, national news reporters, and national commentators. We
have lost our strength and independence as communities. Likewise,
within media we focus on national movie releases, national television
programs, and national sports. We eat at national chain restaurants,
buy food at national chain grocery stores, and buy products produced on
a national scale. In doing so, we have lost touch with the gifts that
our own communities have to give. }

{\selectlanguage{english}\color{black}
Let me say that there is nothing intrinsically wrong with national
groups, movements, laws, media, or anything else. In fact, having
organizations and movements on a national scale helps us employ
economies of scale which make everybody’s lives better. The problem
comes when we hand over the reins entirely to these large-scale
organizations. This makes our communities extremely fragile and
vulnerable. We must move toward independence of all of these things and
strengthen our local households and communities in order to weather the
coming storms. This book calls for a re-imagining of the way we live
our lives in order that we be stronger, more independent communities,
whose welfare is not wholly bound up in the decisions of a few.}

\section{Who This Book is For}
{\selectlanguage{english}\color{black}
There are many books on the market that tell rich people how to protect
their assets in the coming meltdown. These books are for other people.
Not being rich
\textcolor[rgb]{0.32941177,0.5529412,0.83137256}{myself}, I can’t take
very much of their advice. I imagine there are a lot of people in that
situation. You may understand that changes are coming that may affect
you. You pick up a book, and lo and behold, it makes lots of investment
suggestions. Then you look at your wallet and realize that with the
\$20 bill in it, your chances at participating in the stocks, options,
and bonds market is probably pretty small. So you close the book and
wish that someone had a plan or idea about what an average working
family can do.}

{\selectlanguage{english}\color{black}
If this is you, then you are the person this book is for. The
suggestions in this book aren’t necessarily \textit{free}, but if you
can spare \$10, \$20, or \$100 on occasion, then you can at least take
some steps forward.}

{\selectlanguage{english}\color{black}
In addition, \textcolor[rgb]{0.32941177,0.5529412,0.83137256}{and}
\textcolor[rgb]{0.32941177,0.5529412,0.83137256}{as I’ve already
mentioned, }this book is about more than money. It is about rethinking
and transforming the way we live as individuals, families, and
communities. It includes topics concerning money, yes. And in that
sense, it may be useful to you even if you have a large portfolio. If
you are set financially, this book could prove transformative to your
outlook on both life and money.
\textcolor[rgb]{0.32941177,0.5529412,0.83137256}{I hope it is. But
w}hile money is a big issue, the ultimate goal is to use money as a
vehicle to discuss wider issues. }

\section[Some Caveats]{Some Caveats}
{\selectlanguage{english}\color{black}
I should note that the principles and plans laid out in this book are
not foolproof. There are potential governmental practices, policies,
regulations, and laws that could adversely affect these ideas. However,
as I think you’ll see, the types of governmental action that could
adversely affect them are all tyrannical. If the government chose to
step in at these points, it would be at the cost of our liberties in
general, not just your pocketbook. }

{\selectlanguage{english}\color{black}
I should also note that we are still one country. What happens at the
federal level does matter. However, there are already plenty of books
on what should be done at the national level. There are very few books
about what we should be doing to mitigate the damage that has been done
and the damage that is coming. We are all better off when the people at
the top deal honestly, fairly, morally, and intelligently. However, we
also recognize that this is not the case today. So for the present
time, we individuals, families, and communities need a plan of action
against the people making the national decisions.}

{\selectlanguage{english}\color{black}
Although the goal of this book is to make us more independent of
governmental
mismanagement\textcolor[rgb]{0.32941177,0.5529412,0.83137256}{—and
premised on a general skepticism that the government is otherwise
capable—}I will still make some policy suggestions along the way. The
goal of these is not to fix what is broken with the government. There
are plenty of other books for that. No, the goal here is to  consider
small, structural changes to the way the government operates at the
local, state, and national levels
\textcolor[rgb]{0.32941177,0.5529412,0.83137256}{that will} allow more
room for private activity and autonomy.}

\section{How This Book is Organized}
{\selectlanguage{english}\color{black}
This book will begin discussing the basic monetary problems the country
faces that make microsecession a very pressing issue. I believe in
microsecession even in the absence of such issues, but I think that
\textcolor[rgb]{0.32941177,0.5529412,0.83137256}{our country’s} current
monetary problems
\textcolor[rgb]{0.32941177,0.5529412,0.83137256}{force us to look at
microsecession m}ore urgently than we would otherwise desire. }

{\selectlanguage{english}\color{black}
Following the initial discussion of the economy is a series of chapters
on how microsecession works in different aspects of
\textcolor[rgb]{0.32941177,0.5529412,0.83137256}{everyday life}. The
goal of this part of the book is to help us rethink the way we approach
life in order to strengthen ourselves, our families, and our
communities. The book will end with a few chapters on some larger
principles that will help summarize the take-away lessons of the book,
and also help expand the idea of microsecession into new areas
\textcolor[rgb]{0.32941177,0.5529412,0.83137256}{that may be important
but that this book might not cover specifically.}}

{\selectlanguage{english}\color{black}
While this book is mostly about the ways individuals can reorient their
lives to avoid the fallout from bad governmental decision-making, I
occasionally add sidebars that contain policy notes
\textcolor[rgb]{0.32941177,0.5529412,0.83137256}{and} suggestions for
policy makers. My hope is not that the government will wake up and fix
its problems—I wish I believed it could. My hope instead is that if
\textcolor[rgb]{0.32941177,0.5529412,0.83137256}{anyone in government}
reads this book, \textcolor[rgb]{0.32941177,0.5529412,0.83137256}{he or
she} will realize that there are some very basic steps
the\textcolor[rgb]{0.32941177,0.5529412,0.83137256}{ government can
take to} allow individual working-class people
\textcolor[rgb]{0.32941177,0.5529412,0.83137256}{more insulation from
economic fallout}. }
