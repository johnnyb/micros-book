\chapter*{Introduction}
\addcontentsline{toc}{chapter}{Introduction}

\section{What is Microsecession?}

In the aftermath of the 2012 election, there was a short-lived movement
of individuals petitioning for their states to secede from the
Union. This movement
garnered a little over half a million signatures. Like the previous
attempts at secession, I believe such actions are, at best, completely
misguided.   The Union is like a marriage---we stand together,
for better or worse.  If my wife threatened to run off
every time I did or thought something stupid, it would make the entire
idea of marriage meaningless.  Actual secession would also create
numerous practical problems---each state would have to start implementing
its own border security and passport system, and interstate travel, which
we currently think of as our national right, would have the same limitations
and hassles as traveling to foreign countries.  I do understand the
sentiment, however---America is certainly going in many wrong directions
simultaneously, and it is affecting each of us personally.  Going with 
the marriage analogy, if I became an unemployed drunk, it would make sense
for my wife to do something to protect herself and the kids from my stupidity.

Likewise, the federal government is simply not working out for the average person.
Republicans and Democrats pretend to be on different sides of issues,
but really both
parties are driving us to the same place. Modern politics
may make for funny
movies, but unfortunately, our lives and welfare are bound up within
the decisions that lawmakers make on our behalf.

It isn’t good enough to simply vote for better lawmakers. There is a
limit to how much impact any of us can have on the national scale. No
matter how great a congressman you elect, Congress as a whole is still
full of self-aggrandizing idiots. Do you think that winning the
presidency will help?  The president isn’t the one who create laws
or agencies or even appropriate money, though he can certainly waste it. 
Laws and appropriations are handled by congress, and gaining control
of congress is more daunting than the presidency.  It's not about which
party is in control, as they seem to both be willing to drive us over
one cliff or another.  The problem is that they are all tinkering at
the edges of a flawed system that needs major, fundamental changes.
The Republicans need to realize that just being contrarian isn't 
the same thing as governing, and the Democrats need to realize that 
good intentions don't lead directly to good policy.

Therefore, even though seceding from the union is unwise (and probably 
impossible anyway), we do need a
way to insulate ourselves from the tragedy of modern politics. Imagine
that the government is a large limousine with a drunk driver. The
question is, “How do I get home safely?”  When a drunk is driving the
car, the answer is, “Walk.”  What the average person needs is a way to
shield himself from the coming waves of economic and political turmoil
that are the inevitable result of horrifically misguided ideas. We need
to put distance between us and the actions of the federal government. 

I call this microsecession.

Microsecession is about removing yourself from the out-of-control systems
which have trapped the country in a downward spiral.  It is about giving
yourself independence by decreasing your dependence on government
and societal systems which are broken, ill-conceived, or that are 
more prone to failure than they are advertised to be.  

Microsecession is not about profiting from the fall of the economy.
Many books that have predicted the economy’s future problems have the
goal of telling us how to get rich off of the decimation of the
economy. This book
offers neither wealth
nor power over others. What it will do is help you make
yourself, your
family, and your community more independent of what happens on the
national scale. It will allow you to sleep at night knowing that
the government’s
stupidity is not your problem---or at least not as much of your problem.

This book is not only about economics. The same drive to over-rely on
the federal government for economic support has spilled over into other
areas. We focus our political attention on national leaders, national
spokespeople, national news reporters, and national commentators. We
have lost our strength and independence as communities. Likewise,
within media we focus on national movie releases, national television
programs, and national sports. We eat at national chain restaurants,
buy food at national chain grocery stores, and buy products produced on
a national scale. In doing so, we have lost touch with the gifts that
our own communities have to give. 

Let me say that there is nothing intrinsically wrong with national
groups, movements, laws, media, or anything else. In fact, having
organizations and movements on a national scale helps us employ
economies of scale which make everybody’s lives better. The problem
comes when we hand over the reins entirely to these large-scale
organizations. This makes our communities extremely fragile and
vulnerable. We must move toward independence of all of these things and
strengthen our local households and communities in order to weather the
coming storms. This book calls for a re-imagining of the way we live
our lives in order that we be stronger, more independent communities,
whose welfare is not wholly bound up in the decisions of a few.

\section{Who This Book is For}

There are many books on the market that tell rich people how to protect
their assets in the coming meltdown. These books are for other people.
Not being rich
myself, I can’t take
very much of their advice. I imagine there are a lot of people in that
situation. You may understand that changes are coming that may affect
you. You pick up a book, and lo and behold, it makes lots of investment
suggestions. Then you look at your wallet and realize that with the
\$20 bill in it, your chances at participating in the stocks, options,
and bonds market is probably pretty small. So you close the book and
wish that someone had a plan or idea about what an average working
family can do.

If this is you, then you are the person this book is for. The
suggestions in this book aren’t necessarily \textit{free}, but if you
can spare \$10, \$20, or \$100 on occasion, then you can at least take
some steps forward.

In addition, and
as I’ve already
mentioned, this book is about more than money. It is about rethinking
and transforming the way we live as individuals, families, and
communities. It includes topics concerning money, yes. And in that
sense, it may be useful to you even if you have a large portfolio. If
you are set financially, this book could prove transformative to your
outlook on both life and money.
I hope it is. But
while money is a big issue, the ultimate goal is to use money as a
vehicle to discuss wider issues. 

\begin{infonote}[Microsecession and Public Policy]
While this book is mostly about the ways individuals can reorient their
lives to avoid the fallout from bad governmental decision-making, I
occasionally add sidebars that contain policy notes
and suggestions for
policy makers. My hope is not that the government will wake up and fix
its problems—I wish I believed it could. My hope instead is that if
anyone in government reads this book, he or
she will realize that there are some very basic steps the government can
take to allow individual working-class people more insulation from
economic fallout.  

If you work within the government, you can find a
list of all of the policy notes at the beginning of the book right
after the table of contents.  In addition, chapter \ref{chap_policy_principles}
contains a list of principles that can help guide public policy
in a direction friendly to microsecession.
\end{infonote}

\section{How This Book is Organized}

This book will begin discussing the basic monetary problems the country
faces that make microsecession a very pressing issue. I believe in
microsecession even in the absence of such issues, but I think that
our country’s current monetary problems force us to look at
microsecession more urgently than we would otherwise desire. 

Following the initial discussion of the economy is a series of chapters
on how microsecession works in different aspects of everyday life. The
goal of this part of the book is to help us rethink the way we approach
life in order to strengthen ourselves, our families, and our
communities. The book will end with a few chapters on some larger
principles that will help summarize the take-away lessons of the book,
and also help expand the idea of microsecession into new areas
that may be important but that this book might not specifically cover.

\begin{infonote}[Become a Part of the Revolution]
For those that want to go further, after reading this book you should come 
visit the microsecession website at www.microsecession.com.
It contains resources, links, and a blog covering the latest information
in microsecession.  The site also contains an online store that
will help you get started and dig deeper.

In addition, the impact of microsecession will be greater if
more people join in.  Consider buying extra copies of this book
and handing it out to friends and sending it to your local, state,
and national legislators.  Let them know that you want laws
that follow the principles outlined in this book.  Remember, it
isn't just your national politicians that count---many of the policy
ideas in this book apply more to local governance than they do to 
national politics.
\end{infonote}
