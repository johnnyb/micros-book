\chapter{Microsecession Policy Principles}
\label{chap_policy_principles}

Thus far you’ve read
several sidebars on public policy suggestions that would be beneficial
for microsecession.
Let’s take what we’ve
learned in those sidebars and cover some basic principles that can
help guide a wide variety of legislation and policy decisions toward
microsecession.

\section{Principle 1: Subsidiarity}

Subsidiarity is a principle that states that any matter should be
delegated to or handled by the most local authority that is capable of
addressing a given issue effectively. In other words, the federal
government should not be involved in what the state governments can do;
the state governments should not be involved in what the city and
county governments can do;
and none of them
should interfere in affairs that the family can effectively handle.

This principle has been at the core of nearly every successful
large-scale venture, from the U.S. Constitution to the Roman Catholic
Church. The fact is that most issues are either local or have very
strong local components. The U.S. Congress
passes many bills
that are problematic, not because there is anything inherently wrong
with them, but because the issue is better addressed at a lower level.
This is why most criminal law is enacted at the state level. It
isn{\textquotesingle}t
that murder is wrong
in Texas but okay in Chicago, but that the specifics of how crimes are
ranked for purposes of investigation and punishment, and the means by
which such measures are enacted, are all based on highly local
considerations.

This is important for microsecession because it gives communities
greater independence; it doesn{\textquotesingle}t saddle them with
needless outside requirements and regulations. It allows communities to
regulate themselves in ways that make sense within the community. In
addition, when laws are focused on the top of the organization, it
means that everyone must be overly-involved in national politics. This
is where a lot of the heat of politics comes from. If the principle of
subsidiarity were followed, then a lot of the political wrangling would
simply disappear. Take for instance the issue of national health care.
Mandated and subsidized health care was a complete non-issue when
Massachusetts implemented it. Why? Because it made sense to the people
of Massachusetts. However, when a uniform law is implemented across the
entire United States, there is considerable disagreement over what
constitutes a good health-care law between the states. If the
Massachusetts idea was universally great, there is no reason why it
wouldn{\textquotesingle}t have spread to the other states on its own.
By implementing it as a national policy, however, much unnecessary
grief was imposed on the nation. If a state is uninterested in
government-run health care, why should it be forced
to implement it?  Do
we think people in different states are too stupid to come up with
solutions to their own problems?

The principle of subsidiarity makes policy-making both more effective,
since it more adequately addresses local concerns, and less divisive,
for precisely the same reason. It allows a greater independence and
diversity of communities who are not forced to conform to policies that
don{\textquotesingle}t match their values and situations.

\section{Principle 2: Diminishing Returns of Laws}

America is awash in legislation and even more so in regulation. There is
no one alive who can possibly understand all of the laws and their
implications. Chances are, you are probably breaking laws right now
that you don{\textquotesingle}t even know exist. This leads to a number
of problems. 

First of all, it is restrictive to business. Many people are fearful of
starting a business because they are worried that they will run afoul
of some obscure law and will be harshly penalized or even jailed. 

Second, it leads to selective enforcement. If there are so many laws
that someone is likely breaking one of them, it is easy for governments
to abuse their power through selective enforcement. If everyone is
breaking some law, then corrupt politicians can simply enforce existing
laws against people they don{\textquotesingle}t like,
using the cover of
effective governance as a pretext for punishing political enemies.

Last, and most importantly, laws send messages. The more laws there are,
the more confused the message is. This is the power of the famous Ten
Commandments from the Bible. It is a short, easy-to-memorize list of
rules that cover nearly every major offense that someone is likely to
commit. By keeping the list short, it is obvious that this list is more
important than the other rules given (there are over 600 total rules),
and most of those rules can be inferred from the Ten Commandments
themselves. The effectiveness of this approach can{\textquotesingle}t
be overstated. There may need to be some clarification on what legally
constitutes stealing and what the punishment should be (hence the 600+
additional rules), but the everyday person can know they are in the
clear if they follow the commandment: “Do not steal.”  

If there are too many laws, or if they are not easy to follow or
understand, then their value goes down. Laws can be a blessing by
making it clear what a society{\textquotesingle}s standards and
requirements are. However, when there are too many of them, they become
a burden, as they \textit{confuse} what society{\textquotesingle}s
standards and requirements are. If a government official says, “We need
more small businesses,” but then the regulations mean that only those
with superhuman powers or an army of lawyers can start one, we are
sending conflicting messages.

In addition, having too many laws engenders disrespect for law. If there
are millions of pages of regulations, can you really take them
seriously?  On the
other hand, if we make the laws short and simple to understand and
obey, this clarifies both what someone is supposed to do, as well as
the values that are behind
the laws. This leads
to a greater overall respect for the laws themselves and a greater
understanding of the values of the community within the public.

Ultimately, we should
keep in mind that more laws make the laws less effective.  If we limit
ourselves in what we legislate, we can make all legislation more
effective in society. 

\section{Principle 3: Visibility of Risk and Cost}

One of the underlying problems that causes so much turmoil in our
country today is that
we try to mask the
problems that we so often cannot solve.  Masking problems makes them
worse, because no one can see them. The discussion of health insurance
pointed out that having insurance companies bear all medical costs is
actually hiding the true costs and risks of health care from the
public. Since we cannot see where the costs and risks are, we are
unable to make informed choices.

A similar situation is happening with the financial markets. The
government, in the midst of
a financial crisis,
is trying to paper over the problems in the economy. By doing so, they
are making the inherent risks invisible. They think that they are
removing the risks. What they are really doing is making them larger
yet less visible. This is similar to what happened with the housing
bubble. Fannie Mae and Freddie Mac, the two government-sponsored
housing loan giants, would buy risky loans. This
didn{\textquotesingle}t actually reduce the risk of the loans; it just
made them look safer because the government was behind it. Instead, it
concentrated risk into a single entity, making it vulnerable to
catastrophe.

Many problems will never be fully solved. The worst thing we can do with
these sorts of problems is to paper over their risks and costs.
Instead, we should be finding ways to communicate risks and costs
better, so that individuals are better able to make informed
choices.[I feel
confused by this section. Maybe it’s just because I’m tired. Can you
give another example of something that they’re trying to hide that is
risky, or could be a problem, for me?]

\section{Principle 4: Unrestrained Hyper-Local Commerce}

The principle of unrestrained local commerce is actually a corollary of
several of the above principles, but I think it deserves special
attention. The idea is that, just like there are no regulations,
licenses, or special restrictions I need to take into account to serve
my family dinner or fix the family car, or even have friends over for
dinner or fix their car, if commerce is highly localized, it should
have the same restriction-free nature.

Think about it this way. Why do we have regulations on businesses?  If
the businesses were acting improperly, wouldn{\textquotesingle}t we
expect customers to simply not shop there?  Well,
what if I go to a
restaurant on the other side of town? I probably won{\textquotesingle}t
know the owner. I won{\textquotesingle}t be familiar with the
restaurant{\textquotesingle}s reputation and probably haven’t seen the
backside of the kitchen. Therefore, regulations protect me, the
customer, because I have no idea who this person is with whom I am
doing business.

However, if my next-door neighbor sells me something, I have an extra
advantage. I know my next-door neighbor and can ask his other neighbors
about him, since they know him too. I will know something about how
clean he keeps his home and how conscientious he is. Because of this,
it really doesn{\textquotesingle}t matter at all what a legislator
thinks of our transaction. I have all of the information I need to make
an informed choice. In addition, because he is my neighbor, he has a
social pressure to keep his products safe that he
wouldn{\textquotesingle}t have if, say, he were living in one part of
town but selling to another part of town.

Hyper-local commerce
is especially beneficial to the poor because it allows poor communities
to more easily pull themselves up by their bootstraps. If you need to
consult with lawyers, get licenses, and jump through many hoops to
start a business, then the poor will always need \textit{someone else}
to help them get out of their situation. By allowing and encouraging
hyper-local commerce, we give communities and individuals a way to
improve their lives easily.

Likewise, we need to provide easy methods of transition from hyper-local
to local businesses, and from local businesses to regional businesses,
and from regional businesses to national businesses. Each jump can and
should involve more requirements and oversight, precisely because each
community is less and less homogenous and has less and less information
and social connection to the business. But for the hyper-local
business, as long as the business is doing things that are otherwise
legal to do, we should stay out of the way and only intervene if
problems arise, not before.

If oversight were
needed—for example, if a local business had a history of damaging its
neighbors without remorse—there could be a mechanism in place that
removes such a business from the hyper-local definition.  Nonetheless,
I think that the default state for such businesses should be that they
are outside the reach of any business law. 

Another place where having unrestrained hyper-local businesses is
beneficial is in the event of a nationwide economic collapse. What
would help us the
most in such an
instance is for people to already be in trade relationships with their
communities.  This
would insulate
communities that follow these principles from the effects
of the collapse, as
they would already
have people they can depend on
to help them with
their needs within their own communities. We
wouldn’t need FEMA
(Federal Emergency Management Agency) because we
would have one
another. This would
also take the burden off of the government to provide emergency
assistance for individuals in an economic collapse and instead allow
them to focus on rebuilding national
infrastructure.

So, how do you define hyper-local?  I would say that the definition
itself is probably regionally dependent. But in general, I would say
that hyper-local
businesses are those
where the advertised market consists entirely of people who are in the
same neighborhood as the business owner, or people with which he has an
existing non-business relationship with. I think by doing this we can
provide people and communities the opportunity to help themselves in
any circumstance and allow them to be more independent of the fate,
charity, and wisdom (or lack of these) of planners, officials,
traditional businesses, and moneylenders.
