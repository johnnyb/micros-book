\chapter{Seceding from the Education Industry}

In the chapters on money, we covered the idea that there is a difference
between money and value, and that our priority should be on value
rather than money. In the chapter on healthcare, we looked deeper into
what it means to be healthy rather than just take a pill to make
us feel better. The
aim of this chapter is to take a similar approach to education.

When people think of the word “education,” usually the first thing that
pops into their minds is their local elementary school. When we think
of a “well-educated person,” we think of someone who graduated Harvard
with a Ph.D. But is this really what it means to be educated?

First of all, let us draw a distinction between education and training.
The purpose of training is to be better at something. You can train to
be a welder, a computer programmer, or an engineer. Education, however,
teaches you how to think and gives you a large repertoire of knowledge
from which to analyze
the world around you. Training helps you to know what to do in a
specific situation, while education helps you to figure out what to do
in new situations. Education also helps you to know when your training
is wrong. Even more, education helps you to analyze the claims that
other people make in order to determine their accuracy or
reasonableness. Still more, excellent education helps you to formulate
wise courses of action based on accurate information.

But is this what our educational system currently teaches?

It is very interesting that despite requiring people to take math,
science, history, and a plethora of other subjects, two subjects that
are most often overlooked at both the lower and upper levels of
education are philosophy and logic. These are precisely the tools most
needed for a properly educated person. Can we reasonably call someone
educated who does not know the history of the ideas that they hold? 
Philosophy requires that all ideas be held up to scrutiny, even those
which we imagine to be firm or solid. Logic requires that we properly
ground all of our ideas, or at the very least know where they are
ungrounded. These two subjects, which should be at the core of every
curriculum properly called “education,” are conspicuously missing from
nearly every educational standard.

\section{Depth Beats Breadth}

The core issue is that knowledge as it is usually conveyed through
modern education is delivered in a piecemeal fashion. Even though there
may be a wide breadth of knowledge delivered in modern education, the
knowledge is almost always presented in one of these forms: either as
an irrelevant fact or as a useful fact. By “irrelevant fact,” I mean
that the fact is not tied in to anything else. Most history dates, as
studied in elementary schools, fall into this category. You memorize
the date because it is in the curriculum to do so. There may be some
context provided, but the ultimate goal is just to know the fact. By
“useful fact,” I mean that a fact is used to determine something else.
Much of math and science fall into
the latter category.
We learn the equation for the area of a circle, and then we can plug in
numbers to this equation and get an answer.

Real education dives
much deeper than this. Real education certainly makes use of irrelevant
facts and useful facts. But real education teaches the student how
everything fits
together, and how such ideas are constructed in the first place.

For fun I teach a high-school chemistry class. It is easy to take a
chemistry class and miss what is really going on in chemistry. It isn’t
actually science. Chemistry is all about philosophy. An ancient
philosophical idea is \textit{ex nihilo nihil fit}, which means, “Out
of nothing, nothing comes.” Translated more colloquially, it means,
“You don’t get something from nothing.”  This fact may seem trivially
true. In fact, it is currently taken for granted in many areas of
inquiry, but it is
grounded only in philosophy. Its truth was not immediately obvious in
philosophy either, but rather took time to develop and had to be argued
for. In any case, this philosophical idea is at the core of ideas such
as the laws of thermodynamics. 

One of the experiments that my chemistry students hate the most is
determining how sensitive an object's temperature is
to energy coming in or out. This is usually the first time they have
had to really think to apply math to a subject, and it is hard the
first time. However, the calculation is based on a simple principle—the
amount of energy that flows into an object is the same amount of energy
that flowed out of another. Why do we know this? Because, as the
philosophers say, out of nothing, nothing comes. The energy in an
object had to come from somewhere. When an object cooled down, the
energy had to go somewhere. By using this philosophical idea, we can
then determine that, if we carefully control our environment, the
amount of energy lost by one object is the same amount of energy that
is picked up by another object. \textit{Voila}! Out of philosophy, we
get chemistry. Showing students how insights from one field lead to
insights in other fields teaches the students by example how to
integrate knowledge from various areas. It shows them how to take
breadth and create depth.

Without such integrations, what the student has at the end of
her education is
simply a giant grab-bag of facts. This misses the point of having an
education almost entirely. Certainly, a student who knows a variety of
facts is more likely to pull something worthwhile out of
the bunch than
someone who knows nothing at all, but all education is actually at the
expense of other types of education, not at the expense of no
education. If I were to take students and simply employ them in a job
starting in the sixth grade rather than let them finish their
education, they too
would learn every
day. They would only learn a different kind of knowledge, depending on
their job. If they cut hair, they would learn about people. If they
waited tables, they would learn about service. If they ran a cash
register, they would learn at least something about money and
efficiency. 

My point is that, no matter what, living itself provides us with a set
of facts. What sets real education apart is that it provides a deeper
understanding and wellspring from which a person can draw.

\section{Missing Knowledge}

Another large issue in our education
system is the
subjects that are completely missing. We have already discussed the
absence of philosophy and logic. However, there are many other highly
important subjects that get completely passed over. Morality and ethics
are almost entirely missing. Aesthetics is not to be found anywhere.
Theology isn’t even supposed to be discussed. Are these not also
important subjects?  For what purpose are we educating ourselves if we
cannot discuss higher purposes that are present in theology?  How shall
we know if we are putting our knowledge to beneficial or harmful use
except by learning morality and ethics?  And do we really want to build
a world in which the beautiful is ignored?  But this is where modern
education is leading us.

It hit me suddenly one day just how much my education was missing. One
day, while reading about various wars and empires throughout history, I
dared to ask myself a simple question: what was God doing in and
through this history and these wars?  I know this seems like an odd or
even dangerous question – one of those questions that one day leads to
a dogmatic preacher yelling at
his congregation, or
even worse. But here’s the deal: should we not be asking dangerous
questions?  Is that not one of the purposes of education in a free
society—to be free to ask the dangerous questions?  Now, I know that
someone will object and say, “How do you propose we determine that?” 
But that’s just the point, isn’t it?  If we ignore philosophy,
theology, ethics, and morality, we are denying ourselves the tools we
need to discover the answers to such questions. That doesn’t mean we
will get the answer 100 percent right the first time. But that’s also
true in math, chemistry, physics, and even history. The possibility of
being wrong doesn’t invalidate the exploration, and we have
systematically cut ourselves off from the tools we need to address
these important questions of life. 

When such important knowledge is lacking, and especially when such
important questions are not asked, we can be easily manipulated.
Without a grounding in ethics, then whichever publicity agent paints
her patrons with the
brightest colors will get the public approval. With a grounding in
ethics, we can understand the importance and severity of what people
are doing even if someone explains their evil actions using sweet
stories. Likewise, we can recognize when good people are being
improperly maligned. Most importantly, we can recognize the faults of
our own heroes and help to avoid idolizing anyone unquestioningly. But
without education in these areas, we will lack the tools needed to do
anything beyond our emotions, which are so easily manipulated.

\section{The Degree Mills}

Have you ever seen advertisements for those
schools that throw out
slogans like, “Earn a degree in your own home in only two years,” or,
“Earn college credit from life experience”?  These places have become
notorious for practically giving away diplomas for whatever type of
work is turned in. You give them money, and they give you a diploma
that says you have a degree.

The problem is, even though universities combat degree mills in words,
they seem to be giving every effort to becoming degree mills
themselves—just more expensive degree mills. I realized this when I
became responsible for hiring computer programmers. I spent many hours
interviewing candidates for programming positions within the company I
worked for. I was amazed to find that there were an alarming number of
people with computer science degrees from major institutions who didn’t
even know how computers worked!  They could do a few programming tasks,
but they really didn’t understand what was going on. And heaven forbid
you try to get them to think about anything beyond programming like the
business or customers. I was truly shocked at just how little education
people received in schools.

A lot of this problem stems from the attempt to get universal education.
There are some people, many people, actually, who just aren’t good at
academic work. They aren’t necessarily less smart (many are in fact
much smarter, generally); it’s that the academic mode of learning,
thinking, and acting doesn’t suit them. However, if the policy goal is
universal education, then that means the education must be something
that everyone can pass. In addition, college life has moved from an
environment of learning to a party-time drinkfest. While we might
imagine that universities are full of students eagerly drinking in
every piece of knowledge they can find
(pun intended), in
reality, they are
mostly full of overgrown adolescents trying to make excuses to get to
the next party. In the name of universal education, the standards have
to be lowered in order to make sure all such students can get a
degree.

In short, a degree really means nothing. For job skills, you would
probably do better to spend the time doing an internship for free in
the career you want. It would cost less money and get you a real head
start. Not only would you learn the industry, you would start making
contacts earlier. You would be head and shoulders above your peers who
paid \$60,000 or more to go to college.  And you
wouldn't be out \$60,000.

In theory, doing an internship would mean that you weren’t exposed to
the broad academic subjects taught in a university. I would buy that
argument if the universities actually made these priorities. Usually,
however, the deeper academic subjects (such as philosophy) are only
taught to freshmen in a course made so that anyone can pass it. If the
universities were serious about real education, they would teach these
subjects as senior-level courses and force
students to integrate
knowledge of their
degree with the knowledge of the wider world. Therefore, whether
a person goes to a
university or not, to be truly educated,
that person
has to educate
himself.
For the record, you
would probably wind up a more broadly educated person just by reading
R.C. Sproul’s \textit{The Consequences of Ideas} than you would getting
a typical four-year degree. 

Things are not all bad when it comes to universities. A few universities
(or more likely, a few programs within certain universities) really do
educate. The problem is that this is no longer the norm, and it is
almost impossible to tell from the outside the degree mills from the
true educators. The state of the university today means that no matter
what you decide to do about college, you must plan on self-educating
anyway. As Matt Damon famously said in \textit{Good Will Hunting}, one
day, you are going to realize that “you dropped 150 grand on a ****ing
education you could have got for a dollar fifty in late charges at the
public library.”

To secede from the modern educational establishment, we must learn how
to self-educate. It turns out that education is not the mysterious
beast that it makes itself out to be. Just as health winds up usually
being more about eating well than about knowledge of obscure medical
facts, education is more about reading, listening, and discussing than
memorizing arcane details from history and philosophy.

\section{Learning Any Subject}

There are two basic modes of education—learning a subject and learning
wisdom. Truly excellent education can combine these two, but we are
going to study them separately. The first mode is how to learn any
specific subject. In other words, if I want to learn about something
specific (say I
wanted to learn about
carpentry) how would
I go about it?  

The best method I am aware of for learning a subject is known as the
\textit{trivium}, meaning “three ways.”  The \textit{trivium} is
composed of three stages of learning. In practice, the stages overlap,
but it is good for someone just starting out to view them as three
distinct stages. The first stage is known as the grammar stage. The
goal of this stage is to learn the basic parts or actions of a subject.
For academic subjects, this involves learning lots of vocabulary
related to the subject. The goal of this stage is not to develop a deep
understanding but to develop a workable knowledge of many different
pieces. Here are several examples of grammar-stage activities:

\begin{itemize}
\item 
In history, learning the names of the main countries, personalities, and
movements
\item 
In math, learning the basic facts of arithmetic
\item 
In basketball, learning the basic motions of dribbling, passing, and
shooting
\end{itemize}

The goal in this stage is to train your mind into thinking with the same
words and thoughts as other people in the subject. If you are learning
basketball, at the end of this stage, if someone says, “Pass me the
ball,” you know what to do.

The next stage is the logic stage. The goal of the logic stage is to
learn the way in which the different pieces that were learned in the
grammar stage work together in a comprehensible way. Using the
basketball example, in the logic stage we would learn to use passing
and dribbling to get ourselves a better position on the court to shoot.
In history, you would learn how preceding events laid the groundwork
for later events in history. In math, you would learn how the basic
operations in mathematics could be combined into formulas.

The final stage is the rhetoric stage. In this stage, you learn how to
use the logic of your subject creatively. For instance, it is all well
and good to know what causes led to World War I. But for the rhetoric
stage, you should be able to come up with a list of suggestions on how
to prevent world wars in the future. Similarly, you should be able to
look at modern situations and see what sort of results are likely. In
basketball, the rhetoric stage would be the ability to tailor an
offensive play to the defense you are playing against, or even develop
new plays. In mathematics, the rhetoric stage is the ability to discern
how mathematical equations can be developed to relate to novel
problems.

If you want to learn a new subject, you should break your pursuit up
into these stages. First, go and learn the key components of your
subject. Next, learn how to put them together into sequences and
combinations that accomplish a goal. Finally, learn to apply these
combinations creatively.

Not only does this system help you learn a new subject; it helps you to
know where you are in learning a subject. I know many people who think
themselves experts on subjects where their only knowledge is at the
grammar stage. Using the \textit{trivium}, if your only knowledge of a
subject is the basic vocabulary and skills, you know you have a lot
left to learn. 

\section{Learning Wisdom}

The second mode of education is learning wisdom. Wisdom is learned
through dialog and interactions. It is learned by taking ideas,
examining them, and talking about them. 

In \textit{A Thomas Jefferson Education}, writer Oliver Van De Mille
says, “Greatness isn’t the work of a few geniuses, it is the purpose of
each of us. It is why you were born. Every person you have ever met is
a genius. Every one. Some of us have chosen not to develop it, but it
is there.”  The point of his book is to teach us all to be better
students, and to reclaim education for ourselves.

You see, real education cannot be left to others. Others can do
teaching, but education is the job of the student. It is the job of the
student to recognize what they do not know and choose to do something
about it. It is the job of the student to seek knowledge and not be
complacent until they have discovered it. These aren’t the tasks of the
teachers. These are the tasks of the students. Great teachers, in fact,
rather than being a substitute for students educating themselves,
actually encourage students to take the lead on their education. A
mediocre teacher will tell you what they think you should know. A great
teacher will inspire you to seek out truths that are beyond what the
teacher himself knows. Teachers, if they know what they are doing, know
that their main function is to give a student the starting points for
the student to venture off themselves.

While students can learn on their own without teachers, I think that, as
Van De Mille suggests, we should seek out teachers and mentors to help
us on our way. This doesn’t have to be a formal teaching relationship,
just someone who can help to point you in the right direction and
provide a source of discussion about ideas—to point you to other
connections you may have missed.

If a Harvard-equivalent education can be purchased for \$1.50 in overdue
charges at the local library, then a real education can be purchased
for \$16 a month by finding a mentor and taking him or her out to lunch
once a month for discussion. You will still need to budget for the
\$1.50 in overdue charges at the library.

The way it works is simple. Your mentor suggests books to read, and you
read them. Carefully. Understandingly. We have a tendency to view books
like textbooks—as if books contain the “right” answers. Really, though,
books should be thought of as discussion starters, not discussion
enders. The point of reading the book is to open your mind to a new
avenue of thought, not give you a set of facts that are definitively
true. This is where the value of a mentor comes in. While you are
reading the book, you should discuss it with your mentor.

Your mind can be expanded by almost any book, but your time is better
spent if you read really good, deep books. Your mentor will probably
know a few to start out with. These
could be fiction or
non-fiction books. The point is that they contain ideas worth
discussing and worth wrestling with. Education is not about knowing the
right ideas; it is about being able to think about and analyze ideas.
Aristotle, the great grandaddy of philosophy, said that it is the mark
of an educated mind to be able to entertain a thought without believing
it. In other words, to be able to analyze ideas you disagree with—and
perhaps even find some value in them. An educated person will be
excited to learn new ways of approaching a subject, even if at the end
of the day they decide that the given approach is not worth taking.

The process of educating yourself is simple. Find a mentor, start
reading a book, and discuss it with
your mentor and
others. Repeat. Forever. This is the road to education, and it will
open your mind to a whole new world. 

\section{What to Do About School}

Of course, this is well and good for your own lifelong education, but
what about your children?  One option, which has recently gone from an
obscure fanatical position to being mainstream, is homeschooling. This
is what my family does, but I realize that it is not a choice for
everyone. It allows for a great deal of independence but can also be a
drain on time and finances. The fact is, through our taxes, we
\textit{already pay for schooling our children}.
Whether we homeschool
or send our kids to a
private school, we actually have to pay twice. 

While there are many good reasons not to homeschool, I want to cover one
\textit{bad} reason that people often come up with not to homeschool:
believing they are
unqualified.  I want to take some time and look at this objection in
depth, because it is both so common and so problematic. 

First, if you did not
learn the material well enough in high school to teach it to your
children, this represents a problem with public education, not a
problem with homeschooling.
Said another way, if
your teachers didn’t teach you well enough to teach your own children,
why are you entrusting them to teach your children too?  

The second issue with this objection is that many people think they need
a degree to teach anything at all. Nothing could be further from the
truth. The reason that people need degrees to teach is not because
teaching itself is hard, but because teaching \textit{thirty unruly
strangers at once} is hard. The dynamics are quite different when
those whom you are
teaching are your own children. That doesn’t make it easy—in fact, one
of mine tries to make it as difficult as possible—but the solution
isn’t found in a degree in early childhood education. In fact, most
often, if your child is difficult, the teacher is going to focus on
preventing that child from interrupting the other students. If the
unruly child is your own, your focus remains his full education.

The third thing to consider on this objection is that, if you follow the
steps outlined in the previous two sections, you can teach yourself
anything. As such, you can, with sufficient dedication, learn anything
you need to teach your child. This will require time, and it may even
require time that you don’t have. These are valid issues. I just want
you to know that the issue is a practical one. 
You may indeed be
lacking in time to devote to the subject, but your perceived
insufficiencies are not the problem. 

Finally, if there is truly a subject that you can’t master well enough
to teach your children, you should know that the point of homeschooling
isn’t that one person (you) be able to teach your children.
Rather, the parents,
should bear the ultimate responsibility for their child’s education.
Sound scary? It’s not.
Many homeschool communities have co-op groups where someone volunteers
or is paid a reasonable sum for educating on specialized subjects. I am
currently teaching such a co-op program for calculus and chemistry, and
it is actually really fun.

Whether or not you homeschool, you will need to educate your children
to some degree, at
some point anyway.  They will need to know the skills and habits
discussed earlier in this chapter, and they aren't
likely to learn them at school. You will need to teach your children to
be thirsty for knowledge and truth and to know how to obtain it.
Charlotte Mason, whose work in child education inspires many homeschool
parents today, said, “Self-education is the only possible education;
the rest is mere veneer laid on the surface of a
child's nature.”  If
your child excels in
school, great!  But make sure they know how to obtain knowledge outside
of school as
well. Make sure they
know how to evaluate opinions and ideas, and not just take every fact
learned in school as self-evident truth. Then you can truly consider
your child educated.

\begin{policynote}[Knowing When Education Isn't the Problem]
One bad habit that western civilization has is thinking that every
problem can be solved with education. Reinhold Neibuhr, in his book
\textit{The Irony of American }History, pointed out that In the east,
during the first half of the twentieth century, they believed the root
of evil was property and therefore tried to get rid of evil by getting
rid of property. Hence, the eastern hemisphere ran rampant with
communism. The west, on the other hand, believed that the root of evil
was ignorance and tried to get rid of every evil through education.

The problem with both of these views, as Neibuhr points out, is that
they forgets both the good and the bad of the human condition. We are
not singular entities. We are not economic beings nor rational
beings—we are human beings. Therefore, our solutions to problems must
take into account the whole person.  We cannot get rid of evil by
removing property or ignorance.

Often times, the solutions to our social ills have nothing to do with
education but with morality or habits. The solution to a bad habit is
not education but forming good habits. The solution to immorality is
not education on the problems
of immorality or its
potential consequences but in training and meditating in righteousness.
 Focusing on virtue can bring about better and more permanent changes
than merely educating on the problems of evil.

Education is often good and needed, but the sad thing is that our
leaders think education is the cure for the whole human condition. The
only cure I know of for the whole human condition is Christ.
\end{policynote}

\section{Resources}

\begin{itemize}
\item
\textit{The Consequences of Ideas} by R. C. Sproul.  This book is
one of the best introductions to the larger world of ideas and why
they matter.  It is short, easy to read, and very powerful.
\item
\textit{A Thomas Jefferson Education} by Oliver Van De Mille.  This
book goes into great depth on how to best learn wisdom through
literature.  It gives suggestions, reading lists, and encouragement
to help you get started on your own quest for wisdom.
\item
\textit{A Charlotte Mason Companion} by Karen Andreola.  If you are 
interested in homeschooling your children, this book will guide
you in the right direction.  It is not just about education, but
about teaching your child as a whole person---moral, spiritual, and
academic.
\item
\textit{The Everlasting Man} by G. K. Chesterton.  As mentioned in
this chapter, many of the subjects we studied in school, such as 
history, science, culture, and even religion, are missing many
of their more vital aspects.  This book will help you re-imagine
these subjects with more depth and humanity than any other.  In addition,
if you don't have the time to read the book, the audio book is 
available for free from librivox.org!
\end{itemize}
