\chapter{Seceding from the Food Supply}

We are currently awash in food. Are you hungry?  Turn around, and there
will probably be a snack machine or a vending machine offering you
whatever your heart desires for just a few quarters. Want to take the
family out? What kind of food do you want? You can eat Mexican, Asian,
Italian, or American meals just about anywhere in town. Want to stay
home for the evening?  That’s fine, too. Just head to the store and get
some noodles and a jar of sauce, and it’s all set.

As a society, we have so much readily available food that the idea that
it might not be there one day is completely unthinkable. The grocery
store has never failed to have what I needed, except perhaps snack food
that
I tried to purchase
on Superbowl Sunday. Regardless, it is foolhardy to think that this
situation will last forever. We must consider how fragile our food
supply is.

\section{The Fragility of the Food Supply}

What does it take to get the food into the markets every week?  Well,
the food has to be grown. Growing food itself is subject to a lot of
variables. First, there’s the weather.  No matter what you think of the
causes, the fact is that the weather is changing, and that weather
change is affecting how well we can grow food. Don’t think that this
must necessarily be followed by a return to “normal” weather in a year
or two. History is replete with multi-year droughts, mini ice ages, and
other long-term weather anomalies that can adversely affect food
production. The rise of pests and diseases can quickly bring a food
supply from abundance to scarcity. Also, the lack of symbiotic
organisms (e.g., bees) can wreak havoc as well. 

In recent years there has been a gigantic bee die-off, which can
adversely affect how well plants get pollinated, and likewise how much
they produce. But it isn’t enough just to grow the food. Food also has
to be purposed for human consumption—and for your grocery store in
particular. Are you familiar with ethanol gasoline? 
Ethanol takes a large
chunk out of the food
supply and literally burns it up by producing gasoline. That’s not
necessarily a problem if everything else in the food supply chain is
working properly, but it can exacerbate other problems. 

A lot of the food is also grown for feeding cattle, or sold on the world
market. If the dollar becomes worthless, why do you think big
agriculture is going to sell the food to you for your inflated money
when they can sell it to other people whose currencies are not in the
toilet?

Perhaps the weakest link in the chain is transportation. How much has
the price of gas risen in the last few years?  In 2000, gas prices were
at about \$1.70 per gallon. Today, gas prices are at \$3.49 per gallon.
The ability to have a ready supply of cheap food from around the globe
is only as stable as the ability to get it transported to you. Rising
food prices are problematic, but not as problematic as the potential
for not having the fuel to transport it at all.

As the ability to drill for oil on federal lands and offshore continues
to decline, our dependence on dictatorships for our fuel continues to
increase. And, again, this depends on the willingness of even our
domestic producers to sell \textit{to us} in the case that our currency
is heavily devalued. What then?  In that case, even if we have
sufficient food supplies, will we be able to get it to the stores? 
Will you be able to drive to the store to get it?  

I know this sort of situation sounds crazy. This system has worked our
entire lives. It may continue to work—I certainly hope it does! 
However, our prosperity has given rise to complacency. We think that
these systems and institutions are invincible. I assure you that they
are not. If we are wise, we will relearn how to live independently.
Hopefully, we won’t have to. But if we do, it is good to be prepared. 

\section{Food Storage}

We will discuss three ways in which we can secede from the
national food supply.
The first, which I believe is most important, is food storage. Let me
share with you a story from the Bible.

The book of Genesis tells us of the story of Joseph. There are many
important theological points you can pull from the story of Joseph,
like how God can use even terrible circumstances to accomplish His
goals. However, I want to focus on a more practical lesson. The story
of Joseph recounts how the country of Egypt went through a terrible
famine. The pharaoh, through a dream, was warned ahead of time about
the famine, and he asked Joseph what he should do. Joseph’s suggestion
was that during the good times, the people should save a large portion
of their grain so that they would have plenty during the bad years. 

Pharaoh took Joseph’s
advice. He built barns and storehouses of grain and saved all
he could. What most
people miss, however, is that it looks like \textit{only} the pharaoh
was saving the grain. The people of Egypt were largely free at this
point. But it seems they left all of the concerns of saving and storing
grain to someone else—namely the government. In this story, the
government actually did a good job of preparing and saving grain. There
was only one problem. When tough times come, those who have usually use
it as a power grab against those who don’t.

What happened to the people of Egypt during this time? Well, first, they
had some extra money, so they purchased the grain. But as the years
wore on, they ran out of money. They came to Joseph for food, but they
didn’t have any money to pay. So they offered themselves and their
families to Joseph as slaves in exchange for food. They said:

\begin{quote}
“We cannot hide from our lord the fact that since our money is gone and
our livestock belongs to you, there is nothing left for our lord except
our bodies and our land. Why should we perish before your eyes—we and
our land as well?  Buy us and our land in exchange for food, and we
with our land will be in bondage to Pharaoh. Give us seed so that we
may live and not die, and that the land may not become desolate.”  So
Joseph bought all the land in Egypt for Pharaoh. The Egyptians, one and
all, sold their fields, because the famine was too severe for them. The
land became Pharaoh’s, and Joseph reduced the people to servitude, from
one end of Egypt to the other. 

Genesis 47:18-21 (NIV)
\end{quote}

Those who were unprepared fell victim to those who were prepared. Those
who were prepared could demand any price, and require any service,
because people who are truly hungry will do anything for food. As such,
the pharaoh was able to enslave the whole nation because he was better
prepared than they were.

This is the situation you don’t want to be caught in. You don’t want to
be stuck in a situation where you don’t have the food you need for
yourself and your family. We don’t have slavery anymore, but is going
into permanent debt really that much different?  If you don’t prepare,
it is actually a reasonable thing to do to trade in every possession
you have—even your home—for food. However, the point of preparing is so
that in a sticky situation, we aren’t the ones who will suffer and can
actually have a surplus to help relieve the sufferings of others. If
things went south, imagine if you were able to provide for not only
your family but others?  In times of crisis, there is usually a great
reorganization of resources. When that happens, if you have your basic
needs taken care of, you can set yourself up for future success while
others are struggling just to find food.

It turns out that keeping a food supply is rather easy. All it takes is
a set of stackable food-storage buckets and a few hundred dollars worth
of food (mostly grains and beans). Because we are addicted to take-out
and refrigerator foods, we have this idea that food goes bad quickly.
But this isn’t really the case. A lot of foods will store for a very
long time—up to twenty years!  I’m not talking about those special
freeze-dried foods; I’m talking about real foods. Things like wheat and
oats and beans.

The most important thing to do with a food supply is to rotate it. That
is, you should eat from your food supply. That keeps everything fresh
and makes sure that if anything does have problems, you will know about
it earlier rather than later. Most people adopt a first-in-first-out
system. Basically, you have at least two buckets of each important
item, and you eat from the oldest bucket first. This way, you are
always keeping your supply fresh.

How much do you need?  Well, it depends on how much you want to prepare
for. The government (ready.gov) recommends between a three day and two week supply to be
sufficient, but most people who seriously prepare for a short-term
emergency think that a three-month supply is the minimum level you
should keep, and a one-year supply is usually considered sufficient for
most purposes. My family is closer to the three-month supply than the
one-year supply, but we are working on it. In any case, it is always
better to do something rather than nothing, and it is easy to work up
to what you need!  

The general recommendation for food storage is that, for each person
per month, you need
twenty-five pounds of grains (i.e., wheat, rice, corn meal, pasta,
flour, etc.), five pounds of beans, five pounds of sweetener (i.e.,
honey or sugar), a pound of salt, a half-pound of baking powder, a half
pound of cooking oil, some yeast, and anything else you like to cook
with. Also important are cooking and processing items that can work
without power, such as a solar oven or a hand-cranked grain mill.

The one difficult thing to store well is water—both because it takes up
so much space, and because it is so easy for it to go bad. You
generally need to store a gallon per person \textit{per day}, though
some of that can be used for non-drinking purposes such as washing
dishes. However, that quickly adds up. It is wise to keep a two-week
supply on hand, and then, if you need it, find other means for water
purification. 

In an emergency, other water sources can be used if they are purified.
Water can be purified in two steps: filtering and disinfecting. Water
can be filtered fairly simply by pouring it through a tightly-woven
cloth, such as a clean sock. It should be filtered several times until
the water is clear. Then, there are several methods of disinfecting
water. The easiest is to let it sit for several hours (preferably
overnight) in a jug with a silver coin. You can also use bleach. To
bleach your water, use plain 4 percent bleach, and add two drops per
quart of water. Stir it up, and then let it sit for thirty minutes to
an hour. 

To supplement your stored supply of water, you can gather water from
many sources, the easiest of which is the rain. Many natural food
stores sell large rain barrels that can be used to gather and store
rainwater. This can be used at any time for a variety of non-drinking
purposes, such as watering your garden.  In emergencies, it can also be
further purified and used for drinking. By combining water storage and
purification, you can be sure that you can take care of yourself and
your family in an emergency. 

To most modern Americans, the idea of storing food and water sounds,
well, strange.  We don{\textquotesingle}t even have a space in our
house to put it.  If you have a basement, this is rather easy, but what
about everyone else?  Thankfully, stored food doesn{\textquotesingle}t
take up that much space, especially with stackable containers.  If you
can find a corner of a room, or add shelving to your garage, this is
certainly something your can do.  Remember, when push comes to shove,
people will do just about anything for food.  Don{\textquotesingle}t be
that person.  Have a supply, so that if troubles come, you
don{\textquotesingle}t have to worry.

\section{Growing Food}

Food storage is a lot like a savings account. It is great to have a few
months of savings stored up so that if your company goes bust tomorrow,
you have enough money to live on until you can find or create another
job. In addition to having on-hand cash, it is also good to have a
continual source of revenue. Imagine if you not only had cash reserves
for three months, but you also had investments that were paying you a
quarter of your income every month. This would make your cash reserves
last longer and give you more flexibility for your search for new
income.

While food storage is like savings, growing your own food is like having
revenue-bearing investments. Growing your own food gives you a large
measure of independence from the shifts of the economy and poor
governmental policies. What’s better, it doesn’t take a lot of space to
get started. Even if you
live in an apartment,
you have enough space to accomplish some amount of food
production.

If you haven’t grown your own food before, let me give you a few tips
for starting a
garden. First, gardening requires patience. If you are used to
television and video games, results from gardening seem very slow, as
many crops can take two to four months to grow. However, realize that
the ability to slow down to your garden’s pace is good for your soul.
If you have the attention span of a fruit fly, I would suggest starting
with radishes. Whether you like them or not, you can often get results
in almost exactly one month. Lettuce is also easy and comes up pretty
fast.

Second, use raised beds, as they are simpler to manage. Raised beds
don’t have to be complex. Mine are just a few pieces of wood nailed
together. They don’t even need to have bottoms. They just need to make
sure that the soil level is higher than the surrounding area. This does
two things. It marks where you are growing, and it makes sure that the
water from the soil drains well. It is much easier to add water to a
garden than it is to remove it. Therefore, having the garden raised
makes it much easier to keep everything healthily cared for. 

Third, grow leafy vegetables. Leafy vegetables are a lot easier to care
for than fruiting vegetables like tomatoes and cucumbers and squash.
Growing lettuce consists of seeding the ground, watering a few times,
and then waiting forty-five days. Eat it and then start over. Some
leafy vegetables can be cut several times before replanting, though the
plants usually get more bitter tasting with each cut. 

My final tip is a little more controversial—wide-row planting. Wide-row
planting is really just a euphemism for massively overplanting your
vegetables. With the exception of some particularly claustrophobic
plants, most of the time, especially with leafy vegetables, the
recommended spacing on the package has nothing to do with reality.
Plant as many as you could possibly imagine fitting in the space, and
maybe a few more just for spares. I actually just cover the area with
seeds. Remember, it’s much easier to pull them out if they get crowded
than to plant more later if it looks like they don’t need the space.
And with most plants, you can use even the sprouts as sandwich
garnishes when you thin them out.

Once you have a successful year or two growing, you will likely start to
branch out and try other vegetables and fruits. Just go where the wind
takes you. Gardening is a great activity, as it connects you with the
natural growth cycles of living things. As you expand your projects,
you will be amazed at just how much can be grown in a small yard. The
Dervaes family in Los Angeles managed to grow 6,000 pounds per year of
produce on one-tenth of an acre!  You and I will probably never get
that much food, but this lets you know how far you can take it. Most of
us mere mortals probably get about a half a pound of produce per square
foot, which is about a third of what
the Dervaeses are
producing.

Producing food isn’t just restricted to gardening. Depending on city
codes, certain types of livestock can be raised in your backyard. We
love our chickens. We have four, and they are perfect pets for our
children. They live in
their coop summer and winter. They are incredibly easy to keep, they
eat our leftovers, and they produce wonderful eggs. If you don’t know
what breed to get, let me suggest a Rhode Island Red. They produce well
and can also be used for meat if necessary (our kids won’t let me use
our hens for meat, but oh well). You can raise other livestock, but
chickens are probably the easiest and have the fewest setup costs.

As mentioned in the chapters on money and value, one of the best ways to
secede from the money economy is to transform your free time by trading
consumptive past-times for productive ones. The most effective way of
doing that is to take up gardening. The benefits are almost immediate
and don’t rely on anyone else to come to fruition. If you live in an
apartment and don’t have the luxury of a yard, I recommend you read the
book \textit{Fresh Food from Small Spaces}. It has numerous ideas for
what you can grow even in the smallest of areas.

\begin{infonote}[Get Started Growing Today!]
If you have no garden prepared, no space set aside, no nothing, I am
going to tell you how you can start growing food \textit{today}. Mung
bean sprouts (you know, the little bean sprouts that are often in
stir-fry dishes) can be grown in a Mason jar sitting on your counter in
three days. Growing them is simple. First, you need to build a sprouter
from a Mason jar. A sprouter is a Mason jar with the lid replaced by a
screen. Just buy a wide-mouth quart Mason jar with a screw-on,
two-piece lid, then replace the lid itself with a piece of screening
(like from a screen door or window) that is a few inches larger than
the mouth of the Mason jar, and screw it on with the lid ring.
\textit{Voila!}  You now have a sprouter.

Now, go to a whole food store and buy a pound of mung bean seeds. Mung
bean seeds look a lot like peas but are sold dry and hard. You actually
only need two tablespoons of them, but buy a whole pound because you’ll
want to keep doing this. Next, put two tablespoons of mung beans in
your sprouter, and put the screening back on, securing it with the lid
ring. Fill the jar with lukewarm water until the beans are covered, and
let them sit there for twelve to fifteen hours. This will activate them
and let them know to come out of dormancy. After this, dump the water
out. This is why we have the screening over the mouth of the jar.  It
allows us to add and remove water very easily. From this point, every
eight to twelve hours you will want to rinse and drain your seeds.
Don’t leave them sitting in water, or they will mold. They only need
the residual moisture after they have been rinsed and drained. The
rinsing will keep the water from going stagnant and also make sure
there is enough. If you don’t drain out the water, the seeds can’t get
enough air to grow. 

If you keep the rinsing and draining schedule, in three to five days,
you will have your own mung beans!

I tell you this just to show you how simple growing your own food can
be. It can be elaborate, but it doesn’t have to be. It can be as
elaborate as you are comfortable with. In my own gardens, I try to
concentrate on growing things that don’t take a lot of work—and that
takes a lot of trial and error. When I lived in Illinois, tomatoes
would just spring up out of my yard. Here in Oklahoma, tomatoes are a
lot harder to grow because of our hot summers. However, I found that I
can grow chard, kale, and bok choi almost all year, even through the
winter.
\end{infonote}

\begin{policynote}[Neighborhoods and Personal Farming]
It is unfortunate that there are so many laws on the books regarding
what people grow on their front lawns. Many people have had the unhappy
experience of having their front-yard gardens removed by authorities
due to neighbor complaints. We should alter our city codes to
explicitly allow for growing food in our front yards. I can see
reasonable regulations on keeping your front yard from turning into an
unkempt jungle, but if someone is keeping up their garden, and growing
food in it, that is a benefit to the whole community. We should all be
doing that, not complaining about the people who are retaking control
of their food supply.

We should also lighten up laws regarding keeping livestock on our
property. Quiet animals should be allowed, and most limits on the
number of animals kept should be raised. In our city, we are limited to
having four chickens, but that’s really silly. There is no reason we
couldn’t have ten chickens without problems. One dog eats more food,
takes more space, requires more effort, makes more noise, and causes
more problems than my four chickens.

Finally, we should encourage community gardens. There is probably no
better means of creating community than gardening. It puts us
simultaneously in touch with one another, with nature, and with God. It
improves our diets and makes us more active. Community gardens can
rebuild the neighborhood spirit that has been lost in an age of cell
phones and video games.
\end{policynote}

\section{Wildcrafting Food}

“Wildcrafting” is learning to find and gather herbs and food directly
from nature. We tend to think of food as something we grow
intentionally, but really food grows all around us and we hardly even
see it. Every cultivated vegetable began as a wild plant. Most animals
get most of their food
in the wild from
non-cultivated plants. We have developed agriculture, but it is good to
know that we can eat even if we don’t grow anything. It may not taste
great, but it is certainly wholesome food.

In modern society we too often view nature as the enemy rather than an
ally. Food isn’t a special manmade add-on to nature. Food exists
everywhere. We cultivate it to make it more plentiful, more
predictable, and more tasty, but nature abundantly supplies food to us
if we just look. Not only can it supply food, it can supply other
things as
well, such as rope
and building materials. When we open our eyes, we can see that as long
as there are growing things and we have our freedom, we can never be
poor.

So, Back to wildcrafting, what is the enemy of lawn owners everywhere? 
What is the bane of most gardens? What grows so prolifically that we
have to spray it with chemicals to keep it from taking over our entire
yard?  That’s right—dandelions. What most people don’t know is that
dandelions are one of the healthiest foods available!  In fact,
dandelions aren’t even native to North America. They were brought here
by the pilgrims for food and medicine. Yes, dandelions are so healthy
for you that the pilgrims viewed them as medicine.

Now, you might not want to eat the dandelions in your yard, but that’s
your fault, not the dandelions’. If you’ve sprayed your yard with
countless chemicals, it might not be the best idea to eat from it
unless you’ve investigated what you are spraying. But that’s just the
point—food food is so plentiful that we have to spray chemicals to keep
food from taking over our lawns!

Sometimes in the summer I will ride my bike down a trail at lunchtime.
If I haven’t eaten and get hungry, I know that I will still be well-fed
on the trail. Why?  Well, the trail I ride is covered with wild garlic!
 Now, garlic isn’t everybody’s first choice for a snack, but the wild
variety is actually better-tasting by itself than cultivated garlic. In
any case, where I ride my bike, there is a near-infinite supply of
garlic growing along the trail. If I’m hungry, I just reach down, pull
up a plant, dust it off, and then chomp on it while I’m riding. It may
sound strange, but there’s really nothing more natural.

I think it is good to learn and to teach our children how to get what
they need from nature. Things like learning how to start fires with
flint and steel, how to make rope from literally anything (like grass),
how to hunt, fish, and trap, are all things that can teach us—even if
we never have to use these skills—that the earth is not the scary place
that other people make it out to be. We can have nothing and still have
everything. Just remember to never eat a plant unless you have
positively identified it and know that it is safe.

\begin{policynote}[Growing Food in Public Spaces]
One thing I notice when I’m driving is just how much space is given to
growing grass, which is totally useless. There is grass on the medians,
next to the sidewalks, around buildings, and throughout neighborhoods.
This isn’t wild grass. This is actually cultivated and taken care of by
the city. That’s right, our municipalities are paying to propagate
inedible weeds, and they don’t even let us have goats to help out.

What would it do for our communities if we replaced the grass with
something edible like lettuce or thyme?  What if we replaced the trees
with pear trees and the bushes with blackberry bushes?  Wouldn’t it be
great if, while going into a city building, we grabbed an apple from
the tree on the way in?  This can only happen if you go out and make it
happen. Talk to your city counselors. If there are practical problems,
work with them to see if there are places you can start. This whole
book is about small, simple ways to start a big idea. Find a small
thing you can do to help your city reclaim some of its land for food
production, and work to make it happen.

If we repurposed or dual-purposed our public land for public food
production, it would go a long way to making the whole community
independent. Remember, food grows from seed that comes from the food
itself. So, if someone wanted to grow a potato, why shouldn’t they just
go to the nearest public potato patch and grab a potato to grow?  Why
not find a seeding lettuce plant and take a seed pod home and start
your own garden?  By making food plants available throughout the city,
we give people the means they need to make their own gardens.

When you do this, you transform your city’s upkeep budgets. Just like we
converted our own budgets from consumptive to productive hobbies, we
can transform our public sector budgets from expenses to investments
that benefit the whole community.
\end{policynote}

\begin{policynote}[The Decriminalization of Food]
In the book \textit{Everything I Want To Do Is Illegal}, Joel Salatin
discusses his struggles with the disturbing trend of criminalizing
food, largely done in the name of public health. Now, there are things
that are public health concerns. However, most “public health” issues
are nothing of the sort. For instance, the FDA has requirements for
selling eggs that are based on the amount of dirt on the eggshells.
That’s right: the requirements for eggs are based on the shells, which
you don’t eat, rather than the eggs, which you do. Although we have
four chickens, we consume a lot of eggs, so we get the remainder of our
eggs from friends who have chickens. They are generally clean, but
sometimes there is a little bit of dirt or chicken stuff on the egg.
Does it make it unsafe to eat?  Of course not. People have been eating
eggs like that for thousands of years, and it isn’t one bit unsafe.

Similarly, there is a current crackdown on milk. Apparently, raw milk is
almost as dangerous as heroine, given the way the government treats
anyone who dares to sell raw milk across state lines. Goodness
gracious. Our family only buys real, raw milk from the local dairy,
because it’s delicious.

We need to get rid of a ton of idiotic laws and ideas that restrict
commercial food production to the mega-companies. One of the easiest
sources of income for people who are in a tight bind is to grow,
process, and/or sell food. There is no reason that this shouldn’t be
done in someone’s kitchen rather than in a commercial kitchen or in a
commercial food-processing center. Under current laws, you can’t sell a
pie that you bake in your own oven. This is one of the most oppressive
laws, as this is precisely the kind of economic activity that can help
families become independent.

I understand that there does need to be some consumer protection, but
each consumer protection needs to be weighed against the real costs of
implementation and the degree to which the potentially bad practice is
actually happening. We shouldn’t be restricting what citizens do with
their own food based on what we fear they might do rather than what
they are actually doing, or what we fear the consequences may be in the
future versus what the consequences actually are today. In addition,
any restriction put in place should be periodically reviewed to see if
it is still worthwhile. Advances in technology can often make previous
rules invalid or even counterproductive.

Most of all, we need to stop viewing food as an industrial product and
start viewing it as a natural part of living. We should be comfortable
with food, with growing it, preparing it, serving it, and selling it.
There is nothing more natural than food, which means that most attempts
to criminalize it will be similarly unnatural.
\end{policynote}

\section{Resources}

The task of seceding from the food supply is simple to start, but if you
want to get deeper, there is almost no limit to how far you can take
it. I can’t even begin to cover most of it in a single chapter. My goal
is to simply alert you to the problem and give you the first steps
you need to get
started. The resources below will help you
dig deeper into a
wide variety of topics that will aid in food microsecession.

\begin{itemize}
\item 
Harrison, Kathy. Just In Case: How to be Self-Sufficient when the
Unexpected Happens. Storey Publishing, 2008. 
\item 
LDS Preparedness Manual, Handbook 2: Provident Living. 2012 Edition.
Available for free at http://www.ldsavow.com/
\item 
Ruppenthal, R. J. Fresh Food from Small Spaces: The Square-Inch
Gardener’s Guide to Year-Round Growing. Chelsea Green Publishing,
2008.
\item 
Madigan, Carleen. The Backyard Homestead: Produce all the food you need
on just a quarter acre!  Storey Publishing, 2009.
\item 
Soler, Ivette. The Edible Front Yard:  The Mow-Less, Grow-More Plan for
a Beautiful, Bountiful Garden. Timber Press, 2011.
\item 
Markham, Brett. Mini-Farming: Self-Sufficiency on 1/4 Acre. Skyhorse
Publishing, 2010.
\item 
Tullock, John. Pay Dirt: How to Make \$10,000 a Year from Your Backyard
Garden. Adams Media, 2010.
\item 
Salatin, Joel. Everything I Want to Do Is Illegal: War Stories from the
Local Food Front. Polyface, 2007.
%% http://lds.about.com/library/bl/faq/blcalculator.htm
\end{itemize}
