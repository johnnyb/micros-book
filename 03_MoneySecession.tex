\chapter{The First Steps to Money Secession}

We’ve learned why we need to secede from the money economy:
because almost half
of our money is confiscated by the government, and what is left is
being devalued daily through inflation. If almost half of your labor is
going to the government, can you really call yourself free?  If,
instead of paying taxes, the government conscripted you to work from
January until the beginning of June \textit{every year}, would you
consider yourself free?

All of this doesn’t
mean that we shouldn’t all help out—of course we should. While we might
not enjoy it, everyone understands the need to take time out to do jury
service. We all understand that some part of what we do needs to go to
support the whole of society. Unlike libertarians, I would contend that
there is nothing inherently wrong with the government requiring us to
pay taxes or give a certain amount of service. However, at some point,
service turns into servitude, and I think we are far past that point.
This is especially true given the amount of money the government is
printing. When the government prints money to cover its own activities,
that means it is directing not only the actions of the public sector
with its tax money, but directing the activities of the private sector
as well. Not only is that 57 percent that you are keeping going down in
value every year, but it is also being manipulated toward ends that the
government, not you, wants.  And this all assumes that there is no
financial manipulation or fraud occurring simultaneously.  If you add
in corruption in the banking industry, it is surprising that we are
even able to feed ourselves.

The only way for individuals, families, and communities to extricate
themselves of this mess is to be less dependent on money. This chapter
will focus on some fundamental steps we can take to reconceive the
notion of value to help us start making our futures more independent of
the national monetary policy.

\section{Step 1 to Money Secession - Think Differently About Value}

Does it take money to live a good life?  We should stop to ponder this
for a while. Everyone says that they know that money doesn’t buy
happiness, but is that really how we live? Let’s pretend for a moment
that you lose your job tomorrow. Are you freaked out?  If you are, then
you believe that money buys happiness. 

Money doesn’t buy happiness, but we live our lives in such a way that
it does. Living this way means that
if the money stops,
we get the opposite of
happiness plus major pain. 

To reorient your perspective, let me ask you this question: does your
family or household produce anything other than money?  To give you an
idea of what I am asking, consider these questions: Have you ever
entertained guests at your house?  Have you ever cooked a meal for
someone who was sick?  Have you ever made a home-made craft?  Have you
ever given professional advice for free?  If you answered “yes” to any
of these questions, then your household has generated non-monetary
value. 

We have been conditioned to think of value entirely in terms of money.
In fact, we have been conditioned to think of nearly everything in
terms of simplistic measurements and numbers. But value is about much
more than money—money is merely a medium for exchanging things of
value. Once we realize that the value we produce is quite independent
of the money we make, entirely new possibilities open up.

Thankfully, governmental policy regarding value is almost entirely done
based on money. Therefore, if we want to avoid the effects of bad
monetary policy, the best thing to do is to decrease the amount of
value we produce in money, and increase the non-monetary value that we
produce. This is a good mental shift anyway—focusing too much on price
makes us ignore real treasure when we find it—and the hazards of the
current economy make this shift absolutely necessary.
Let’s look at a few
examples. 

Start by thinking about yourself and your hobbies. Are they productive
or consumptive?  That is, when you engage in your hobbies, does it
increase or decrease the value for your family? 

If you own every DVD of every TV series made for the last ten years, are
you better off?  Really, you aren’t. You are a \textit{consumer} of
entertainment. On the other hand, if you were to write a script for a
play for your family to do together, you would be a \textit{producer}
of entertainment. The difference is important. How much did that
complete series set cost you?  How much more valuable would it be for
you and your family to learn the art of stagecraft? 

Sound impossible? Let’s
stop for a minute and think about this. Yes, this is but one example,
and everyone is gifted differently, but have you ever really thought
about trying something like this? Even if you’re the opposite of
creative genius, remember: you’re working with
\textit{children}.
Anyone can sit down with children and, within ten minutes, have
\textit{too}
many characters to write about. And who knows? If you become
well-practiced enough at writing plays, perhaps someone will one day
buy one from you!  This may be sounding unrealistic, but it is
certainly much more possible than the idea that someone might want to
spend money on you watching more television.

The larger point is not that watching TV is intrinsically bad but that
we tend to surround ourselves with consumptive rather than productive
activities. We pay large sums of money to strangers to entertain us and
leave us with nothing of value. What if, instead of seeking out lazy
forms of entertainment, we found something productive that we enjoyed? 
It doesn’t need to be highly productive. The point is that we have
several points at which we are oozing money, and if we stopped
consuming and replaced each of them with something that produces even a
little bit, we would wind up being a lot more ahead than we were. We
would not be spending the money; we would be building value (which is
always good); and that value wouldn’t be monetary (which, as we’ve
seen, increases its value compared to monetary assets by 75 percent).

I’ll give you a few quick examples from my own life. My children wanted
pets. There’s nothing wrong with pets. We previously had a Betta fish,
which is pretty inexpensive—mostly just requiring cleaning of her small
bowl. However, after the fish died, I decided that our next pet would
be productive. So we bought chickens. Now, getting set up for chickens
costs money. I managed to set up the coop relatively cheaply by just
fencing in the bottom part of the playset in our backyard with chicken
wire and a door. However, that money increased the productive value of
our household. Our pets are our companions—just like dogs and cats—but
these pets are also productive. We feed them; they give us eggs. They
would probably give us even more eggs if I knew anything about
chickens. But nonetheless, even though it is only a small gain in
productivity, it is much better than other pets, which cost more and
don’t produce anything. In addition, if we wanted to be more productive
in this endeavor, we could use the knowledge we’ve gained raising three
chickens to help us raise a whole flock if we wished.

Similarly, I took up gardening as a hobby. Not just gardening but
vegetable gardening. We actually grow some of the food we eat. Now,
just like with the chickens, if I knew anything about gardening, we
would probably grow a lot more. However, instead of a hobby that eats
value, such as anything requiring a membership or consumable assets,
gardening provides a source of value for our home. Since this is not a
monetary value, it is not taxed. Not only do we save the purchase price
of the vegetables we grow; we actually save another 75 percent because
we did it without participating in the money economy. If a tomato costs
\$1.50, then each tomato we grow saves us \$2.62. And we get better
tomatoes. And we didn’t waste the money on consumptive activities.

If you get really good at new productive hobbies, they might even turn
into a business. At that stage in the game, however, one is thrust back
into the money economy.  Remember, though, it{\textquotesingle}s not
bad to make money, it{\textquotesingle}s just bad to be dependent on
it.  In addition, a later section will provide some tips for converting
any money made in the money economy back to non-money assets. 

If you have children, then your family is certainly producing value.
However, most people view children as liabilities rather than
productive assets to the family. We also tend to treat them that way,
which leads to our children being obsessed with buying things rather
than helping others—or
producing anything. If we treat
children as if they
were money holes rather than productive contributors, chances are they
will act like it. If, instead, we treat them as partners in production,
we might be surprised to find out just how valuable children can be. 

Let me first
communicate clearly that children are worth
much more than their
potential productive output. Every child is valuable for the simple
reason that he or she is a person made in the image of God.  This is an
intrinsic value which cannot be bought or sold for any amount of money.
 But that{\textquotesingle}s not all.  Children can also function as
productive members of a  household, and contribute still more value to
their families and society.

Let’s start here. How
much would it cost for a maid to do your housework?  What if
instead you got your
children to do your housework?  In doing so, they are generating value
for you—non-money value. In such a non-money activity, you get the
value but don’t have to pay for the service. You also don’t have to pay
the tax on the service.
What’s even more
important, you don’t have to pay the tax on the money that you used to
pay the service. Remember, 43 percent of your income is going to the
government. If the maid charges \$100, you would have to make around
\$175 more income to pay the bill in order to also pay all of the
associated taxes.

In this example you
can see that if you train your child to be a producer for the household
rather than just a consumer, this can be a huge gain to your family,
and it comes tax free!

Children can produce monetarily as well, and I think it is important to
train them in this way. By teaching our children how to run their own
business, we can teach them how to find and create value in any
situation they are in. Can they make candy, jewelry, or crafts?  Body
care products such as lotions and soaps are easy to make and easy to
sell. Learning how to
do any of these things and then teaching your children how do them—and
setting them free to do so on their own—is a great way to allow them
to be productive while they are still
at home. This
productivity comes as a boon to both you and them. You can shift more
of the responsibility of paying for things they want to them, as well
as teach them a life skill that will be ever more important as they
grow older. And
they’ll have fun doing it.

\section{Public Policy Note - Child Labor}

While I agree that we should treat our children well, I think that most
child labor laws are ill-conceived and are focused on keeping our
children in a juvenile state. In history, even pre-teens have shown
themselves capable of leadership and innovation both in industry and
even in battle. 

Child labor laws, by denying our children early opportunities to
materially participate in society, both stunt their personal growth and
remove potentially highly productive low-cost assets from the economy.
I agree that there are excesses that should be warned against, but I
think that there are a number of children for whom the greatest
long-term gift would be to allow them to do real, meaningful work. Our
attempts to romanticize childhood have only backfired and left us with
a society filled with children who know neither the meaning nor the
value of such work.

Child labor laws were originally instituted, not to keep children from
performing productive work, but because children were being employed in
dangerous factories from sunrise to sunset. Children might
have
worked for years
without ever really seeing the light of day. In 1842, for instance,
Massachusetts instituted a law that \textit{limited} children to
ten-hour work days. That’s right, before this law passed, children were
regularly working more than 10 hour days in very poor working
conditions.

I think it is easy to see that, although there are very good and
important limits that we should set, we have swung incredibly far back
the other way and are impoverishing ourselves and infantilizing
ourselves by keeping children out of productive activities. We should
support reasonable reform in the other direction, to allow children to
become more responsible within their families. 

\section{Step 2 - Make Less Money}

The title of this section implies that we should make fewer dollars
every month. While I contend that this should be a long-term goal, it
is probably not a near-term reality. We all have bills that come due.
We have mortgages that need to be paid. We have to buy food. This is
fine for today’s present reality. What I want to change is not
today’s reality but
the way we imagine the future.

I know that I have long dreamed of having a six-figure income for my
family. As of yet, I have not achieved it. I thought that if I made
more money, we could live more independently. After much thought, I
believe that dream was mistaken. I should, instead, be dreaming of a
life where I earn less money, both because I need less money and
because I am producing significant value within my household.

Think about your house. Do you really need all that space?
(If you’re answer is
yes, the question that immediately follows is: do you really need all
that stuff?) Do you need the nice neighborhood that you live in?  If
you took all of the equity in your house, could you buy a smaller house
debt-free?

What percentage of your income goes to your house?  I know that for us,
just under 10 percent of our income goes to paying our mortgage. To be
sure, some of that is insurance and taxes. But a lot of it is money we
have to pay out every month just to stay put. It would be nice if we
could stay put for free. Then I could afford to work less—even more
than 10 percent less, because I wouldn’t have to pay income tax on the
money that I’m not spending. And what could I do with an extra 10
percent of time? I could generate productive value in my home! I might
even be able to take some time to learn to fix various household
appliances. Then, not only would my home increase in value; my wife
would like me a lot more, and that could land me benefits elsewhere, if
you know what I mean.

Now, I have not been able to downsize my family{\textquotesingle}s house
because we still have a very small amount of equity in our home. The
homes we would be able to buy on equity alone would simply not suffice
for me, my wife, and my children. Most of the houses you could get for
that price would require someone much more handy than myself to do
major renovations. The point, though, is that while most people look to
trade up, you should be looking to trade down, especially if you can
pay cash. Paying cash gets you substantially out of the money economy
for your house.

Ponder this—if you owe money on your house, and the economy slows down,
you might not be able to afford the mortgage. If the value of your
house goes down, you might not even be able to sell it to pay your
remaining debt. Given the massive amount of money printing that the
Federal Reserve is doing, wide swings in market prices are probably
going to be the norm, rather than the exception. If you have fully paid
for your house, then a large part of your family’s well-being and value
is completely independent from the money economy. Debt increases risk
and ties you down into the money economy, because debts—at least debts
to the bank—have to be paid back \textit{in money}.  Therefore, it
leaves you dependent on having a money stream to stay afloat.  We are
trying to get away from that, and make our lives more independent of
money.  This means minimizing or eliminating our debts.

Other major sources of repeated expenditures are cell phone service,
internet service, gasoline, food, and entertainment. For each of these,
you have to ask yourself—am I really getting the value I am paying for
each of these? While you ask yourself this, keep reminding yourself
that these things actually cost you 75 percent more than their price
tag when all taxes are factored in. Are they still worth it? If you
spent \$50 taking your family out to eat, it really cost you \$87.50.
Are you okay with that? How much time did you have to spend at work to
generate that amount of money? Did you really spend more than three
hours today working to buy a single meal?

This doesn’t mean we can never celebrate or relax with a meal out or go
to a movie. But remember, each of these things has a cost. That cost is
75 percent larger than its price tag, and you could have spent both the
time at work and the time it spent to spend that money doing something
productive. So, if you spend two hours working and one hour being
entertained, that’s two to three hours you could have spent being
productive. And, as I’ve said, “being productive” doesn’t have to mean
work. These can be our hobbies!  It is merely a matter of reorienting
ourselves to what is important and realizing that we are often much
better off unplugging from the money economy rather than succeeding in
it.

\section{Family Policy Note - The Two-Income Trap}

One of the easiest and most significant ways to reduce the amount of
money you make without losing value is to shift from a two-income to a
one-income family. Making this shift, though difficult monetarily, adds
an amazing amount of value to the family.

First of all, remember that 43 percent of what is earned will be lost by
taxes. Therefore, just to break even, the second earner must be 75
percent more productive value in their job than they are in their home.
Then there are
additional problems. Remember, you still have a home that needs taking
care of. You may have kids who need taking care of. You may have a
mother or aunt or uncle or friend who needs taking care of. If both
members of the family are working, then that means these other tasks
are not being done. Or, more probably, you are having to pay someone
else to do them!

So, not only do you have to be 75 percent more productive at your job
just to break even, you now have to pay someone to do the tasks you are
not doing. And they have to charge money on a scale such that they can
pay their taxes and take care of their family. If a family member needs
continual care, which is cheaper? For a family member to watch them for
free, or for you to earn enough money so that after taxes you can pay
another person (or perhaps multiple people) to watch them?  Even if you
hire someone else to take care of your own home and family, does that
really do the job?  You don’t really get to reclaim all of that time,
because it is still yours to manage. Instead, you wind up on a hamster
wheel, barely able to keep up with all you have to do.

Regarding taking care
of your parents and kids. If you pay someone else to do this, are they
getting a better or a worse deal?  Would my parents be better off if I
moved them into a nursing home (which costs money) or moved them in
with me (which costs nothing except their food)?  Would my child be
better off with a parent around to guide them morally and spiritually,
or are they better off in the care of a stranger?  This is the thing
about money; it isn’t equivalent with value, and there are some
valuables that it just can’t buy.

In addition, with a two-income family, you have less tolerance for risk.
If you have structured your life so that you only need one income, if
something terrible comes up, then you have extra room in your life to
take care of that catastrophe. When one of our children was in the
hospital for an extended period of time, I was glad that my wife was
available to be by his side. I can’t imagine if that had happened and
we were both committed to full-time jobs. First of all, for her to give
our child the care he needed, we would have had to suddenly cut our
income in half. Second, most people who work have made a commitment to
the people they work with, and suddenly leaving them isn’t much of an
option. By only having one of us work, our family has had the
flexibility to handle the disasters as they have arisen—and there have
been many in my family. In addition, when we needed help, the people
who were available to come and help us were other single-income
families, because they too had the flexibility in their lives that
comes from having a single-income family.

\section{Step 3 - Bank Your Gains}

Banking your gains is
one of the most important steps to seceding from the money economy. It
is probably impossible to secede \textit{entirely} from the money
economy, especially because property taxes have to be paid whether you
made money or not. In addition, there are probably going to be many
times when the value we produce within our homes can gain us even
greater value by putting it on the market in the money economy. These
things are not necessarily a problem for people trying to exit the
money economy. We just have to remember that at the end of the day, we
don’t want our lives and livelihoods to depend on whether the Federal
Reserve chairman, Congress, or the president make good decisions about
how the economy should work. The best way to do this is to “bank” your
gains.

If you have extra cash in the money economy, what should you do?  What I
have been doing is getting that value out of the money economy. There
are two main ways to do this—long-term value assets and capital assets.
A long-term value asset is something that holds its value over a long
period of time. With our consumptive habit, we have forgotten the value
of purchasing permanent things. Nearly everything we buy is
replaceable. Not only can it be replaced; it \textit{must} be replaced.

How many years do you think your iPhone is going to last? If it manages
to last more than four years, will there be any apps left that are
compatible with it? What about your car? How much is it worth after
100,000 miles? What about your lawnmower? Do you repair it or just buy
a new one when it breaks down?

We have gotten into the habit of replacing things when they break rather
than fixing them. In other words, we have treated our possessions as
value-less, and rather than supporting their value, we have simply
bought other things. As such, we now purchase things that have short
expiration dates, because we weren’t planning on keeping them long-term
anyway. 

I have to admit that, in this, I am probably worse than most other
people. I am a very cost-conscious person, but I have discovered that
it costs more to buy cheap. Walmart prices may be unbeatable, but the
value of buying something that breaks within days of buying it is
practically zero.  This fact makes most of what they have actually
overpriced. 

Let me ask you this question—do you own anything of value to hand down
to your children?  Or did you buy everything on sale just to have it
disintegrate in a few short years?  I know, I know. You were trying to
be sensible. You were trying to balance your budget. And certainly
there are things that we must do on a low budget that we might not want
to. We can’t buy the best of everything. However, we often have enough
money to splurge on something. Should it be an iPhone or genuine silver
silverware?  One will be dead in four years. The other you can hand
down to your children, and they can hand it down to their children.

I agree we should spend less and save more. But one of the best means of
saving is by having valuable assets. In other words, what we save
shouldn’t be money but rather things that hold their value over the
long term. Can you name me one stock whose inflation-adjusted price
will last two hundred years?  Can you name me one currency whose
inflation-adjusted price will last as long?  And if it did, what good
did it do you while you waited?

My suggestion is that when you get extra money, you bank it by
purchasing valuable long-term assets. What sort of long-term assets
should you buy? The cheapest and easiest is silver coins. A half-ounce
silver coin will cost you, at the time of this writing, about \$15; a
one-ounce coin will cost you \$30; and a government issued one-ounce
silver coin will cost you between \$35 and \$40. Not only are these
coins an excellent way to retain long-term value, but some of them are
absolutely beautiful. The Chinese “panda” coins are breathtaking to
look at.

Note that I am not suggesting you should buy silver because its value is
likely to shoot up at any moment. It might—it might not. I am also not
suggesting that you purchase silver exchange-traded funds. That would
largely defeat the purpose, as an exchange-traded fund is a money
asset, even though it is largely backed by silver. I am suggesting you
buy real, live, physical silver because it is an easy, non-monetary way
to retain value that can last for generations. You might ask, “What
about gold?”  Well, the same one-ounce gold coin will cost you over
\$1,600. If you have that kind of cash lying around, more power to you,
but I certainly don’t. 

Let’s say, for the sake of argument, that you do have an extra \$2,000.
Rather than investing it in the stock market or buying a gold coin that
will sit in a safe, you could purchase a real gold necklace for your
wife. This would not only make an excellent gift to your wife, but it
would be a lasting store of value for generations. The same could not
be said of a big-screen television. That value might last for fifteen
years, if you’re lucky, but I doubt it will be much more than scrap in
a generation or two. 

You could also bank your earnings by doing improvements to your house.
However, for it to truly be a valuable long-term asset, it needs to be
an improvement that will likely last as long as the house. Adding on a
new wing to your house to improve your square footage is certainly a
good way to do this. Interestingly, a \textit{lot} of people that I
know who made the transition from wage slavery to financial
independence did so by treating their home as a storehouse of value.
They purchased broken-down homes and then repaired them a piece at a
time as they had finances, until the home was truly new. This actually
combines two of the steps into one. They converted their hobbies into a
value-producing activity (improving their home) and stored the value in
a long-term asset (also their home). 

Note that this activity is different from “flipping.”  Flipping a house
usually means that the owner of the house wants a short-term asset
purchase with a fast, monetary turn-around. What I am talking about is
taking a long-term asset and increasing its value. This is a valuable
activity whether or not a sale takes place—and whether or not the home
increases or decreases in price on a monetary basis. There are
innumerable factors that determine the price of a home. You would never
want to flip a house in a falling market. However, improvements to
long-term assets that you own are always worthwhile, whether or not
they are converted to money at a later time. Whether the
house{\textquotesingle}s total monetary value is up or down, it is
certainly better off if you have added value than it would be if you
didn’t. That is the ultimate point: you want to take the long-term
asset you already have and make it store more value than it did
before.

The second way to bank your gain is to buy “capital assets.”  In
business, a capital asset is one that allows the company to produce
things, or produce things faster, or make life more efficient. We can
do this in the home too. We can bank gains from the money economy by
purchasing assets that add continual value to our home. Even if they
aren’t permanent like land, gold, and silver, capital assets have the
benefit of providing increased value to the home. 

For something to qualify as a capital asset purchase for your home, it
should be something that either makes it cheaper to live or something
that enables you to produce extra value, which can be used, given,
sold, or traded to others. My chicken coop was a capital asset – it
allows us to raise chickens for eggs. 

Last year, our family bought a small food dehydrator. This is a slightly
different kind of capital asset, because it helps convert short-term
value assets (i.e., food) into longer-term value assets (i.e., storable
food). It allows us to bank on short-term changes in price and convert
those into long-term stores of value. If there is a huge apple sale
mid-season, those apples aren’t going to last forever. However, using a
food dehydrator, the value from the apple can be saved for a year or
longer. Thus, while we might not be able to eat all of the apples
available when they are cheap, we can dehydrate them and save that
value for future use. This is a lot of work, mind you, but it can be
very effective if you have the time to do it. This can also be helpful
for food secession, which we will discuss in a later chapter.

While some kitchen
gadgetry, such as the
dehydrator, falls into this category, most of these items are junk or
frills. The cheap knife set at Target is neither a store of long-term
value nor a capital asset. It’s just an expense. That doesn’t mean you
can’t spend your money doing it. You just need to be aware of what you
are doing and how it affects the value of your home. On more of an edge
case, your nice new Keurig would only be a capital asset if it
\textit{replaces} your morning Starbucks run.

There are an endless
number of ways in which you can buy capital assets to increase your
household’s value. For instance, I have several friends with chronic
pain. One thing that alleviates chronic pain is a combination of
turmeric and ginger. However, there is really no good way to eat enough
of that stuff to make a difference, so loading them into gel caps makes
it more worthwhile. Purchasing a small capsule-filling machine makes
the process faster and therefore increases our production of these
little wonders. We don’t sell the pills, but nonetheless, such devices
increase the value we are producing.

When purchasing capital assets, it is easy to fall back into the money
economy. Capital assets can break, often need inputs to work, and have
other sorts of dependencies. Therefore, when choosing an asset to
purchase, it is important to know if it is readily serviceable. The
problem with plastics is that it usually takes a manufacturing facility
to make specialized parts for them. Metal-based assets can often be
fixed locally—sometimes even in your own garage. Also, plastic and
polymer parts are usually just trash after they are broken, while, even
if a metal part cannot be fixed, it can at least be recycled through
the scrap pile.  Another consideration is the asset{\textquotesingle}s
inner workings.  Simpler assets, for instance, just tend to need less
work.  When they do break, simpler assets are more often serviceable
than more complex ones.  

It is also important to keep in mind the supply chain for both the asset
and its maintenance and continuing operation.  Assets manufactured
locally (and thus require less of the world market) are less likely to
suffer from part production issues that arise due to bad federal
monetary policy.  If your asset requires parts and inputs that come
exclusively from other countries, you have dramatically increased your
risk.  Likewise, if your supplier is heavily leveraged in debt, then
they are more sensitive to changes in the money supply, and would be
unavailable to make non-monetary contracts should money cease to be
available.  

An example of these considerations is my lawn-mower. I used to have a
standard gas-powered push-mower. Largely through my own stupidity, I
broke it while trying to clean it. But then, at the hardware store, I
noticed a push-reel mower. For those of you addicted to high-power
machines, a push-reel mower is a hand-powered lawnmower. This thing is
basically just a handlebar, a blade, a gear, and two tires, all stuck
together. There’s almost nothing there to go wrong with it. If gas goes
up to \$20 a gallon, I can still mow my lawn just fine.  If it breaks,
it can probably be fairly simply welded back together. The lawn mower
isn’t \textit{quite} a capital asset, but I think you can see how the
lessons from the mower can help with the decisions of what kind of
thinking should go into purchasing capital assets for your home.

\section{Policy Note - Encouraging Micro-Local-Businesses}

FIXME - put stuff here
