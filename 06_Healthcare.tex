% This file was converted to LaTeX by Writer2LaTeX ver. 1.0.2
% see http://writer2latex.sourceforge.net for more info
\documentclass[letterpaper]{article}
\usepackage[utf8]{inputenc}
\usepackage[T1]{fontenc}
\usepackage[english]{babel}
\usepackage{amsmath}
\usepackage{amssymb,amsfonts,textcomp}
\usepackage{color}
\usepackage[top=0.5in,bottom=0.6in,left=1in,right=1in,includehead,head=0.5in,headsep=0.461in,includefoot,foot=0.4in,footskip=0.76100004in]{geometry}
\usepackage{array}
\usepackage{hhline}
\usepackage{hyperref}
\hypersetup{pdftex, colorlinks=true, linkcolor=blue, citecolor=blue, filecolor=blue, urlcolor=blue, pdftitle=}
% Footnote rule
\setlength{\skip\footins}{0.0469in}
\renewcommand\footnoterule{\vspace*{-0.0071in}\setlength\leftskip{0pt}\setlength\rightskip{0pt plus 1fil}\noindent\textcolor{black}{\rule{0.25\columnwidth}{0.0071in}}\vspace*{0.0398in}}
% Pages styles
\makeatletter
\newcommand\ps@Standard{
  \renewcommand\@oddhead{}
  \renewcommand\@evenhead{}
  \renewcommand\@oddfoot{}
  \renewcommand\@evenfoot{}
  \renewcommand\thepage{\arabic{page}}
}
\makeatother
\pagestyle{Standard}
\title{}
\begin{document}
\clearpage\setcounter{page}{1}\pagestyle{Standard}
{\centering\selectlanguage{english}\bfseries\color{black}
Seceding from the Health Care System
\par}

{\selectlanguage{english}\color{black}
We all know that our health care system, despite being one of the most
technologically advanced in human history, is somehow broken. It is
hard for us to put our finger on it, but it feels that even while
\textcolor[rgb]{0.32941177,0.5529412,0.83137256}{the system} tries to
help, it is somehow failing at its job. I think that there are two
separate problem areas within the health care system. The first is
financial, and the second is with the way health is viewed in modern
society.}

\section{Health Care’s Financial Problems}
{\selectlanguage{english}\color{black}
Most people don’t recognize the cost of their own health care. If you
ask them how much their health insurance costs, they usually say
something like, “I get it for free but pay \$100 a month to include my
family on the plan.”  The problem is that, just like the government
hides so many of its cost on the employer’s side of the paycheck, the
health care industry does the same. That is, your employer is charged
on a per-person basis, and therefore, you have to work that much harder
in order to earn that money
\textcolor[rgb]{0.32941177,0.5529412,0.83137256}{so that they can}
justify your employment and salary. You have to earn that money even if
you never see it in a paycheck.}

{\selectlanguage{english}\color{black}
Just how much is this cost?  For most families, the cost of insurance is
around \$15,000 \textit{per year}. Yes, if you have health insurance,
it is likely that you are paying \$15,000 per year or more for it, even
if they only take out \$100 from your paycheck. And with the coming
health care laws, there is going to be no way out of paying this.}

{\selectlanguage{english}\color{black}
Why does insurance cost so much?  There are several
factors.\textcolor[rgb]{0.32941177,0.5529412,0.83137256}{ The first of
which is that }health insurance covers nearly everything. Think about
your home insurance. Your home insurance probably covers fires, but if
you need a plumber, you are on your own. Imagine how much more they
would charge \textcolor[rgb]{0.32941177,0.5529412,0.83137256}{for}
insurance if it covered fixing toilets, changing light bulbs, changing
filters, broken appliances, and re-roofing your house every fifteen
years. }

{\selectlanguage{english}\color{black}
That’s not all. Let’s say for a moment that your home insurance did
cover calls to the plumber. How would that change your buying habits? 
If someone else were paying for the plumber, would you look for the
best price or just go straight to the absolute best?  Think about it
from the plumber’s side. If the buyer is no longer price conscious, why
would the plumber try to offer the best price? }

{\selectlanguage{english}\color{black}
If your home insurance is covering calls to the plumber, is there any
reason \textcolor[rgb]{0.32941177,0.5529412,0.83137256}{whatsoever
}that you would put off calling the plumber?  For most of us, there are
some issues that we might try to fix ourselves, or they may not bother
us for the time being. As such, we can save ourselves money by either
doing it ourselves or leaving it undone.
\textcolor[rgb]{0.32941177,0.5529412,0.83137256}{But if plumbing is
covered, w}hy plunge the toilet ourselves when we can call a
professional to do it for us?  Why leave a slow drain when we could
just call a phone number and have it fixed? }

{\selectlanguage{english}\color{black}
When price no longer matters to the consumer, the cost of delivering the
service goes up, almost automatically. As the price of health care
delivery goes up, the insurance companies simply compensate by charging
us more and more. There are only three real solutions to this problem:
insurance could become more unaffordable; insurance could stop covering
ordinary needs; or insurance companies could start dictating the care
that people receive. The last thing I want is for insurance companies
to dictate the care that people receive, but that is the most likely
consequence of an increasing reliance on health care insurance.}

{\selectlanguage{english}\color{black}
In addition, the over-reliance on health insurance means that the
perceived risk of not having it goes up. If people go to the doctor
more and more, even if they could avoid it, it counts as health care
costs. Therefore, it is easier for health insurance companies to scare
people into needing insurance by pointing to the high cost of health
care that is created by the large-scale reliance on insurance
companies. }

{\selectlanguage{english}\color{black}
That{\textquotesingle}s not the only way that people are scared into
health care. Most people don{\textquotesingle}t realize that the amount
of money that doctors and hospitals charge for their services is
completely bogus. By completely, I mean absolutely, 100 percent,
out-of-your-mind insane. I{\textquotesingle}m not complaining about
them being too high, but rather that they don{\textquotesingle}t
reflect any reality whatsoever. }

{\selectlanguage{english}\color{black}
Here{\textquotesingle}s what really happens in medical billing,
according to Dr. David Belk, who has exposed much of the idiocy in
medical pricing. There are many ways that a hospital might bill for a
service. One could imagine them billing on a per-day basis, on a
per-disease basis, or for specific services, or any
\textcolor[rgb]{0.32941177,0.5529412,0.83137256}{other number of ways
one might go about billing}. However, different insurance companies pay
for different things. Some pay per day, some pay for services, and some
pay for diseases.  But the hospital gives every insurance company the
same bill – with every imaginable service charged at the largest
imaginable rate – and each insurance company decides what it will pay
for based on its own policies and agreements with the hospital.}

{\selectlanguage{english}\color{black}
The problem comes in if you ever want to pay with your own money. Since
they hand the same bill to everyone, they will hand you the full bill.
However, since you aren{\textquotesingle}t an insurance company, you
don{\textquotesingle}t get to pick and choose what you will pay—you
just have to pay full price for everything. There are many cases where
an insurance company might only pay \$2,000 for a bill of \$20,000.
Obviously, if they are only getting paid \$2,000 for a service, it
can{\textquotesingle}t actually be costing them \$20,000. Therefore,
the perceived cost of health care goes through the roof, not because of
health care itself but because of the way it is being paid for. And the
actual cost of health care actually is bad if you try to pay for it
out-of-pocket.}

{\selectlanguage{english}\color{black}
The first step to secede from the health care industry is to secede from
health insurance companies.
\textit{“}\textit{\textcolor[rgb]{0.32941177,0.5529412,0.83137256}{Scandalous!”
}}\textcolor[rgb]{0.32941177,0.5529412,0.83137256}{you say. Not really.
}There is a new trend of health care companies that, quite simply, do
not take insurance at all. In Tulsa, my family goes to Grass Roots
Health Care.  They do not take insurance at this clinic, and instead,
their prices for basic services are posted on the wall. 
That{\textquotesingle}s
right\textcolor[rgb]{0.32941177,0.5529412,0.83137256}{! T}hey
don{\textquotesingle}t send you a mystery bill; they
don{\textquotesingle}t tell you after the fact that you owe them your
soul; they post their prices in the waiting room so that everyone knows
exactly how much they are paying.}

{\selectlanguage{english}\color{black}
And how much is an office visit?  Twenty-five dollars. It costs the same
amount of money as a co-pay to a clinic using insurance. In other
words, the only thing that the insurance company added to the equation
was an extra cost of doing business. It didn{\textquotesingle}t save me
a dime. This is also true with many prescription pills.
\textcolor[rgb]{0.32941177,0.5529412,0.83137256}{You might already know
this, but b}efore presenting your insurance card, you should ask the
pharmacy how much the generic version of the drug costs without
insurance. Often, especially at discount stores, it is much less than
the cost with the insurance.  The same is true of tests and scans. If
you can get the non-insurance rate, it is often less than what you are
paying with insurance.}

{\selectlanguage{english}\color{black}
\textcolor[rgb]{0.32941177,0.5529412,0.83137256}{I suppose, therefore,
that}\textit{\textcolor[rgb]{0.32941177,0.5529412,0.83137256}{ }}the
first step to seceding from the health care industry is to at least
pretend you don{\textquotesingle}t have insurance. Find doctors who
operate outside of the insurance business. Find a discount drugstore
(like Costco) and ask them about their rates without insurance. When
your doctor orders a test, find out if there are any private centers
that offer the test for individuals at a lower price. }

{\selectlanguage{english}\color{black}
The second step is a harder one to take. Given the state of health care
today, I can{\textquotesingle}t even recommend it directly, but I do
recommend we take a hard look at it. That is, getting rid of your
insurance altogether, especially employer-provided plans.
\textcolor[rgb]{0.32941177,0.5529412,0.83137256}{ }If we have taken the
first step (going to doctors/pharmacies where insurance
isn{\textquotesingle}t needed) this is a little easier, since we
aren{\textquotesingle}t depending on our insurance for daily care, only
catastrophic care.  Unfortunately, we are not allowed to get an
insurance policy for catastrophic care only, and soon we will not be
allowed to go without insurance at all.  But nonetheless, this may be
precisely what we need to do to fix the broken health care system.  It
is the health insurance companies themselves which are the ones
inflating the cost of uninsured medical care.  Rather than asking how
we can all be their customers, maybe we should extricate them from the
process altogether.}

{\selectlanguage{english}\color{black}
\ \ There is an additional problem with employer-provided plans, and it
is the same problem that we see with the government taxing you before
you receive your check.  These plans are true costs to you – these are
costs that you must work to pay off, but you are never billed for it
directly!  That is, if your employer is paying for your \$15,000-a-year
health plan, then you still have to work for that \$15,000, even if it
never shows up on your paycheck. That is an expense your employer
incurs on you personally, and therefore,
\textcolor[rgb]{0.32941177,0.5529412,0.83137256}{they
}\textcolor[rgb]{0.32941177,0.5529412,0.83137256}{have }to be paid by
your productivity. What would you do if you were given a \$15,000 raise
tomorrow?  I can guarantee you that there would be more money in your
own budget for health care!  And if we got rid of the insurance
companies, it would be a lot more affordable, too.}

{\selectlanguage{english}\color{black}
\ \ There have been other policy recommendations to replace health
insurance.  One of them is the Health Savings Account (HSA).  The idea
behind an HSA is that it works partially like a savings account and
partially like a health insurance plan.  Basically, you get a special
savings account when you are born, and periodically you contribute
money to this account, tax-free.  Then, you, as the patient, decide how
the money is spent.  Rather than dealing with a health insurance
company, you make your medical choices yourself based on how much money
is in the bank.  I think this is certainly a better option than health
insurance, and can lead to numerous public benefits if it forces the
medical industry to change their billing practices.  However, I am a
little leery of it for the same reasons I am leery of the money economy
– such savings would be based in money and tied to the value of that
money in savings.  Since the buying power of money is going down,
saving money in an HSA may mean that the money is worthless by the time
it is needed.}

{\selectlanguage{english}\color{black}
\ \ I think the best long-term way to handle this is to move health care
into a normal expense category, to pay for larger expenses out of
savings, and to rely on your community when your own resources are
exhausted.  Some people argue against this, saying, “we
can{\textquotesingle}t hold a church bake sale and expect to raise the
\$100,000 needed for cancer therapy!”  I would argue that if this were
true, it would also be true of insurance.  If you, as a community,
cannot afford to come together and help someone out with cancer therapy
as individuals, how does adding in the overhead of an insurance company
magically make this affordable?  However, most people saying this
don{\textquotesingle}t realize how much they are spending on health
insurance.  Once we realize that many of us will be carrying around an
extra \$15,000 in our pockets, it becomes much easier to see that, yes,
we can afford to come together and pay for these expenses as a
community, if only we have enough faith in ourselves, each other, and
God to do so.}

\section{Policy Note - The Patient Protection and Affordable Care Act}
{\selectlanguage{english}\color{black}
Like most legislation, the Patient Protection and Affordable Care Act
has a nice-sounding name but is
\textcolor[rgb]{0.32941177,0.5529412,0.83137256}{actually} problematic,
especially for those wanting to secede from the system. This act,
passed in 2008, actually forces every American to have insurance.
That{\textquotesingle}s right, this bill will force you to have a
mandatory payment of over \$15,000 every year (current estimates are
that it will be more like \$20,000 for a minimum plan). And, if you
choose not to pay, the government will tax you!  }

{\selectlanguage{english}\color{black}
As you can see, this doesn{\textquotesingle}t bode well for
microsecession for either money or health care. It will force you to
earn money to pay either the health care premium or the penalty tax,
which is just over \$2,000 per family. Note that choosing to pay the
\$2,000 tax doesn{\textquotesingle}t give you any health care. 
Nonetheless, it will probably still be the option that many people
choose, because \$20,000 is much more than most people can afford. 
This doesn{\textquotesingle}t affect corporate America too much,
because office jobs regularly come with health care anyway – we were
already paying this money out, though not as much.  The people who will
be hit the hardest are low-wage employees whose companies
don{\textquotesingle}t provide insurance, self-employed workers, and
small business startups.  These people will be taxed simply for not
having enough money to pay \$20,000 a year for insurance.  I should
also point out that the \$2,000 penalty tax would raise the tax rate
that we estimated for a family making \$60,000 per year by another 3
percent, bringing us to a tax rate of 46 percent. If you do buy
insurance, you are basically an indentured servant at that point.}

{\selectlanguage{english}\color{black}
The problem in public policy is that the debate is always, always,
always framed around how hard it is to get affordable insurance.
Instead, we need to be finding ways to get more doctors, pharmacies,
hospitals, and independent labs
\textcolor[rgb]{0.32941177,0.5529412,0.83137256}{to go }outside of the
health insurance system altogether. In addition, we need patent reforms
to make sure that drug and device companies don{\textquotesingle}t hold
on to their patents longer than they should so that generics and
copycats can enter the market in a timely fashion.}

\section[Policy Note {}- Another Way to Make Health Insurance
Affordable]{Policy Note - Another Way to Make Health Insurance
Affordable}
{\selectlanguage{english}\color{black}
\textcolor[rgb]{0.32941177,0.5529412,0.83137256}{Our national health
care policy needs radical new ideas – the old ones just
aren{\textquotesingle}t working.  So here{\textquotesingle}s one.  What
if we
}de-professionalize\textcolor[rgb]{0.32941177,0.5529412,0.83137256}{d}
large portions of
\textcolor[rgb]{0.32941177,0.5529412,0.83137256}{health care?} Doctors
are scarce resources, but unfortunately, they are mostly tied up
looking at runny noses and managing diabetes patients. This is not an
effective use of four years of medical school and four years of
residency training.}

{\selectlanguage{english}\color{black}
What most visits to the doctor actually require is someone to take a
look at you and tell you that you aren{\textquotesingle}t dying. When
my wife panics because a child has a bumpy rash, she
doesn{\textquotesingle}t need an in-depth examination by a
highly-trained doctor. She needs a caring, compassionate person to take
a look at it and tell her if it is serious or not. If someone gets a
yearly sinus infection, it doesn{\textquotesingle}t take a medical
doctor to know what is wrong when that time of year comes around again.
}

{\selectlanguage{english}\color{black}
What we need to do is to separate medical care into {\textquotedbl}basic
care{\textquotedbl} and {\textquotedbl}advanced care.{\textquotedbl} We
can create a specialized program for anyone to administer
{\textquotedbl}basic care{\textquotedbl} and provide it with a very low
liability limit (perhaps only forfeiting their license), so that no one
going into the field needs malpractice insurance. We then define what
{\textquotedbl}basic care{\textquotedbl} consists of and assert that,
if the training requirements are met (probably a two-year program), a
basic care specialist can perform any of the basic care tasks without
supervision. Thus, we can create an army of low-cost caregivers to let
us know when it is not serious and leave the medical doctors to examine
us when it is. In the long run, this should both lower costs for
patients (since most of their care doesn{\textquotesingle}t require a
medical doctor) and raise the paychecks for doctors (since most of
their money is now for more difficult cases).}

\section[Thinking Differently About Health]{Thinking Differently About
Health}
{\selectlanguage{english}\color{black}
So far we have covered some of the costs of medical care and why we may
need to secede from the health insurance industry. Now I want to
propose something
\textcolor[rgb]{0.32941177,0.5529412,0.83137256}{even} more
radical\textcolor[rgb]{0.32941177,0.5529412,0.83137256}{. Ready? M}aybe
we should secede from the health care industry itself.}

{\selectlanguage{english}\color{black}
\textcolor[rgb]{0.32941177,0.5529412,0.83137256}{As with other sections
}of this book, I am not arguing for total
secession\textcolor[rgb]{0.32941177,0.5529412,0.83137256}{,} but 
\textcolor[rgb]{0.32941177,0.5529412,0.83137256}{I am }questioning the
increasing role that\textcolor[rgb]{0.32941177,0.5529412,0.83137256}{,
in this case,} the health care industry plays in our daily lives. The
health care industry has become less and less about health and more and
more about disease management. There is an important difference. Real
health care looks primarily at the person as a whole individual and
helps the person maintain a healthy body throughout their life. Disease
management, on the other hand, looks primarily at the disease and
treats the person as a set of numbers that must be precisely
controlled. We will refer to the first way as \textit{holistic} health
care, since it looks at the whole person, and the second way a
\textit{reductionist} health care, since it reduces a person to just
their lab numbers and diseases.}

{\selectlanguage{english}\color{black}
These two perspectives on health are very different.
I{\textquotesingle}ve known many doctors who were perfectly willing to
make someone{\textquotesingle}s life worse in order to make sure that
their lab work looked better. My mom, for instance, lost the use of her
kidneys a long time ago and has been on dialysis for a number of years.
Some doctors use drugs to try to keep her lab numbers in line with a
normal human being’s. These doctors, by and large, tend to make her
sicker (if she follows their advice). Other doctors, however, recognize
the fact that \textit{she doesn{\textquotesingle}t have working
kidneys} and that having perfect labs just isn{\textquotesingle}t in
the cards.  These doctors focus on treatments that help her as a
person, whatever the labs may say.}

{\selectlanguage{english}\color{black}
Many times in medicine the treatment can be worse than the cure. I
remember when I first was on my own after college; I went to the doctor
because my knee was swelling. The knee wasn{\textquotesingle}t
bothering me–I just wanted to make sure there was nothing serious
\textcolor[rgb]{0.32941177,0.5529412,0.83137256}{going on}. I told my
doctor that I wanted to do whatever was in the long-term interests of
my knee. It wasn{\textquotesingle}t painful, and I was perfectly happy
with my knee being swelled up if that was the best long-term option. }

{\selectlanguage{english}\color{black}
The doctor suggested I get a cortisone shot. Even though I had just told
him I was interested in the long-term situation, I decided to ask him
directly if the cortisone shot would help my knee long-term. He said
no, it wouldn{\textquotesingle}t—it would just reduce swelling right
now but would probably degrade my knee overall for the long term. Why
would the doctor go directly against my wishes for long-term health? 
Because he is looking at me as a set of numbers. My knee is swelled,
and that{\textquotesingle}s not the right number to have, so he is
going to give me a shot to make the swelling smaller.
He{\textquotesingle}s not going to look at the broader impact of this
maneuver on my health.}

{\selectlanguage{english}\color{black}
Many doctors, when they get a patient, feel they must do something.
Since they are trained in prescribing pills, that is exactly what they
do.  In his book, \textit{Overdiagnosed}, Dr. Gilbert Welch makes the
case that many people are diagnosed with conditions which will never
harm them, but where the treatments themselves might.  Because of the
strong effect of medications, most modern medications have large
potentials for side-effects. What often happens is that the patient
simply substitutes one disease for another. }

{\selectlanguage{english}\color{black}
There is a story titled \textit{The King, the Mice, and the Cheese},
which illustrates this problem well. A king who loved cheese had a
problem with mice in his palace. To get rid of them, the
king{\textquotesingle}s servants got cats. But the king hated living
with the cats even worse than the mice, so the servants brought in
dogs. And again this repeated—they brought lions to rid themselves of
the dogs, and then elephants to get rid of the lions. What was done to
get rid of the elephants?  Well, they brought back the mice, and the
king learned to make peace with the mice, and the mice with the king.}

{\selectlanguage{english}\color{black}
Doesn{\textquotesingle}t this sound like modern disease management? 
Take this pill for that. Take this pill to counter the side-effects of
this one. That pill gave you something else?  Well, try this one
instead. With reductionist medicine, the cure can be worse than the
disease, and can actually cause disease.}

{\selectlanguage{english}\color{black}
 My first introduction to holistic medicine was Richard
Fleming{\textquotesingle}s book titled \textit{Stop Inflammation Now}.
My dad gave this book to me because he had used it to revitalize his
own health, and he hoped that he could help me out before I ran into
cardiovascular problems like he did. The interesting thing about the
book is that the treatment is not medical but food oriented. It says
that the foods we are eating are what are causing us to get sick. They
are causing our entire body to have an inflammation reaction, which is
causing many of our health problems. The suggestion of the book was not
pills or surgeries but a change in lifestyle–stop eating foods that
inflame the body\textcolor[rgb]{0.32941177,0.5529412,0.83137256}{!}}

{\selectlanguage{english}\color{black}
This was a revolutionary idea to me—that the foods you eat (and not just
their quantities) can directly affect medical outcomes. It turns out
that this is actually one of the oldest means of treatment available.
Hippocrates, who basically invented the medical profession, said,
{\textquotedbl}Let food be your medicine, and let medicine be your
food.{\textquotedbl}  }

{\selectlanguage{english}\color{black}
Food and plants are much more powerful than we usually give them credit
for. The reductionists who run the medical industry give us a little
panel on the side of each food item, which supposedly tells us
everything we need to know about a given food—how many calories it has
and what percentage of our daily nutrients it contains. According to
the reductionists, if we keep our calories and fats low but still get
100 percent of the U.S. recommended daily allowances (RDA) of each
nutrient, we will have eaten healthily and need not worry about
anything else.}

{\selectlanguage{english}\color{black}
Well, in short, this is total crap.}

{\selectlanguage{english}\color{black}
It{\textquotesingle}s not that we don{\textquotesingle}t need the things
that the RDA says. Rather, it is that the RDA is simply not even close
to the most important considerations of what we eat, and it ignores the
actions that the food itself performs in our body. Just like other
aspects of reductionist medicine, the RDA views food entirely in terms
of its component numbers rather than looking at the food as a whole. }

{\selectlanguage{english}\color{black}
Holistic medicine, on the other hand, recognizes that different foods
perform different functions within the body, and then recommends that
you eat foods that match what your body needs to be doing better. By
solving our health problems
\textcolor[rgb]{0.32941177,0.5529412,0.83137256}{through} focusing on
the foods we eat, we get smaller effects but longer-term results. In
addition, this mode of healing brings fewer side effects.  We tend to
acknowledge this intuitively, but fail to apply the knowledge
systematically.  For instance, most people know that coffee and tea are
stimulative, and eating too much turkey will make you sleepy.  We
recognize these because they are short-acting, strong effects. 
However, holistic medicine observes the way that plants and foods act
in the long-term to affect your health.}

{\selectlanguage{english}\color{black}
Not every malady can be solved this way, but many can. Reductionist
medicine, for instance, is good at fixing traumatic injuries, while
holistic medicine has very little to offer. If I sever my arm from my
body, no amount of good eating is going to fix that. But for minor or
chronic conditions, holistic approaches using food and plant-based
solutions are often more effective in the long term
\textcolor[rgb]{0.32941177,0.5529412,0.83137256}{and} with fewer
side-effects.  Rather than stacking up the number of pills we are
taking, holistic medicine suggests that we should be tuning what we eat
every day.}

{\selectlanguage{english}\color{black}
Many people throughout history have searched for a magical cure for
every disease. What would you say if I told you I knew of something
that \textcolor[rgb]{0.32941177,0.5529412,0.83137256}{can effectively
fight} cancer, infections, viruses, and nearly any condition
\textcolor[rgb]{0.32941177,0.5529412,0.83137256}{known to man}?  You
wouldn{\textquotesingle}t believe me?  Well, it turns out that there is
something that is this effective, and everybody has one—it is their own
body.\textcolor[rgb]{0.32941177,0.5529412,0.83137256}{ }One of the best
ways \textcolor[rgb]{0.32941177,0.5529412,0.83137256}{to treat a
disease is to} \textcolor[rgb]{0.32941177,0.5529412,0.83137256}{let}
your body do its own healing. Sometimes you may need to help it along,
but the best place to start is by strengthening your body and your
body{\textquotesingle}s systems so that they are in a better position
to work correctly. This is how many food and plant-based solutions
work—by strengthening your body so that it works better.}

\section{Getting Started in Holistic Care}
{\selectlanguage{english}\color{black}
There are many means of holistic treatment,
\textcolor[rgb]{0.32941177,0.5529412,0.83137256}{but }since we are just
starting out, we will focus on herbal medicine
\textcolor[rgb]{0.32941177,0.5529412,0.83137256}{(i.e., }using foods
and plants for
healing\textcolor[rgb]{0.32941177,0.5529412,0.83137256}{)}. Other
treatments are available, but I find that the herbal approach is both
more accessible and more effective for a wider
\textcolor[rgb]{0.32941177,0.5529412,0.83137256}{range} of people.}

{\selectlanguage{english}\color{black}
One nice thing about herbalism is that it is easy for the beginner but
almost limitless in depth. There are basic things that help just about
everybody, but there are additional considerations that can be brought
in on a per-person, per-problem basis. Eating better will help everyone
in the long term, but as you advance in knowledge, you can learn
techniques and recipes that will provide speedier and more specific
help. }

{\selectlanguage{english}\color{black}
The easiest way to start is to eat more plant-based foods. A common
dictum is this, {\textquotedbl}If it came from a plant, eat it. If it
was made in a plant, don{\textquotesingle}t eat it.{\textquotedbl}  Of
course, this is a gross oversimplification. First of all, with
genetically modified foods, we are blurring the distinction between
growing and manufacturing. I find it interesting that the sudden
increase in allergies over the recent years has been to foods that are
precisely the ones people are growing genetically—modified versions of,
like corn, wheat, and soy. The exception to this is the peanut, though
this may be due to the use of peanut products in vaccines. In addition,
having a small amount of meat in your diet is beneficial. However, we
should limit our meat to small amounts. Don{\textquotesingle}t eat a
large steak. Eat a little steak in a large salad, without salad
dressing. Eating this way will help to strengthen your whole body.  No
salad dressing, you ask?  Actually, there is very little need for salad
dressing if you include the right things in your salad!  There are a
lot of oily or sugary fruits and vegetables that can be added to take
care of the need for dressing.  For instance, a salad full of
strawberries or avocados taste just fine without dressing.  But, if you
really need salad dressing, you should make it yourself from real
ingredients, rather than eating manufactured salad dressing.}

{\selectlanguage{english}\color{black}
As a second step, I would recommend eating dandelions. Yes, those
dandelions. For dandelions, the whole plant is edible, nutritious, and
healing.  Dandelion seeds were actually brought to America on the
Mayflower so the pilgrims could grow them for food and medicine.  Most
herbalists start with dandelion because it is easy to use. Just pull it
out of the ground, wash it off, cut it up, and put it in a salad.
Dandelions are great because they stimulate your digestion (they
actually stimulate your entire secretory system), and digestion is one
of the keys of health. Think about it this way. Your digestive system
is the gateway by which everything else enters your body. Your
digestive systems needs to be able to fully digest food for nutrition,
let the right substances in, and keep the wrong substances out. If any
of these fail, it harms your entire body. Therefore, by stimulating and
strengthening your digestive system, dandelions can help to heal your
whole body by helping your body better process the food it receives. In
addition, dandelions can improve your mood. By improving your
digestion, your mood, and stimulating your body{\textquotesingle}s
secretory system, dandelions can be a great way to get started healing
through herbalism.}

{\selectlanguage{english}\color{black}
As a third step, you should make herbal tea. Herbs, while they work by
ingesting them normally, can often be better assimilated into your body
by preprocessing them to make them easier to digest. Two of the most
common ways of doing this are by making tea (basically, pouring hot
water over them and letting it sit for twenty minutes or an hour) or by
making a tincture (pouring alcohol over them and letting it sit for a
few weeks). Teas are easier and faster, so they are a good place to
start. You can improve the effectiveness of dandelions by making
dandelion tea. Just chop up the dandelions, pour near-boiling water
over them, and let the mixture sit for an hour. Strain out the
dandelions, and drink!  }

{\selectlanguage{english}\color{black}
What{\textquotesingle}s interesting is just how much you can do with
what is probably already in your spice cabinet. Cinnamon, rosemary,
thyme, ginger, turmeric, and many others have distinctive
health-related properties. You can make teas with most of these as
well. If you have frequent aches and pains, a great anti-inflammatory
pain reliever can be made by mixing a quarter teaspoon of ginger with a
quarter teaspoon of turmeric, and making a tea out of it.}

{\selectlanguage{english}\color{black}
You should be able to find teas and tinctures of just about any herb
\textcolor[rgb]{0.32941177,0.5529412,0.83137256}{you want at any health
foods store, plus information on how to use them}. However, I would
recommend against depending entirely on what the packaging says. The
FDA regulates the kinds of things that can go on a package, and so
sometimes important information and uses of herbs are missing from the
packaging. The best place for a beginner to go for that sort of
information is probably LearningHerbs.com.}

{\selectlanguage{english}\color{black}
There are many cases where the current health care system is still the
best choice. However, a key to freedom is having options. By
understanding how you can take care of your own health using food, you
will be less likely to be lured into problematic treatments from
reductionist doctors. 
\textcolor[rgb]{0.32941177,0.5529412,0.83137256}{Living a healthy
lifestyle can keep you out of the doctor’s office you don’t want to
visit in the first place.}}

\section{Note - The Placebo Effect}
{\selectlanguage{english}\color{black}
Many people think that herbalism can be entirely explained in terms of
the placebo effect. For those who don{\textquotesingle}t know, the
placebo effect is an observation that people can get measurably better
from medical conditions simply because they think they are being
treated (though there is some disagreement about whether they get
objectively better or just feel better). In any case, many people think
that herbal treatments are simply examples of the placebo effect in
practice. }

{\selectlanguage{english}\color{black}
Let{\textquotesingle}s pretend for a moment that this is true. If
\textcolor[rgb]{0.32941177,0.5529412,0.83137256}{it is}, why is
\textcolor[rgb]{0.32941177,0.5529412,0.83137256}{it} a problem?  If we
can treat, say, 5 percent of the population with placebo treatments,
with minimal side-effects, why shouldn{\textquotesingle}t we promote
this?  If the goal is health and not power or money or control, then it
seems that we should first see if someone{\textquotesingle}s problems
can be treated without drugs
\textcolor[rgb]{0.32941177,0.5529412,0.83137256}{that have} large,
damaging side effects first, even if the healthy effects come more from
the person{\textquotesingle}s mind than the treatment itself.}

{\selectlanguage{english}\color{black}
But really, many herbal solutions work very well.  In fact, many of the
drugs we use on a daily basis are actually synthesized or concentrated
forms of chemicals that were first found in plants. In other words, the
drug industry finds a plant with an effect that it likes, finds the
primary ingredient, and then either concentrates that ingredient from
the plant itself or tries to synthesize that ingredient.  This would
seem to argue against the idea that the plant was ineffective in the
first place. One could argue whether the plant or the drug is more
effective, but it would be difficult to argue that the plant is
\textit{ineffective}, since it is used as the basis for many of the
drugs themselves.  What the drug companies have done, though, is to
remove the nutritional value of the plant, increased its effect (which
will also increase its side effects), and removed all other effects,
which may have been acting in concert with the primary effect to make
the plant work better for health.  In “Why Plants are (Usually) Better
Than Drugs,” Dr. Andrew Weil points out that, for instance, the drugs
made from coca leaves can be used to cure constipation by stimulating
the digestive system, but the coca leaves themselves can cure both
constipation and diarrhea.  How does this work?  Well, the coca leaf
actually contains chemicals that both stimulate and inhibit your
digestive system, but which ones are active depend on the state of your
body.  So, as you can see, the physiology of whole plants give them
greater flexibility than single, isolated chemicals. }

{\selectlanguage{english}\color{black}
 Another reason that I generally prefer plants over drugs is because of
my belief in creation. I believe that God created the world as a
system. The plants carry nutrition and healing power in their totality.
By separating the plant into its chemicals, much of the holistic power
of the plant is lost. You might get a stronger individual effect, but
you may also miss smaller but important secondary effects that were
also in the design of the plant. There may even be benefits of plants
which are totally independent of their chemistry.  In any case, our
goal should be to understand the system better God made and to use it
better, not just assume that we must go around it. Certainly, there are
exceptional circumstances that require exceptional intervention, but I
think that when we have decided as a society that we need pills to be
whole, something is missing.}

\section{Policy Note - Keeping Herbalism Legal}
{\selectlanguage{english}\color{black}
There is a growing movement to criminalize or regulate herbal
medicine—sometimes even regulating the plants themselves!  Just like I
support the ability of a person to freely own guns to protect
themselves, I support the ability of a person to freely use herbal
remedies and plants to heal themselves and their families.
Unfortunately, there are a lot of people who want to
{\textquotedbl}protect{\textquotedbl} you from the potential to abuse
herbs. Just like everything in life, herbs and food can be misused and
overused. Coffee and tea are both deadly if abused. But fundamental to
a free society is that each person is competent to make
\textcolor[rgb]{0.32941177,0.5529412,0.83137256}{his} own choices about
\textcolor[rgb]{0.32941177,0.5529412,0.83137256}{his} own life,
especially when the choice only affects himself.  There are limits to
this as to every other freedom, but by default we should allow people
to exercise their own judgement.}

{\selectlanguage{english}\color{black}
In the European Union, herbs and herbal medicines are already being
banned or regulated. Licensing is starting to be required for many of
these. This is a tragedy. Herbal medicine is probably the best way that
people can start to take charge of their own health. Many people think
that drug companies are behind many of these policy decisions, because
it removes their primary competition.
\textcolor[rgb]{0.32941177,0.5529412,0.83137256}{Regardless},
\textcolor[rgb]{0.32941177,0.5529412,0.83137256}{isn’t it better to}
allow \textcolor[rgb]{0.32941177,0.5529412,0.83137256}{the people} to
grow, make, use, and sell their own healing concoctions to their family
and neighbors?  This removes a major cost
\textcolor[rgb]{0.32941177,0.5529412,0.83137256}{for individual
families }and can even replace
\textcolor[rgb]{0.32941177,0.5529412,0.83137256}{that}
\textcolor[rgb]{0.32941177,0.5529412,0.83137256}{cost} with a small
income stream for their family.}

\section[Resources]{Resources}
\begin{itemize}
\item {\selectlanguage{english}\color{black}
\textit{Stop Inflammation Now!} by Richard Fleming. If you are not quite
ready for herbal medicine, this book is still based on more typical
ideas of medicine but shows how foods can help or hurt your health.
This is an excellent starting point for using plants and food for
healing.}
\item {\selectlanguage{english}\color{black}
\textit{Learning Herbs} website by John Gallagher
(www.learningherbs.com). This website is a great introduction to
herbalism for people just getting started. It includes a video series
on supermarket herbalism and a number of other resources. If you wanted
to become even more advanced, the website provides a pay service,
herbmentor.com, which helps people become even more adept at using food
for healing.}
\item {\selectlanguage{english}\color{black}
\textit{The Herbal Home Remedy Book} by Joyce Wardwell. This book is
probably the best book for someone just getting started in herbal
health. It is primarily a practical book—giving herbs and concoctions
for help for a variety of conditions. However, it also contains stories
which introduce the reader to the character and spirit of herbal
medicine, which is missing in most introductory books.}
\end{itemize}

\bigskip
\end{document}
