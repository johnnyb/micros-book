\chapter{The Money Economy and Its Problems}

While there are innumerable areas of idiocy in the federal government
that we need to extricate ourselves from, it is the economic issues
that are most pressing
and immediate. The
others are no less important, but the economic problems are coming
sooner rather than later, and it is important to be prepared.  But
before we understand how to fix the problem, we need to look at some of
the foundations of money and taxes and how they work in the economy.

\section{The Money and Non-Money Economy}

The most noticeable area where the federal government impacts our lives
is in the economy. If legislators alter taxes, for better or worse,
that affects us deeply. If the Federal Reserve (which we will discuss
shortly) alters the money supply, that affects us too. If the
government has to borrow money to pay for its operation,
what do you know? That
has an effect on us also.

What makes the actions of Washington so deeply affect us?  The answer is
simple—money.  Not all economic activity involves money. This may be a
surprise to some. When we think of the word “economy,” we usually think
of money. However, the definition of the economy as given by Thomas
Sowell is that it is the “use of scarce resources that have alternative
uses.”  Therefore, even mundane activities such as cleaning and
gardening are economic
activities, as they
use scarce resources (time, land, supplies) that have alternative uses.
For example, I could
use my time to play basketball instead of gardening, or use my garden
to grow flowers rather than vegetables. When we use this definition, we
find that there is quite a bit of economic activity that does not deal
in money. And, similarly, most of this non-money economy is not
adversely affected by the actions of the national government. We will
deal with the non-money economy more in the next chapter. The purpose
of this chapter is to focus on the problems that are presently lurking
in the money economy.

If the government mismanages the value of the dollar, does that affect
the taste of the food in my refrigerator?  No, it doesn’t. How about
the enjoyment of my pets?  Nope, not that either. Does it affect my
love for my family?  No. Does it affect how well my garden grows?  No.
Does it affect how I worship God? No, not that either. It turns out
that when the government mismanages money, the only thing it affects
deeply are dollar-denominated transactions. It has almost no effect on
individual activity, barter exchanges, or the buying power of precious
metals (i.e., silver and gold). Therefore, moving more and more of our
lives away from the use of dollars will insulate us from the adverse
affects of bad money policy.

\section{A Short, Short Lesson on Economics}

Most people are baffled when it comes to the economy, but it is actually
quite simple. “Money” is simply something that can be used in a common
exchange of goods. Usually it is something that is lasting (so you can
store it without going bad), has some use (so that there is a floor to
its value), and can be easily dividable and countable (so that you can
have transactions of varied amounts). This is why precious metals such
as gold and silver have historically been used for money. They don’t go
bad, they can be used in jewelry, and you can easily weigh out
different amounts of them. Grain, for instance, is a bad source of
money, as it can easily rot. 

However, it is a common misconception that money is equivalent to value.
This is not at all true. The value in an economy is \textit{not} the
money but the production of the economy. For instance, let’s say that
we have a very small economy. Let’s say there are three of us living in
a village together. I produce 300 apples this year; you produce 300
oranges; and a friend of ours, we’ll call him George, produces 300
peaches. Now let’s say we use silver coins for exchange and that we
each value our oranges, apples, and peaches equivalently.
If we have 900 coins
available—one for each piece of fruit produced—we can say for this
example, though
oversimplified, that the price of a piece of fruit is one coin. 

Now let’s say that George produces a new growing method so that we can
each produce 600 pieces of fruit,
making 1,800 pieces
of fruit in the economy. Did we increase the amount of money?  No, we
did not. We increased \textit{production}. Since we only have 900
coins, it means that each coin now buys \textit{two} pieces of fruit
rather than one.

Let’s say that I store away a few extra coins every year. Next year,
instead of starting with the 300 coins I received from selling my
apples, I will start out with 400 coins. How many pieces of fruit will
this buy me? The answer is, quite simply, it depends on the amount of
production. Since I have 400 out of the 900 coins, it means that I can
purchase 4/9s of the economy’s production. But if we have a famine, and
there are fewer fruits produced, that may mean that even with more
money, I can’t purchase as much. I may be able to purchase more than
you or George, but that doesn’t necessarily mean a real increase in my
buying power from last year.

Now, let’s say that over time I starved myself until I saved up 850 of
the 900 coins. So next year, I try to starve out you and George. Well,
at that point, you and George would rather keep your fruit than have
the money. So, even though I have lots of “money,” it doesn’t buy
anything because you and George would rather have your products than
the money. You and George might decide to stop doing business with me
and instead do business with each other, using a different mode of
exchange, perhaps simply switching the coins we use.

From a different perspective, let’s say that George found a stash of
coins. Now, we have 9,000 coins in our economy, but we are back at
producing 300 pieces of fruit each per year. As with the earlier
example, George can’t introduce all of the money at once into the
economy because, at some point, we would prefer having the stuff over
the money. But having this large stash means that George will have a
greater buying power than any of us for the near future. It also means
that, when all 9,000 coins are in the economy, if we are not producing
any more than before, each piece of fruit will eventually cost 10 coins
rather than 1. 

Because he is the source of the cash infusion, George controls the
economy (because he is the originator of the currency) and
simultaneously reduces the value of all of the money the rest of us
have. Perhaps my wife and I wanted to eat like kings next year, so we
decided that this year we would sell more than we ate, and save the
coins. This should give us more buying power next year. However, if
George floods the market with his coins, it means each coin buys less.
So, even though we have the same number of coins in savings, each coin
buys substantially less. If the economy started off with 100 coins, and
we managed to save 50 of them, we would be wealthy indeed. But if
10,000 extra coins were introduced into the economy, then all of a
sudden our “wealth” of 50 coins would be practically irrelevant. If
George is infusing the economy with currency, he not only has control
(because he has more currency than anyone else) but it comes at the
cost of making anything the rest of us have saved worthless. 

However, if that infusion of cash had matched a corresponding increase
in production that matched the magnitude of the cash increase, the
purchasing power of those 50 coins would have stayed stable, because
each coin would represent the same amount of “stuff” available in the
economy.

As you can see, while money may be expressible as a quantity, it is only
the production of goods and services within the economy that makes it
worthwhile. When production slows or stops, the value of your money
goes down. If there is a sudden influx of cash that doesn’t correspond
to increased production, it makes the value of everybody’s money go
down. 

To summarize, we have
looked at two major forces that determine the value of money: the
amount of money available (more money makes each piece of money less
valuable) and the amount of production available (each unit of
production makes money more valuable). Another major force is the
velocity of money, which is how many times a single coin changes hands
within a year. If a coin changes hands twice in a year, then that is
the same value as two coins changing hands once a year.

Now, think about a larger economy where we have employers and employees.
If I contract with you to do work for \$1,000 per month, what does that
buy you?  The answer is: it depends on the production of the economy
and the amount of money in the system. If the economy produces fewer
goods this month, then that \$1,000 won’t buy you as much as it did
last month. If someone introduces a large amount of money into the
system, then, again, that \$1,000 won’t buy you as much. If the economy
produces more, and the supply of money stays constant, then that
\$1,000 will buy you more than it did last month. Likewise for savings.
If you save \$5,000 dollars, the value of that savings depends both on
the production of the economy and the amount of money in supply. If the
amount of money goes way up, it diminishes the value (as buying power)
of your savings without costing you a single penny.

When money is used for exchange, salaries, and contracts, the value of
those contracts depends heavily on both the production within the
economy and the management of the amount of money within the economy. I
should also point out that in our large, modern, complex economy, it
often takes months or years for such changes to work all the way
through the system. But these are the fundamentals that determine the
long-term result of economic activity and monetary policy.

\section{A Short, Short History of Modern Money}

So where does money in the United States come from?  It makes the most
sense to talk about money in relationship to its history. I won’t
pretend to give a full history here, but I will note some important
highlights that relate to our current situation.  

Gold and silver have traditionally been used for transactions. Gold has
been used as a metal to enable its owner to carry large amounts easily
(currently, the value of gold is well above \$1,600 per
\textit{ounce}), while silver (currently valued at around \$30 per
ounce) has been used for daily transactions. The problem, though, has
always been making sure that the silver or gold is pure and is of a
known weight—and, by extension, a known value. 

Therefore, governments have “coined” money. That is, they create coins
of a known weight and purity with a special stamp. If your money has a
given stamp, you know the value of your coin. Earlier in United States
history, we had what was known as “free coinage,” which means that any
private party could bring gold and silver bullion to the mint, which
would then produce coins for them. Thus, an easy mode of exchange was
created. The U.S. mint would mint coins out of silver and gold so that
the public could have trust in the value of the coins used for
exchange. This is a key point. In order for something to work
efficiently and effectively as money, the public needs to trust its
value.

Paper money started out as promissory notes. That means that the note
was a promise of something—such as gold or silver or something similar.
Therefore, a bank could hold silver reserves but issue promissory notes
for that silver (these are called “bank notes”). Since the promissory
notes could be easily exchanged for silver at the bank, the note had
the same value as the silver. 

Promissory notes can also be a promise of future payment even if you
don’t have the money
currently. Thus, you
could incur a debt by issuing promissory notes that were redeemable at
a later time. The U.S. Continental Congress financed the Revolutionary
War using promissory notes backed by the anticipation of tax revenues
when the war was won.

The problem with promissory notes is that they are not standardized.
Different banks hand out different notes with different potentials of
being redeemed. It may be difficult to redeem a bank note from Oklahoma
in New Jersey, for instance. Therefore, the U.S. decided to standardize
bank notes and created the Federal Reserve System. The point of the
Federal Reserve is to standardize the issuing of notes so that money
can travel much easier. You will notice that all of your dollar bills
have “Federal Reserve Note” at the top.  Unlike its name, the Federal
Reserve is not a part of the government at all.  It is actually a
private group of private banks, though the board of directors is
selected by the president of the United States.

At the start, Federal Reserve Notes (hereafter referred to by the common
name of “dollars”) were redeemable for gold and silver. This led to
their widespread use both nationally and internationally. Since they
were readily redeemable for a fixed amount of gold, other countries
stocked supplies of dollars as well. Thus, the U.S. dollar became the
world’s “reserve currency”—the currency that was used for international
trade between nations, and the currency that other countries stockpiled
for future use. This was because the dollar had a stable value, and it
had a stable value because it was redeemable for gold.

Now, even though dollars were redeemable for gold didn’t mean that every
dollar represented actual gold in the treasury. Notes can be issued
based on other types of assets. However, being redeemable in gold means
that the Federal Reserve has to maintain a fairly solid correspondence
between the value of the assets that it holds and the issuing of money.
So, for instance, if I wanted to produce wheat, and I needed a \$70
loan, the Federal Reserve could give me dollars in exchange for my
debt. That is, they give me \$70 (a promise for \$70 in gold), which I
can then use for money. They receive a promise from me to repay the
loan. 

In order to make this loan, the Federal Reserve needs to ensure that the
value of its loan to me is \$70. Let’s say, for instance, that it made
100 such loans, but only 10 of them were paid back. This would be a
problem because there is now \$7,000 of money created, but only \$700
of it was paid back. However, the dollars that were printed are still
redeemable for gold. Therefore, the Federal Reserve has basically given
away \$6,300 in gold that will not be paid back. As long as the assets
of the Federal Reserve are reliably valued, however, it doesn’t matter
that they are not specifically precious metals. In fact, allowing other
forms of backing actually helps the economy, because it allows the
number of dollars in circulation to expand and contract with
production. It can actually be the case that, if handled correctly, the
dollar could be more valuable than gold based on the exchange rate.

The beauty of this system is also that, if done correctly, it stabilizes
the value of the currency by increasing the amount of money when there
is an increase in production, and decreasing the amount of money when
there is a decrease in production. Therefore, the purchasing power of
the dollar should stay relatively flat.
For example, if I get
a loan for increasing production, that will also increase the amount of
money available within the economy at precisely the point where the
production is happening. It prevents people from merely discovering
money in their backyard and forces the economy to produce to increase
the money supply. Then, when fewer people are producing (and therefore
fewer are taking out loans), the amount of money in the economy will go
down.

But again, all of this depends on the Federal Reserve making sure that
it has firm assets in exchange for money. If it has bad assets, then we
are just printing money and making everyone’s money worth much less.
When the dollar had a fixed exchange rate for gold, that gave at least
a minimum on the amount that it was worth. It also forced the Federal
Reserve to make sure that all of its assets were healthy. The problem
with a fixed rate was that, since some of the assets
were not gold, the
banking system could
not survive a “run” (i.e., if everyone tried to redeem their dollars
for gold at once, the banking system
could not handle it).

This is true of any enterprise. Apple Computer exchanges money for
iPhones at a fixed rate. However, if every dollar in the U.S. economy
was spent buying iPhones, Apple Computer would not be able to keep up
with the orders.

In 1933, in order to protect the banks from runs, Roosevelt made the
possession of gold money illegal, and therefore, there could be no run
on the banks for it. In a similar decision for similar reasons, in
1971, Nixon ended the international redeemability of the dollar into
gold. Therefore, the dollar was then only as worthwhile as the assets
that the Federal Reserve held for it. I should point out that in both
of these cases, it was actually bad monetary policy that led to the
need for these measures. However, both of these measures also caused
the way that we look at money to change; it changed our relationship
with the Federal Reserve and eventually caused further problematic
monetary problems.

In theory, the Federal Reserve should be printing money based on
production. In practice, the Federal Reserve has been used as a means
of countering bad economic policy. As mentioned earlier, the Federal
Reserve works when money is given to people producing value. This adds
money in direct return for production. When those things don’t work
out, the money is printed, but nothing of value is produced. As we saw
in the previous section, adding money without production simply causes
the price of everything to go up, and it hands the power over the
economy to those receiving the money.  

In recent years, the Federal Reserve has been used to buy bad assets
from banks. It has printed trillions (yes, \textit{trillions}) of
dollars to purchase these bad assets from banks. This has been termed
“quantitative easing,” but it really means “printing money to bail out
banks from bad loans.”  In addition to bad assets from banks, the
Federal Reserve has also purchased large amounts of U.S. governmental
debt—debt which the U.S. has never shown any hint of paying back.
Therefore, the increase of the supply of money has gone up but without
corresponding to increases in production. There have been small
increases in production, but nowhere near what is justifiable for the
increase in money supply. And, since the money is going to purchase
\textit{bad} assets, it is giving power (in the form of money) to those
people who are using it unwisely (the ones who created or purchased the
bad assets in the first place).  Likewise, it is also removing power
(by decreasing the value of saved money) from those people who are
using it wisely and
producing enough extra to actually have
savings.  So, not
only is the price of everything going up because of increases in the
money supply; the power is being shifted from those who spend money
wisely to those who spend money unwisely. This is not a policy that
leads to long-term benefits.

If you have wondered why you feel financially strapped despite the
increase in your income, this is why. While the amount of money you are
being paid may be increasing, the purchasing power of that same money
is steadily decreasing. Production will also probably be going down
because the money is flowing to those people who are using it unwisely
instead of going to those people who are producing value.

Not only has the situation gotten bad; it is only going to get worse.
The Federal Reserve announced in late 2012 that it will start printing
approximately \$85 billion \textit{every month}. It will spend \$40
billion on bad bank assets and \$45 billion on bad U.S. debt. In other
words, it will print \$85 billion dollars every month that correspond
to no increased production, and then give that money to the people
least likely to spend it wisely or productively. And it plans to do
that indefinitely.

Now, I’m not going to claim that I know where this is leading. I don’t.
I think the likely case is that this will lead to massive inflation in
the coming years. 
However, there are other factors at play as well, some of them going in
the opposite direction.  For instance, right now there are 100 trillion
dollars in outstanding loans. If those are paid back (a highly dubious
assumption), it will reduce the amount of money in circulation and could
lead to massive deflation if not offset by
something else.  In reality, we will probably get both. Loan repayments
will wipe out large amounts of money from the economy.  However, the
Federal Reserve will pump in large amounts of money, but into the wrong
(nonproductive) end of the economy.  

Rather than try to see the future correctly through some magical,
mathematical crystal ball, we should choose instead to become
independent of it.

\begin{policynote}[How \textit{Should} Money Be Created?]
I spent a lot of time being critical of the current economic system.
That leaves the question of how should we organize our monetary system?
Are there any near-term policies that could move us back onto the right path?
First of all, I should point out that any real solution will cause immediate 
short-term pain.  The reason for this is that a solution requires us to
admit that we have a problem, and our currency is propped up internationally
by other countries thinking that we don't have a problem.  As soon as we
publicly admit to the problems we face, it will cause immediate and large-scale
short-term trouble, with the length of the term dependent on how effective 
we are at fixing it.  

In any case, I think that a fairly simple change
can drastically affect the support of our currency without having to return
to the gold standard.  Currently, our monetary supply is fueled by debt.
Money is created only when someone borrows money.  This means that we
are banking on the fact that the person will do something smart with it.
In addition, we are creating more obligations than we are providing for
because of the interest.  Therefore, to at least balance this out, we
should shift from a debt-based currency to an asset/commodity based currency.
The Federal Reserve should be authorized to print money in exchange for 
commodities.  Individuals could participate by having any assets they
want converted into cash bundled into a commodity by local banks or
bundlers, and then sold in bulk to the federal reserve.  

This would allow
for a money supply which would be able to expand and contract based on 
the assets that people are willing to forego.  In other words, the money
would all be based on existing, hard assets, not on promises of future
payoffs.  This isn't a panacea, but I think it would be a good move
in the right direction, and would signal to the world that our currency
really means something.  Gold and silver would certainly be a part of this,
but there is no reason they must be the only or even the main commodity.
Any commodity which could be bundled and sold could play a part, and therefore
it could include anyone who can grow food or create goods in their home.
\end{policynote}

\section{How Much Are We Really Taxed?}

Note that all of this massive money printing is being done \textit{in
addition }to taxation. While bad monetary policy stealthily removes
wealth by decreasing the value of the dollar, taxation overtly removes
wealth by taking it directly. However, there are so many ways in which
we are taxed; it is sometimes hard to realize just how much of our
money is going into taxes.

Let’s say you are married with no kids, and you and your wife, combined,
make \$60,000 a year. How much do you pay in taxes?  First of all, it
is important to know \textit{which} taxes to count. We will count as a
tax any tax that is assessed on you as a person or that is assessed on
purchases you make. Therefore, sales tax is a tax that you pay, but the
corporate tax paid by the company you bought it from is not a tax that
you pay. Many people are under the mistaken assumption that their tax
rate is 15 percent. This is hogwash as we will soon see.

Let’s start at the beginning: your income tax. Because of exemptions and
the standard deduction, you get to shave off about \$20,000 from your
taxable earnings.
That leaves \$40,000 at a tax rate of 15 percent, or \$5,100 (the first
\$17,850 is taxed at 10\%). However, in addition, from your full,
non-deducted salary, you have to pay Social Security and Medicare,
which amounts to \$4,590. Then you have state taxes. My state charges a
5.5 percent income tax
rate. Using that
percentage as an example, you would have to pay an additional \$2,200.
This means you’re
paying \$11,890 in taxes on \$60,000—basically 20 percent of
your income. 

Unfortunately, we are
only getting started.

The government has hidden several additional taxes that you never even
see. They aren’t listed anywhere on your statement. These are called
“payroll taxes.”  There is a myth that these taxes are paid by your
employer. This is not the case. They are taxes on \textit{you }that are
paid by\textit{ you}. Think of it this way. If your employer has to pay
\$100 in taxes on
your work, that means that, in addition to you working enough to pay
your salary, overhead, and profit, you also have to work enough to pay
the \$100 tax. If you don’t earn enough for your company to cover their
expenses \textit{including the tax}, then you aren’t worth hiring. So,
you have to do the work in order to pay the tax. However, the tax is
paid before you ever even know about the money.

In the case of the \$60,000 salary, the employer side of the payroll tax
is usually comprised of Medicare, Social Security, the federal
unemployment tax, and the state unemployment tax. The payroll tax for
Social Security and Medicare is the same amount as the tax you paid on
it—\$4,590. This is \$4,590 that you earned but never showed up
anywhere on any of your statements. It went straight to the government.
Then you have unemployment insurance. This amount varies wildly by
state and company but can be as high as 9 percent. If your company is
being charged 9 percent unemployment, that amounts to an additional
\$5,400. In other words, the government is charging you a
\textit{hidden} fee of \$9,990 every year!  If you are self-employed,
then this isn't so hidden – it is called
self-employment tax.  But if you are working for an employer, they pay
it on you before you ever see it.  It is managed by the company as part
of your compensation, but you never get to see it.  Again, this is
money that is not reflected on any check, form, or pay stub that you
see. However, to be fair, we should also count this as income.

To make the math a little easier, we’ll round up \$10 to say that it is
a \$10,000 tax. Therefore, if your employer says you are making
\$60,000, you are actually making \$70,000, but you’re paying \$21,890
in taxes. We’re now at 31.3 percent of our income going to taxes. But
we’re not done yet.

Next we come to fixed asset taxes. Everyone who owns a home pays a
property tax. Where I live, a  home that costs around \$150,000 will generally get
charged \$2,000 per year in property tax. Then, if you and your wife
each drive a car, you have to pay registration on them, which, where I
live, adds up to about \$180. These only add modestly to our taxes,
bringing them to \$24,070.

Next we come to sales taxes. Some products, such as gasoline, have their
taxes baked in. For each gallon of gas that you buy, 50 cents goes to
some government entity. Depending on how much you drive, this could
wind up being about \$1,000 per year. Then you have sales taxes on
other items. The average sales tax rate is just under 10 percent.
Therefore, if you
spend \$30,000 buying groceries, clothes, household goods, and, of
course, taking your wife out on
dates, you will then
pay \$3,000 in tax on this amount.
We’re up to \$28,070
in taxes. 

Then come various
tolls. The most obvious one is the toll for road usage. This can be
anything from \$0 to \$2,000 just for commuting to and from work in
major cities.  There are other taxes you may or may not be paying. If
you smoke, depending on where you live, taxes can be upwards of \$4 per
pack, costing you another \$1,000 per year in taxes. Alcohol has its
own tax rates. Medical devices have additional taxes, which are going
up under Obamacare.

At the end of the day, if your salary says \$60,000, then you are
probably paying the government over \$30,000 every year. Adjust your
salary for the hidden amounts being funneled to Washington, and that’s
nearly 43 percent of your income going to the government.

Let me put it to you this way. If you work all year long, starting in
January, even if your “tax rate” is 15 percent, you won’t finish paying
off the money you owe the government until June. If you ever wonder why
it feels like no matter how much you work, you can never get ahead, now
you know the answer. Almost half of your year is owned by the
government. And this doesn’t even include forced purchases such as car
insurance, worker’s compensation, or, starting soon, healthcare.

Take this together with the previous conversation about printing money.
Not only are you paying the government almost half of your salary each
year; the government still can’t balance its budget and is printing
\$85 billion per month. They are taking almost half of your money and
then devaluing the remaining money you get to keep. At the current rate
they are printing money, this could cause inflation to go up ten or
twenty percent per year. 

They keep taking more, and it still isn’t enough. They have to print
money continuously to
keep up with their spending. This is insanity. This is why we have to
secede from the money economy.

\section{Resources}
\begin{itemize}
\item
\textit{How an Economy Grows and Why it Crashes} by Peter Schiff and Andrew Schiff.  
This is one of the best books on the economy for beginners.  It is an illustrated 
allegory following how an economy grows from three fishermen (Abel, Baker, and Charlie)
into a modern, full-scale economy.  This book is a fun and informative read, and is 
the best and easiest start to learning about how the economy works.
\item 
\textit{Basic Economics: A Common Sense Guide to the Economy} by Thomas
Sowell.  This is another great book on economics, focusing on how public policy
affects the economy.  It is littered with example after example from history
illustrating all of its points, and demonstrates over and over
the difference between the \textit{intentions} of economic policies and their
actual effects on the economy.  If you don't have time for a book of
this size, at least check out Sowell's
\textit{Economic Facts and Fallacies}, which covers some of the more
important topics in a concise manner.
\item
\textit{Wealth and Poverty} by George Gilder.  Gilder's book is a great 
re-thinking on the basis of capitalism.  Reading Schiff and Sowell 
helps to understand the basic workings of the market and
money---with Schiff focusing on production and Sowell focusing on incentives. 
Gilder, however, rethinks the basis of wealth and poverty in immaterial terms.
Gilder points out that faith and family structure has more to do wealth and
poverty than any starting distribution of assets.
\end{itemize}
