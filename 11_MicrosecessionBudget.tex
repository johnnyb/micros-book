
\chapter{Making a Microsecession Budget}


Now that we{\textquotesingle}ve covered the ideas behind microsecession,
we need to take a look into how we fit this into our daily lives. While
I will give a sample budget, I really can{\textquotesingle}t tell you
how much of a priority this should be for your family. I think that the
principles and practices of microsecession are wise in any economy and
situation. However, I wrote this book because I think that the
\textit{need} for these practices is coming upon us sooner than we
might want or expect.


How soon?  I{\textquotesingle}m not really sure. The economists I read
tend to put the window between 2013 and 2015. I{\textquotesingle}m
hoping for the latter number because I am still preparing, and I hope
that others read this book and themselves have time to prepare.
Nonetheless, I don{\textquotesingle}t think we should replace good
sense with hope. There may indeed be some urgency to this. The example
budget will assume that there isn{\textquotesingle}t a lot of spare
money each month. But, if there is, I would suggest getting started a
little bit quicker.


My suggestions are going to be very similar to many other systems of
saving and wealth-building but will be focused on the microsecession
implementation of them.
Keep in mind that
these steps are not meant to be done one at a time. You should start
step one before you start step two, but you shouldn{\textquotesingle}t
wait until step two is finished until you start step three. Each month,
you should consider whether you should add in the next step.

\section{Step 1: Evaluate Where You Consume and Produce}

The first thing to do for microsecession is to assess your life and your
family{\textquotesingle}s life. Where are you producing value, and
where are you consuming it?  If you look at your life like a boat,
where are your slow leaks, and where are you taking on water?  The
first thing to do is to reduce unnecessary consumption and replace any
consumptive habits that you can with productive ones. This is both a
long-term and a short-term step. While you will continually find new
ways to do this, you should first take a look and see if there is
anything you can reduce, cut off, or replace immediately.

\section{Step 2: An Emergency Supply}

If you follow Dave Ramsey, the get-out-of-debt specialist, you will know
that even before he has you pay off any debts, he tells you to create
an emergency fund. It{\textquotesingle}s \textit{going} to rain;
therefore, if you don{\textquotesingle}t have a rainy-day fund, you
will find yourself back in a hole. Similarly for microsecession, once
we have stopped the consumptive bleeding, the next step is to build up
a basic food supply. 


This can be done on whatever schedule you think is wise, but I would
suggest a minimum of buying one food storage bucket and filling it
every month. Label the outside of the bucket with the date and the
contents. Remember, you do need to eat from and use your storage supply
so that it doesn{\textquotesingle}t go bad, but I would wait until
after you had a three month supply to start that process. When you do
start rotating your supplies, eat from your oldest food first to make
sure your supply stays fresh.


As of this writing, food storage buckets cost around \$8, and a 25-pound
bag of rice costs about \$20; a similar bag of beans costs about \$40;
and non-GMO wheat costs about \$35. So, for something around \$40 a
month, you can buy food for storage and buckets to store them in. Go
back to the chapter on food supply to find out everything you need, but
with this budget, a family of five can be reasonably prepared with a
three-month supply in a little over two years.

\section{Step 3: Saving Long-Term
Value}

The next step is to start buying things with longer-term value. My
preference for this is silver coins, but the important thing is that it
is something that will hold value long-term and independently of the
state of paper money and the world. By long-term, I mean
multi-generational. We need to start looking at our purchases as things
to hand down.


I buy silver coins for my long-term value savings because they are so
easy and inexpensive to get into. As of this writing, a half-ounce
silver coin costs \$15, and a one-ounce coin costs \$30. So, for \$15 a
month, I can generate a legacy of long-term value that I can use in
emergencies or hand down to my children when they are grown.

\section{Step 4: Purchasing Capital Assets}

Once you have started implementing your savings and storage plans, you
are well on your way. Your home has become a storehouse of value. The
next step is to increase the amount of production that your home can
do. Some of this may be essentially free. You can grow a vegetable
garden in an existing space for almost nothing. However, producing more
and increasing the value of your production almost always requires at
least some capital assets. This is a difficult thing to price, as it
will depend on the specifics of what you are producing as a family.


However, if you budget to save \$50 a month for purchasing capital
assets, most home enterprises will be able to purchase some capital
asset at least every two to four months. We build our raised garden
beds for about \$35 each, make our chicken coop for about \$70 in added
material (we used an existing backyard play set), purchased our food
dehydrator for about \$100, and purchased our grain mill for about
\$250. 

\section{Step 5: Educate Yourself}

The education budget can be small, but it is important. Throughout this
book, I{\textquotesingle}ve put in suggestions for further reading.
This book can get you started, but you should really dive further into
each subject. In addition, microsecession is not the only subject you
should educate yourself on. We should all devote ourselves to being
lifelong learners.


Thankfully, education is fairly inexpensive. You can usually get away
with buying books for less than \$30. Depending on how fast you read
(or how much you like to), you probably don{\textquotesingle}t need to
spend more than \$30 each month. 


The monthly budget breakdown for all of the steps is as follows:

\begin{itemize}
\item 
Emergency Supply - \$40}
\item 
Long-term Value Savings - \$15}
\item {\selectlanguage{english}\color{black}
Capital Assets - \$50}
\item {\selectlanguage{english}\color{black}
Education - \$30}
\end{itemize}
{\selectlanguage{english}\color{black}
You can get started on all of the steps for \$135 a
month! If that is too
much for you, then pick what you can do and start there. If you can do
more, I would suggest investing more in steps one through three and
then shifting toward step four as your supply is built up.


Think of it this way—steps one, two, and three will help insulate you
from short-term problems, and steps four and five will help transform
your home into a sustainable long-term value producer.

\section{Note – Microsecession and Debt}

We have not discussed debt too much in this book. This is not because it
isn{\textquotesingle}t an important subject, but because I think it is
somewhat secondary to the important shift in transforming your home
from a consumptive to a productive mode of operation. Once that shift
occurs, debt is easier to eliminate because your mindset has changed. 


Having said that, it is certainly very important to eliminate debt. Debt
is a huge risk. It means that you have to pay a given amount of money
every month. What happens when the value of money changes?  What
happens when the availability of money changes, either through local or
national economic problems, or just because you lost your job?  All of
these things make debt very risky. You should get out of any debt you
can as soon as possible. Whether you get out of debt first or store up
your food supply first is a personal decision, but I think
I{\textquotesingle}d feel safer knowing that I have a good supply of
food backing me if anything were to happen. In any case,
don{\textquotesingle}t get into any \textit{more} debt!

\section{Note – The
Microsecession View of Purchasing}

It should be noted that, for microsecession, we should actually rethink
the way we make all purchases, not just the ones for the steps above.
First of all, we should focus on buying permanent things rather than
disposable things. Our family used cloth diapers for our children. Do
you know what we did with our cloth diapers when our children outgrew
them?  We used them for the next child. And when all of our children
were out of diapers, we gave the cloth diapers to friends.
Disposable diapers
last for three hours. Our cloth diapers have lasted for over a decade.
They are more expensive to purchase and a small amount of money to
maintain (you have to wash them), but the savings and value you get
from them is phenomenal. This is true for so many things in our lives.
Think about the containers we buy. Have you ever bought those cheap
plastic containers for leftovers?  After about ten dishwasher cycles,
they don{\textquotesingle}t really contain food anymore. On the other
hand, the glassware that we got at our wedding is still going strong. 


Along the same lines, we should opt to fix rather than replace when
possible. Similarly, when we purchase, we should focus on purchasing
things that are fixable. When I buy a car, I look under the hood. I
know nothing about cars, but I know if I look under the hood, and I
can{\textquotesingle}t see the ground, the car is probably too
complicated and packed in to fix easily. Fixing instead of replacing
helps save money and
change the way we think about the things we own. It makes us view the
world in terms of permanent, important, valuable things rather than in
terms of temporary, replaceable things. It helps us look at
something and say,
“How can I make this last longer?” Not, “How can I get rid of this and
get a new one?”  


This mentality helps us not only with our
stuff but
with our
relationships as well. Instead of looking to replace our friends when
we run into problems,
we need to think in
terms of fixing broken relationships rather than just cutting them
off.


We should view ourselves less as owners and more as stewards. What we
have will eventually be left to our children. We should handle it
carefully and attempt to return it in good condition. We should enjoy
what we have but also take care of it, and keep what good things God
has entrusted to us in good enough condition for the next generation.


Finally, we should purchase things that have the least number of
dependencies possible.
For example, we need
to make sure that we have equipment that remains useful even if the
power goes out, if the water mains break, or if the Internet is down.
We should be sure that our equipment doesn{\textquotesingle}t rely on
specialized parts that are difficult to attain. For instance, most
powered weed eaters have a “bump head” that allows you to bump the weed
eater so that it lets out more string. I was excited because my new
weed eater was “bumpless”—it would feed string automatically when
needed. That sounded fantastic until I realized that in order to
achieve this feat, the manufacturer had to create specially crafted and
balanced cord spools. Now I can{\textquotesingle}t buy in bulk.
I{\textquotesingle}m stuck with the specific cord the manufacturer
sells, and I just have to hope they don{\textquotesingle}t discontinue
it.

{\selectlanguage{english}\color{black}
To summarize, our purchases should be long-term, reusable, fixable, and
have few dependencies, and we should view ourselves as temporary
stewards of our possessions rather than owners.
