\chapter{Protecting Your Community}

The overall theme of this book is that our dependence on outsiders for
our daily living—whether for money, food, or health—leaves us very
vulnerable to bad decisions, or even bad luck. The best mitigation
against these risks is
to be sure that we can take care of ourselves and our families if the
time ever comes that we need to. Once we can provide for ourselves and
our families, it is much easier to help others out when they are in
times of difficulty as well. When we cannot provide even for ourselves,
helping out others becomes a difficult, and sometimes impossible, task.
While in other chapters we focused on separation from faulty systems,
this one will be more focused on supplementation.

\section{Understanding Your Responsibility to Your Community}

Protection is very similar to the other topics we have looked at. It
took me a while to wake up to this fact. We are used to depending on
the police for our protection and safety. In my community, the police
are wonderful. Public services of all kinds (police, fire, and even
utilities) come promptly when an emergency happens. It is easy to think
about this sort of thing as always being someone else’s job. 

However, I also know several people who live, or have lived, out in the
country. In the city and the suburbs, there is help all around. In the
country, communities
often can’t support a full-fledged fire department. So, instead, they
have a volunteer fire department. A friend of mine works as a volunteer
firefighter. It is tough, sometimes dangerous work for almost no
return—except that you live in a community that is safe from
out-of-control fires.

So, while many people may live in comfortable neighborhoods where a
simple phone call will give you instant access to any number of
services, we have to remember several important things. First, we have
to remember that it isn’t like that everywhere. If you support a policy
that depends on having ready access to public services, you are doing a
disservice to a large portion of the population who doesn’t. Second, we
have to remember that even though we have ready access to such things
\textit{today}, it is a very bad assumption to think that what life is
like today will continue forever. And it is really bad to base public
and social policy on this assumption. Third, we must remember that
while it is beneficial to have other people we can turn to, we can’t
use these other people as an excuse not to act when it is in our power
to do so.

If your neighbor’s house is burning down, you should definitely call the
fire department. But this doesn’t remove your own obligation to your
neighbor to make sure that they are safe and out of the house, and to
help them yourself if they need help. The same is true of protection.
In fact, if the economy deteriorates, the public’s ability to afford
police will be reduced, and we will need to protect ourselves and one
another. Some cities are already advising their citizens to prepare
themselves, because the available police protection is insufficient. 

What woke me up most out of my slumber to the realization that we needed
to protect ourselves and one another was the terrible massacre at Sandy
Hook Elementary. I realized that the reason why so many people died was
that there was no one there who was able to step up and help. It made
me think—if I were in a similar situation, what could I have done to
help?  What could I have done to stop the killing?  The police arrived
as fast as they could. The teachers and students did everything they
could. But there was no one there with the power to stop the killing.

It was at that point that I decided to learn how to use a firearm and to
get my concealed handgun license from the state. Prior to this, I had
never shot a firearm—not even once. But I decided that I would put the
time in to learn how to shoot, how to be safe, and how to be
responsible with a firearm. I decided that should the day ever come, I
didn’t want the reason that a killer wasn’t stopped to be because I had
left the responsibility of the safety of myself and my community to
other people.

\section{Christians and Guns}

Many Christians are uncomfortable with the idea of guns, and especially
of self-defense. I can understand this. Jesus made a lot of statements
like this: “But I tell you, do not resist an evil person. If someone
strikes you on the right cheek, turn to him the other also” (Matthew
5:39, NIV). This certainly wasn’t an isolated passage, and in Jesus’
example, He delivered Himself to die at the hands of his enemies. But,
unlike what some will tell you, neither Jesus nor the New Testament is
entirely against the ownership and use of weapons.

One of the more telling verses is this one, which Jesus said shortly
before He was taken to be crucified:

\begin{quote}
Then Jesus asked them, “When I sent you without purse, bag or sandals,
did you lack anything?”  “Nothing,” they answered.  He said to them,
“But now if you have a purse, take it, and also a bag; and if you don’t
have a sword, sell your cloak and buy one.  

(Luke 22:35-36, NASB)
\end{quote}

Jesus’ point was that the blessing that they normally enjoyed was not
always going to be there, and they needed to be ready, even to the
point of arming themselves.

Unfortunately, we don’t really know when or even if the disciples ever
used their armaments. It is fairly clear that they did not use them
against the Roman soldiers, but it is not clear that they would not
have used them against thugs or robbers who tried to jump them. Paul
tended to opt for escape rather than fighting. But nonetheless, it does
seem that Jesus at least allowed for the possibility that there would
be situations where arms were needed—even to the point where you might
prioritize it above other essentials.

Not only did Jesus once recommend buying a sword; nowhere in scripture
does Jesus ever ask someone to give up their arms, despite having many
opportunities to do so. Jesus often told people to give up their
money—even all of their money (as He did to the rich young ruler).
However, in none of His interactions with soldiers did He ever tell
them that they must give up their use of weapons. If Jesus were as
anti-weapons as some would have you believe, how come Jesus never
rebuked soldiers for carrying them or using them?

Similarly, when soldiers asked John the Baptist what they should do, he
said that they should not extort money, make false accusations, and be
content with their wages (Luke 3:14). Note that the word “extort” has
been translated as “do violence” in some translations (most notably
KJV), but this is actually a technical word used in known documents for
extortion. In addition, Paul references fairly explicitly the armed
power of the state as a legitimate use of force (Romans 13:4). 

All of this to say that the New Testament is not at all against the
legitimate use of arms for individuals or for the state. The larger
questions of love, mercy, justice, and self-sacrifice, however, are
deemed to be more important.

While I am not Catholic, I often look to the Catholic Church for insight
because they have a longer history of thinking through social issues
and the way that they integrate with Christianity. They are not always
right, but you can almost always find from them a well-reasoned
opinion.

The Catechism of the Catholic Church says, 

\begin{quote}
The legitimate defense of persons and societies is not an exception to
the prohibition against the murder of the innocent that constitutes
intentional killing. {\textquotedbl}The act of self-defense can have a
double effect: the preservation of one{\textquotesingle}s own life; and
the killing of the aggressor.... The one is intended, the other is
not.{\textquotedbl} [note - Quotation from the \textit{Summa
Theologica}]

Love toward oneself remains a fundamental principle of morality.
Therefore it is legitimate to insist on respect for
one{\textquotesingle}s own right to life. Someone who defends his life
is not guilty of murder even if he is forced to deal his aggressor a
lethal blow: 

If a man in self-defense uses more than necessary violence, it will be
unlawful: whereas if he repels force with moderation, his defense will
be lawful.... Nor is it necessary for salvation that a man omit the act
of moderate self-defense to avoid killing the other man, since one is
bound to take more care of one{\textquotesingle}s own life than of
another{\textquotesingle}s.

Legitimate defense can be not only a right but a grave duty for someone
responsible for another{\textquotesingle}s life. Preserving the common
good requires rendering the unjust aggressor unable to inflict harm. To
this end, those holding legitimate authority have the right to repel by
armed force aggressors against the civil community entrusted to their
charge.

(\textit{Catechism of the Catholic Church} 2263-2265) 
\end{quote}

I would say that this is largely my own opinion on the matter as well.
In American society, the ones charged with legitimate authority to
repel aggressors against the civil community
are not a third party
but we the people. The police power acts as a supplement to, not a
replacement of, individual authority within the United States.

\section{Real Pacifists vs. Pacifist Posers}

Having said all of
that, I must say that I have a special place in my heart for true
pacifists. By “true pacifist,” I mean someone who fully and
consistently rejects any form of violence for any reason. This is
primarily represented in America by the Quaker movement. An example of
this is Bayard Rustin, who, after being savagely beaten with a stick,
rather than fighting back, found another stick to offer to his attacker
and asked if the attacker wanted to beat him with that one, too. 

This kind of commitment to pacifism I find very respectable. These are
people who truly believe that violence is never, ever the answer. These
people certainly give me pause and make sure that what I am doing and
advocating is truly just and not a wonton exercise of power. One can
easily use the legitimate reasons for force as excuses for oppressive
violence, and we should always be vigilant about guarding our hearts
and minds against this. Those who practice pacifism help even those who
do not by helping us reshape our thinking, and call us into account
when we are doing wrong.

Unfortunately, however, there is another, much larger group, which I
will call Pacifist Posers. Pacifist Posers are those who use the
pretenses and moral authority of the pacifists, but without truly being
pacifist. A large example of the Pacifist Posers are those on the left
who disagreed with Bush’s war in Iraq because they said they believed
in nonviolence, but then were silent regarding Obama’s aggressions
against Libya. It’s one thing to agree or disagree with one war or
another. But what one cannot do is use pacifism as a pretext to
protesting a war when in fact it is just this particular war that you
disagree with.

Pacifist Posers have also arisen regarding gun rights in America. If one
is a pacifist, one can legitimately say that they believe everyone
should give up their weapons. However, if one is a pacifist, one cannot
legitimately say that they believe in a gun \textit{ban}. The
difference is subtle but important. A gun \textit{ban} means that the
force of law will be applied to remove the guns. This means that people
who disobey the law will be forced to comply—using guns. One can’t
claim to be a pacifist only to use someone else’s guns to enforce their
wishes. It is certainly a matter worthy of debate if any, some, or all
guns should be banned. However, one cannot be for a forcible gun ban on
the basis of pacifism. It is simply self-contradictory. A pacifist may
call for all men to voluntarily give up their arms, but he cannot do it
forcefully.

Along these lines, G.K. Chesterton warns us against congratulating
ourselves for having a given virtue just because we don’t have the
opposite fault. As Chesterton states in \textit{Conceit and
Caricature}:

\begin{quote}
Before we congratulate ourselves upon the absence of certain faults from
our nation or society, we ought to ask ourselves why it is that these
faults are absent. Are we without the fault because we have the
opposite virtue? Or are we without the fault because we have the
opposite fault? It is a good thing assuredly, to be innocent of any
excess; but let us be sure that we are not innocent of excess merely by
being guilty of defect... Let us then, by all means, be proud of the
virtues that we have not got; but let us not be too arrogant about the
virtues that we cannot help having. It may be that a man living on a
desert island has a right to congratulate himself upon the fact that he
can meditate at his ease. But he must not congratulate himself on the
fact that he is on a desert island, and at the same time congratulate
himself on the self-restraint he shows in not going to a ball every
night....

Some priggish little clerk will say, {\textquotedbl}I have reason to
congratulate myself that I am a civilised person, and not so
bloodthirsty as the Mad Mullah.{\textquotedbl} Somebody ought to say to
him, {\textquotedbl}A really good man would be less bloodthirsty than
the Mullah. But you are less bloodthirsty, not because you are more of
a good man, but because you are a great deal less of a man. You are not
bloodthirsty, not because you would spare your enemy, but because you
would run away from him.{\textquotedbl} 

[author’s note - the second paragraph precedes the first in the
original]
\end{quote}

Those Pacifist Posers who just want to delegate violent acts to the
state are often acting as cowards. The fact isn’t that they don’t
believe violence is the answer; it is that they want someone else to be
the one pulling the trigger. They aren’t against wars to end
aggression; they just don’t want to be the ones called up to do so. I
don’t fault anyone for cowering in the face of armed aggression, but I
do fault people who cloak their fears and cowardice into a false pride
of pretend pacifism. This can have dramatic results, even leading
eventually to despotism. Again, as Chesterton warns in \textit{The
Everlasting Man}, “As fatigue falls on a community, the citizens are
less inclined for that eternal vigilance which has truly been called
the price of liberty; and they prefer to arm only one single sentinel
to watch the city while they sleep.”  While pacifism can lead to
powerful social and political change, Pacifist Posers lead to
despotism.

While I do not believe in pacifism generally as a rule, I do think that
there are some—perhaps many—who are called to be pacifists. However, I
think that this is a special calling, similar to celibacy, which is for
some but not for all. We are all called to use our sexuality
righteously, whether in marriage or in celibacy. Likewise, we are
\textit{all} called to be peacemakers and to act justly. I think some,
however, are called to go beyond that and be true pacifists. Therefore,
in deciding what you should do about your own protection and the
protection of your community, you should spend time in thought and
prayer to help you decide what the best action is for you.

\section{Moral Hazards with Gun Ownership}

It should be well noted that with guns come not only rights and
responsibilities but also moral hazards. There are two main moral
hazards that come up when someone bears arms. The first one is fairly
limited because of its obviousness: firearms can give someone a false
sense of moral authority. Even though “might makes right” is certainly
not true, having physical power gives one an automatic sense that their
moral power is pure—or at least shouldn’t be interfered with. It can
lead to cockiness and swagger, both of which are immoral, and even
dangerous.

However, this hazard is fairly easily dealt with because it is so
obvious. It is easy to see yourself going down this road, and it is
easy enough to combat. It can actually be advantageous in a sense,
because it can make you more self-aware of your own moral behavior.

The second moral hazard is much more problematic because it isn’t so
readily apparent.  This hazard is that we might start believing that it
is the guns, and not God, that grant us our security. Psalm 127:1
states, “Unless the Lord builds the house, its builders labor in vain.
Unless the Lord watches over the city, the watchmen stand guard in
vain.” (NIV)  Depending on God for security doesn’t mean that we don’t
take our own efforts to secure ourselves. In fact, as you can see in
the first part of the verse, this even applies to building. Obviously,
the psalmist is not suggesting that we don’t use builders any more than
he is suggesting we don’t use weapons. The point is not that we can
cease to be diligent because the Lord is watching us, but that we
depend on God’s grace to be with us.

In the book of Judges, God sends Gideon to defeat the Midianites.
However, God told Gideon to send 31,700 of the men home and only fight
with 300. Why?  According to Judges 7:2, “The Lord said to Gideon, ‘You
have too many men. I cannot deliver Midian into their hands, or Israel
would boast against me, “My own strength has saved me.”’” (NIV)

God is concerned that when we fight in strength, we will think that it
is our strength of arms that has saved us. Certainly, God uses arms
just like God used the 300 armed men that Gideon took into battle with
him. But the power is not in the strength of those arms. It is in God.
Carrying guns can lead us to leaning on our own power and strength and
forgetting that it is God who takes care of us each and every day.

In a similar vein, if we have strength of arms, we may too easily forget
that God’s merciful ways of dealing with situations can actually be
more powerful than the weapons we have. God’s ways are not our ways,
and we cannot see the ends that God can see. God would have us choose
mercy over justice. We must choose mercy in the situations where we
can. There are many situations where that choice is not available, but
we should always remember the possibility. Making the choice to be
merciful can lead to your own death or injury, but as a Christian, my
own suffering or even death does not mark either as a failure or the
end. 

In Tacoma, Washington, for instance, when Brendan McKown heard the shots
of a mass shooter ring out, he headed toward the sound to stop the
violence.  When he got there, he drew his gun and ordered the shooter
to cease firing. McKown’s compassion on the shooter prevented him from
taking the shot before giving the shooter a warning, and instead of
heading the warning, the shooter opened fire on McKown, who, though he
survived, will probably never walk again. Interestingly, this turned
out better than it seems. After encountering McKown, the shooter
stopped the rampage and went on the defensive, eventually holing up in
a shop with hostages and then giving himself up to the police.  

The point is, God works in larger ways than we can think, and God
prefers mercy to justice, as He showed us by His son on the cross.
Therefore, while there are appropriate occasions for deadly force, as
people of Christ, we should look for opportunities to be merciful where
we can.

Finally, I want to point out that, for Christians, the world is not our
home. There are reasons to defend your life and perhaps even your
property, but there is a danger into looking at “my” life and “my”
stuff (and even “my” country or “my” rights) as if this world is where
we belong. We are ambassadors to this world, with the goal of bringing
life to others, not protecting our own position or role. Guns can cause
us to think about protection first and pouring ourselves out to others
second. For the Christian, the reverse should be true.

\section{The Social Benefit of Gun Ownership}

Just the same, however, we cannot use mercy as an excuse to not protect
the innocent. Having a firearm has saved lives and does indeed promote
order. Increased gun ownership nearly always reduces violent crimes.
For women, it reduces incidence of violent sexual crimes.  In his book
\textit{More Guns, Less Crime}, John Lott shows the numerous benefits
of gun ownership in deterring violent crime.  Gun ownership, on a
societal level, allows the weak to defend themselves from being
trampled by the strong.

Many people who think that guns should be banned usually rely on bad
statistical analysis. This involves everything from comparing two
countries on absolute numbers instead of crime rates (i.e., we are in a
big country with over 300,000,000 people, so smaller countries will
have lower absolute numbers automatically), focusing on gun violence
but excluding other sorts of violence
(e.g., does it really
matter if you are killed with a gun or a knife?), differences in
reporting criteria, and just outright idiocy with numbers. In addition,
even with crime rates, they are useless unless they give you
before-and-after pictures within the same country. Studies relying on
between-country numbers are unlikely to say anything of importance
because it ignores other factors that might play into the situation.

In nearly every country where gun ownership has increased, violent crime
has been \textit{reduced}. In the United States, concealed carry
laws—laws that allow law-abiding citizens to carry a firearm on their
person at all times—have reduced violent crime, with the greatest
reduction being in urban inner cities. 

With mass shootings, the difference is even more dramatic. Successful
mass shootings (ones in which there are four or more people killed)
almost always occur in gun-free zones. This is for two reasons. First
of all, killers actually target gun-free zones. The killer in Aurora,
Illinois, shot up neither the theater nearest to his house nor the
theater with the largest crowd. He picked the only theater in the area
(out of seven) that was a gun-free zone. Most people don’t know that
one of the Columbine High School killers heavily lobbied
his state legislators
to prevent passage of the Colorado concealed carry laws, especially the
part that
allowed people to
carry on school property. The fact is that mass shooters aren’t looking
for a fight. They are looking for killing, and they know that armed
citizens get in their way.

The other reason why mass shootings only occur in gun-free zones is that
in other locations, the killer is often stopped by legally armed
citizens before their massacre can qualify for the FBI definition of a
mass shooting.  As was the case with Brendan McKown, simply pointing a
gun at a mass shooter often causes him to rethink what he is doing. In
Portland, Oregon, there was a mass shooting at a gun-free mall.
However, one concealed carry permit holder apparently hadn’t noticed
the sign. When mass shootings began, Nick Meli pointed his gun at the
shooter but didn’t fire because there were bystanders behind the
shooter. The mass murderer then decided that, in the face of
resistance, killing himself was the better option, leaving only two
dead from a mass murder in a mall during the Christmas season.  This
doesn{\textquotesingle}t get reported as a mass shooting that was
stopped by an armed citizen, because the citizen managed to stop the
shooting before it qualified as one.

It is also good to note that while mass killers in America often opt for
guns, it is not the guns that cause the problem. In China, a gun-free
country, on the very same day as the Newtown shooting, a man broke into
a school building and stabbed twenty-three children. Twenty-three! 
While it is true that mass killers tend to choose guns in the United
States, it is false to presume that they would choose something other
than mass killing if guns were not available to them.

While mass shootings are what got me personally thinking about gun
ownership, they are not the only type of crime to be reduced by gun
ownership. All types of violent crimes are reduced by increased gun
ownership. As
previously mentioned, gun ownership by women creates an even more
dramatic drop in crimes against women than gun ownership by men does
for men. For women,
gun ownership closes the power gap that naturally exists between men
and women.  
As an example, Colorado State University changed their policy on student concealed 
carry on campus in 2003 to allow students to carry concealed on campus.  The next year saw an immediate, drastic decline
in sexual assaults.  Within five years, the number of sexual assaults
dropped by more than 90%.  
An old cliche says that, “God made men and women, but
Samuel Colt made them equal.”  Gun ownership, especially concealed
carry, closes the power gaps that make violent crime possible.  

\section{A Quick Case Study - The LA Riots}

In April of 1992, a jury acquitted four Los Angeles police officers of
brutality in their videotaped beating of Rodney King following a
high-speed chase. This verdict sparked public outrage, and violent
riots erupted throughout Los Angeles. Nearly every type of crime was
being committed during these riots, looting probably being the most
common, but assaults and murders were happening as well. The total
damage to the city was one billion dollars. 

While this destruction
was taking place, there was one area of town that did not experience
the looting and mayhem present in other establishments. The Korean
district was almost entirely at peace. Why?  Korean business owners
brought every type of weapon to their businesses, camped out on
rooftops, and prominently displayed their firearms to the mobs. If you
look on YouTube, you can see videos of this time period which show the
very dramatic difference in how this part of town fared from the rest
of the city. Where the Korean business owners patrolled their
businesses together with firearms, the city was at total peace. Where
there were no citizens with guns, it was total mayhem and chaos.

What was funny, though, was how the media reported on it. The media
seemed quite critical of the Koreans, saying that the Koreans were
looking for a fight. How ridiculous!  The Korean district was the only
area of town that was not fighting and not disputing. Everywhere else
in Los Angeles people were looking for fights. The law-abiding citizens
with guns were the ones looking to actively \textit{prevent} the fight.
And it worked.

\section{What You Should Do}

The issues involved with protection and gun ownership mean that there is
not a one-size-fits-all solution. If you feel that God is calling you
to be at peace with the world even if the world is at war with you and
one another, then that is what you should do. If you think you should
own or even carry a firearm, you should educate yourself about what
type of firearm is best for you and your family. You should also learn
about ammunition and the various decisions that go into what type of
ammo you put into your gun. You should think ahead and train yourself
for various situations you may find yourself in, and know ahead of time
what you are going to do. If the time comes, and if you decide to pull
the trigger, you can’t take it back. You should decide ahead of time
what your decision will be, because in a tight situation, you won’t
have time to go through the moral choices in your head. You will need
to already have your mind made up. But you should always remember that,
all other things being equal, mercy is preferred to justice. 

A short word of warning: if you are on antidepressants or any other kind
of psychiatric drug, even with a doctor’s prescription, please don’t
keep guns readily accessible to yourself. These drugs have the
potential to alter your view of reality, and, when you add guns into
the mix, this combination can be dangerous to yourself and even to
others. This is similarly true if you are a heavy drinker or use
recreational drugs. It is simply not worth the risk to have access to
firearms while your thinking is impaired.

\section{Public Policy Note - Gun Control Laws}

I am in favor of widespread gun ownership in order that people can
protect themselves, their families, and their communities. I don’t
believe in an unlimited right to gun ownership. I think that reasonable
laws can be made. However, I think that largely, the laws already on
the books are the reasonable ones. Most calls for gun control assume
that there is one, single situation in which one needs a firearm. For
instance, many argue that you don’t need certain types of rifles,
certain capacities of clips, or other similar restrictions. Most of
these ideas, however, presume that there is one idealized need for a
firearm. The needs of a rancher in the middle of nowhere might be
different than the needs of a suburbanite, whose needs might be
different still than the needs of a business owner. Someone in a rough
part of town probably has different needs than someone in an
upper-class area.

I think that most of these laws give very little benefit, while
restricting people from making choices that might benefit them in their
situation. Part of the point of microsecession is to be able to live
independently, and this also allows others to live independently from
you. It means allowing for others the same kinds freedom you might want
for yourself. As such, most gun laws are extremely short-sighted,
forgetting the great many people and situations present in this
country.

\section{Resources}
\begin{itemize}
\item 
\textit{More Guns, Less Crime: Understanding Crime and Gun
Control Laws} by John Lott.  This book dives into the numbers,
comparing violent crime statistics before and after the implementation
of gun control laws across many different states and nations.  Lott
points out that the states with the largest increases in gun ownership
also have the largest drops in violent crime.
\end{itemize}
