\chapter{The Principles of Microsecession}

Now that we{\textquotesingle}ve covered how microsecession works in a
variety of areas of our life, we need to step back for a moment and
look at the larger principles at work. Knowing these principles will
help us apply microsecession to other areas of life and give us fuller
liberty in everything we do.  This chapter will look deeper into the
reasoning that has guided our thinking to this point, and step back and
see what makes these ideas work together.

\section{Principle 1: Focus on Reality instead of Measurements and Indicators}

There are many measurements we use to indicate the status of reality. We
measure individual wealth by money or societal wealth by jobs. We
measure health by our weight
or our BMI. While all
of these are important indicators, we must be forever wary
of the idea that
these indicators are equivalent to the things themselves. My doctor can
show me perfect numbers, yet I still may be near death. If I lived in
Zimbabwe, I could have one hundred trillion dollars and still be poor.
(Zimbabwe actually printed one hundred trillion dollar bills. They sell
for about six dollars in the United States.)
This recognition that
measurements do not necessarily equal reality is the first principle of
microsecession.

There are three main keys to pursuing this principle. The first is to
always be a little bit skeptical of your indicators. If your gut tells
you that something isn{\textquotesingle}t right with a situation (such
as people on TV telling us that government overspending
isn{\textquotesingle}t a real problem), then the right thing to do is
double-check to make sure that the indicators are following reality.
For instance, if someone says, ``This will create 1,000 jobs,'' you
should check and make sure that it isn't at the
expense of other people’s jobs, or of the economy{\textquotesingle}s
health. The government can provide an infinite number of jobs, but if
the jobs
don{\textquotesingle}t produce anything, this doesn{\textquotesingle}t
help anyone. Depth is often the cure for seeing the flaws in shallow
indicators.

The second way to pursue this principle is to focus on the basics. It is
true that paper money allows the economy to do things that it
couldn't do otherwise. However, paper money is several
steps removed from actual value. As we pointed out in the chapters on
money, value production is entirely separate from money. The next step
out from value production is permanent assets such as gold and silver.
The next step from \textit{that} is paper money. Stepping out even
further are various derivatives from money, such as loans, insurance,
and other similar monetary tools. However, at each step you risk losing
sight of the actual valuable thing. Therefore, while it is fine and
good to keep and use paper money, it is also good to not neglect
holding things of more basic value
(i.e., gold, silver,
capital assets, and food
storage). Doing this
helps you to not only be prepared in case the paper money collapses,
but to keep
you in closer touch
to the things that are intrinsically valuable. Likewise with
health—there is nothing wrong with taking medication if needed, but
focusing on medication rather than healthy living and eating makes
one lose sight of
what health really
is. When you concentrate on the basics, it also helps prevent
charlatans from taking advantage of you, because you have a better
handle on where the real value is and what is happening to it.

Finally, be aware that most valuable things in life are non-measurable.
Even if all of the measurements of something are right, it could be
that a policy or idea is draining something that is important but
non-measurable.  For
instance, one of the important, non-measurable foundations for the
economy is a moral society.  Moral societies allow us to exchange goods
freely and without fear.  Therefore, if a policy undercuts public
morality, even if it has a short-term boost to the economy, in the long
term it will weaken its very foundation, and numbers will be of little
help diagnosing the problem.  We should look at policies, laws, and
ideas not only for their near-term effects, but also at how they
reshape the way society thinks and behaves. Laws send messages, and it
is important to know not only what effects a policy is having but the
message it is sending, and what that message will do to society if it
is received. For instance, Mother Theresa said
regarding abortion,
“Any country that accepts abortion is not teaching its people to love,
but to use any violence to get what it wants.”  In other words, instead
of asking what the effects of abortion policy are on crime or poverty,
Mother Theresa is pointing out that legalized abortion sends a message
that if something is in the way of your happiness, any violence, even
killing, is justified.  

\section{Principle 2: Create Margin}

We tend to live our lives without margin. That is, we spend all we make,
we are always busy, and there is never any extra. Margin is the “free
space” in your life and finances – the difference between your
obligations and your available resources.  We need margin in many
aspects of our lives – in our time, our money, our attention, and even
our friends.  Having margin is important for a number of reasons. 
First of all, it helps your well-being to not always be living on the
edge of disaster.  When our lives have no margin, any small deviation
can send us over the edge.  If there is no available time, if something
runs over it causes a crisis.  If there is no available money, then an
unexpected expense puts you into debt.  But this is not just a personal
problem for yourself and your family – it also limits your ability to
be a neighbor to those in your community. 
When margin is
removed from your life, what is there left to give?  

My own life has been abundantly blessed precisely because there were
people who created margin in their own life that allowed them the
ability to help our family when we needed it. If your food supply has
margin you will be able to help others out
in a crisis, not just
your own family.  If
your time has margin, then when your neighbor is sick, you can be there
to shop, cook, and take care of them when they need it.

I think that this is what primarily irks me about taxes. I really
don{\textquotesingle}t mind working and giving money to help the
community—even a lot of money. However, with the government taking 43
percent of my labor,
plus even more when they devalue the dollar by printing money, 
that’s a major reason,
or \textit{the
}reason my family
struggles to create margin. It’s difficult to help others when your
money is being removed by over-taxation.  Imagine if the government
only took 10\% of your labor.  How much could you help your community
if you had three extra months to do so?

Many pro-big-government people think that this is fine because the
government will provide to those in need. This is bogus for several
reasons. First, when individuals no longer have the ability to help
others, this decreases
their compassion,
making each person more concerned about him or herself.  Second, I
reject the notion that
any government,
especially a non-religious government, could possibly offer the same
scope of help that a
community of people with enough margin could do to help out. People
can effectively respond to and help the whole person. Governments
cannot. 

Therefore, despite so much of our labor being taken from us, we must
work to re-establish margin in our lives and livelihoods, so that we
can better prepare ourselves for the unexpected, and better help our
neighbor when life{\textquotesingle}s troubles come upon him.

\section{Principle 3: Produce Value Instead of Consuming It}

So, how do we do this?
How do we create margin in an economy that makes it seem impossible to
do so? Before I get to an actual budget, I want to impress this point
one more time: we should always value production over consumption.
Most of modern society
begs us to be
consumers. Most politicians and economists talk about how they can get
consumers to increase spending.
Remember the Bush tax
credit?  That time when you got that \$300 check some years ago that
you were supposed to go spend on something to stimulate the economy?
 The truth is, in
general, we don{\textquotesingle}t need to increase our spending. We
need to increase our production of value. Then we
wouldn{\textquotesingle}t need to consume in the first place.

Whether it is knitting your own clothes, growing your own food, or
making your own medicine, we are better off when we can produce what we
need and more. When we produce an abundance, presuming we are allowed
to keep it, then the variations in the outside economy will affect us
very little. 

We are often oblivious to the amount of consumption in our lives.
However, there is an easy way to measure what you are consuming. First,
keep every receipt you get for a full month. Then go back through those
receipts with a fine-tooth comb. Don{\textquotesingle}t just note that
you spent \$100 at Walmart—note exactly what you spent it on. Was this
a wise purchase? Could you have produced something else rather than
consumed that money?  

I know that I often try to solve problems by looking for that one big
thing to change, but I often miss the fact that I am actually having
problems because of a thousand small mistakes. Consumption is a lot
like that. I used to have a habit of buying a slushie every time I
purchased gas. No biggie—just a dollar!  But that one dollar, over a
year, or over a lifetime, adds up. Over a year, that{\textquotesingle}s
fifty dollars. Certainly I could find something better to spend fifty
dollars on!  I can make tea in the morning for five cents. If I make
tea from the lemon balm and spearmint plants growing in my front yard,
I can have it for free. How many different places are you leaking
dollar bills?  Could
you replace those
things with something that consumes less?  Even better, could you
replace them with
something that \textit{produces} instead?

\section{Principle 4: Remove Dependencies}

One of the problems we face living in such a rich nation is that we are
used to having everything work all the time. I{\textquotesingle}m glad
that things do generally work all of the time, but
isn{\textquotesingle}t it strange how when the power goes out for a day
or two, or the city shuts off the water for a few hours, we all get
tense and distressed? We have become too dependent on our luxuries.

Therefore, a principle of microsecession is to make a conscious effort
to avoid unnecessary dependencies. This comes in two flavors. First, if
you make a purchase, try to purchase an item least dependent on other
things. Don{\textquotesingle}t buy an electrical kitchen gadget when a
hand-crank one will do. Second, if you do buy things that depend on the
availability of modern niceties, be sure to also have a backup that
doesn{\textquotesingle}t. Buy candles so that you still have light if
the power goes out. Check the fireplace and have enough blankets so
that you will be fine if the power or gas is out for a week or a month
(I need to add that one to my own list!).

There are also more abstract entities that we need to remove our
dependencies from, and
we’ve touched on many of them here: the banking system, the food
supply, the healthcare system. Now, there will always be things that
we need these systems for. I{\textquotesingle}m writing this book on a
very nice computer, which, I can assure you, won{\textquotesingle}t run
for very long if the power goes out. But we need to be conscious about
what we are doing and the natural limitations and potential risks that
our decisions
impose.

\section{Principle 5: Make Friends with Nature}

We are addicted to the artificial. Even our food is becoming artificial.
We have forgotten about the abundance that is everywhere available
within nature. I would encourage anyone to spend time learning about
nature and about wilderness survival. I don{\textquotesingle}t expect
that survival skills are immanently necessary, but I think that
learning survival skills can help us
keep from panicking
when things get tough. When you realize that the whole world is
continually producing, then the whole world becomes a lot less scary.

We have conditioned ourselves to be scared of nature. We are worried
about snakes, poisonous plants, wild animals, deadly diseases, and an
array of other real and perceived dangers in the wilderness. The key to
this is knowledge and practice. By becoming acquainted with nature, we
can learn to be safe within nature. 

We drive cars every day. Cars are probably more dangerous than anything
nature could throw at us. However, we let people drive cars after a
minimal amount of training, and
we
don{\textquotesingle}t think too much of it. Especially
within a culture of
driving, the younger
generation will pick up habits, rules, and good or bad driving
practices—because they have been watching their parents drive for many
years. Similarly, anyone can learn to be safe in the wilderness, and
it is easier if we lead our children in it from an early age. Learn how
to start a fire with a flint and steel. Learn how to determine which
plants are safe to eat. Learn how to build a shelter and protect
yourself from the elements. Learn how to make rope from anything
fibrous. 

There is always an element of risk, but the same is true of driving.
There are 2 million people who are permanently injured by car accidents
every year, and 40,000 people who die. However, we
don{\textquotesingle}t let this stop us from driving to work every day.
By taking the time to get to know nature and learn how to navigate her
difficulties, we will find that the whole world gets a little less
ominous and worrisome. 

\section{Principle 6: Never Stop Learning}

Modern technology brings with it a never-ending supply of things we need
to know. However, I don{\textquotesingle}t think we should be so much
concerned with learning the ins and outs of the latest doohickey.
Rather than focusing on toys, we should enrich ourselves by both
deepening our understanding of life through philosophy and theology,
and widening our horizons and abilities through more skill-oriented
learning.

The great thing about learning is that anything you love you can learn
to do better. Even if there is no book, taking time to think and
analyze what you are doing is also learning. Do you love to watch
movies? Well, what makes a good plot?  How does a director or actor
pull on our heartstrings or raise our adrenaline?  What message is the
writer trying to convey?  As you can see, even for something as
mindless as watching movies, there is a sea of learning to be had.

There is a difference between a cook and a chef. A cook follows
instructions and produces a standard meal. A chef invents instructions
to make a masterpiece. And a chef can make a masterpiece out of
anything. There is a restaurant where I live called The Tavern. They
have what is called the “Grilled Cheese of the Day.”  When I make a
grilled cheese sandwich, it is pretty much what you expect—bread and
cheese. Tasty but not exciting. At The Tavern, each grilled cheese is
new, exciting, and amazing every single day. Why?  Because it is built
by a chef, not a cook. 

The difference between being chef-like and cook-like in our own lives is
based on whether we continually strive to expand ourselves by learning
new things. If we do, then our horizons open up and our creativity
expands; we can better adapt to changes and
more readily come up
with solutions to the problems that we face individually, as
communities, and as a nation.
