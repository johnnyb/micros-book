
\chapter{Money Secession for Communities}


While the ideas in the previous chapter are ways that we can
individually secede monetarily, this chapter will focuses on
larger-scale changes we can make. My point of these is not to recommend
a particular course of action but to help you understand the
possibilities available, and to start discussions within communities
and families about ways to organize such that they can become more
independent of the larger-scale economic problems. I don’t imagine that
any of these will be universally implementable, or even adoptable as
presented, but they should help stimulate ideas for your own action
within your own community.

\section{Using Precious Metals for Exchange}

FIXME - Insert section on bartering before this one


As has been pointed out
earlier, one of the reasons that monetary policy matters is because a
mishandling of monetary policy leads to a devaluing of the currency for
people, such that neither savings nor even monthly income holds its
value.


I am not one who believes that paper money is intrinsically bad.
However, the problem with paper money is that it requires a government
that is responsible and moral enough to handle it properly. I thought
about adding “intelligent” to that list, but, on reflection, I wonder
if intelligence isn’t more often deployed to circumvent responsibility
and morality rather than in support of it. In any case, the advantage
with using precious metals for exchange is that the government can’t
print it at will, forcing a monetary policy that at least somewhat
comports to reality.


When paper money was on the gold standard (i.e., it was redeemable for a
fixed amount of gold), there was very little, if any, inflation.
Inflation first started kicking in when individuals were prevented from
using gold for exchange. It started its leap upward when the nation
went off of the gold standard internationally under Nixon. As I said,
there is nothing intrinsically wrong with paper money—even paper money
not backed by precious metals. The problem, though, is that not backing
paper money with a physical asset leads to moral hazards. It leads
people to come up with tricks and gimmicks to get around sound policy.
When paper money is tied to an asset, that asset is very close to
reality and gets in the way of people who want to implement bad
policies. When paper money is not tied to hard assets, it is easy for
people to fudge the numbers and start thinking of their paper bills as
monopoly money.


The solution in such times is to go back to using hard assets for
exchange. Gold and silver have traditionally played this role because
of their scarcity and exceptional imperviousness to deterioration. This
makes them intrinsically valuable metals. Gold and silver are also
beautiful, and they both have a number of industrial uses (though
silver has a few more than gold).


If the government is going to gamble with the value of its currency, it
is up to the rest of us to choose some other way of denominating the
value of our assets, contracts, and employment. I would suggest
denominating these in ounces of silver. Now, unfortunately, the
government has mandated that its currency be accepted in payment of all
debts. However, we can set the payment terms of contracts in terms of
the value of silver, so that currency can be used to pay the contract,
but the value adjusts based on the price of silver. 


While some tax issues make it difficult to do denominate contracts in
silver, some states have already started making headway in this
direction. While no one can prevent you from tying the value of a
contract to silver or accepting silver as payment, there are certain
laws that make this easier to do, and incur less of a tax burden. Utah
enacted a law in 2011 to help move toward a metals-based currency.
Basically, they eliminated state capital gains tax on the sale of
silver and gold, which makes it easier to exchange it back and forth
between dollars.


I should note that if you are looking to purchase silver as a future
transactable currency, you might consider focusing on American Eagle
coins, minted by the U.S. government. Most silver laws reference U.S.
minted coins, so if you want your coins to be usable as money in the
future, these seem to be the best option.


To restate, the benefits of silver and gold as currency is that the
value of physical assets are tied more directly to the value of the
economy than are dollars.  Dollars are tied closer to monetary policy
than the actual value being produced. Using silver and gold (for most
of us probably just silver) does not insulate us from the ups and downs
of the value of the economy as a whole—nothing can do that except for
the value produced in your own household. It does, however, reduce your
dependency on the morality and responsibility of those in charge of
monetary policy.

\section{A Community Gift Economy}

FIXME - Revisit this section


In a typical economy, gains are made by specialization. If I’m really
good at computer programming, and you are really good at fixing cars,
then together we can achieve more if I leave fixing cars to you, and
you leave computer programming to me. It would take me days to do a
simple fix to my own car that someone who knew a lot about cars could
probably do in fifteen minutes. Thus, almost all economy books point
out that everyone’s value increases when we specialize. However, it
isn’t really specialization that’s at work per se. To say that
specialization is what causes the increase in value is to misplace
where over-specialization causes problems. It isn’t that every time we
increase specialization,
we all get
get more money, and
every time we decrease specialization, we all get poor. It’s just that
different people have different efficiencies, and it makes the most
sense for those most efficient at something to be the ones who do it
the most.


In order for specialization to work out, it requires that some form of
trade happen. If you fix cars all day, that doesn’t help out the
economy if they are all your own cars. It only helps everyone else out
if you fix your cars and other people’s cars, and if I program not only
my own computers but others’ computers as well. But there must be some
way of getting my cars to you and your computers to me. Normally that
is done through trade—whether through money or through a barter system.
Since we are trying to avoid the monetary system, it turns out that
even bartering puts us into the money economy, because we will be hit
by taxes, which are charged in dollars, not in commodities or services.
Therefore, all real transactions wind up tying us in some way to the
money economy and to taxes.


To extricate ourselves from taxes, we are going to have to try something
that is quite foreign to our modern ways of thinking. We need to
implement a gift economy. A gift economy is one where things are given
to one another without any individual expectation of return. If you
have seen the movie or read the book \textit{Pay it Forward}, you have
an idea of what a gift economy might look like.


The most important part of a gift economy is that it cannot be
formalized. If it is formalized, then it is no longer a gift economy.
The transactions are then more like taxes or payments than gifts. A
gift economy is done by a group of likeminded, generous people. A good
example of a gift economy is from the early Christian church. This
group of people, who organized themselves shortly after Jesus’
resurrection, were described in the book of Acts like this:


All the believers were together and had everything in common. They sold
property and possessions to give to anyone who had need. Every day they
continued to meet together in the temple courts. They broke bread in
their homes and ate together with glad and sincere hearts, praising God
and enjoying the favor of all the people. 

{\raggedleft\selectlanguage{english}\color{black}
Acts 2:44-46 (NIV)
\par}


There are several features of this that are important. One of the most
important aspects of this is that there are several parts that all went
together to make it work. The people engaging in this gift economy were
not isolated individuals. All of the members of the community had a
shared relationship with one another. They met together daily. They ate
together in their homes. They knew one another well and were very
close. Second, participation was truly voluntary. Nothing in this
economy was forced. Third, they had a shared religious outlook. That
is, they agreed on moral codes and the purpose of life. Finally, they,
as a community, focused outwardly on God rather than inwardly on
themselves.


These are the reasons why this sort of economy has rarely been
successful in the world and is impossible to implement on a large
scale. In larger scales, the participants cannot be in direct community
with one another. This keeps the bond of trust from being able to form
as well, which is crucial
in a gift economy. It
also leads to differentiated moral understandings, which can easily
lead to members of different communities not trusting that
others are handling
their gifts properly. Also, if gifts or even membership in such
communities is coerced, then the spirit of giving that drives such
economies is quickly squelched. Part of the notion of giving is the
freedom not to give.


A gift economy rarely works on a large scale
because of these
things, but it is actually fairly easy to implement in small scales as
partial solutions. In fact, most churches already have some sort of
system like this. For instance, many churches, when someone is sick,
will coordinate others bringing them food while they are ill. There is
no compulsion, but everyone in the community seeks to help out their
neighbor who is in distress. 


As I said, this sort of economy works great even as a partial solution.
However, there is an important ingredient needed for a gift economy to
work well: families need to have sufficient time and flexibility to
provide gifts as needs arise. This may seem obvious, but many families
have organized themselves in such a busy manner that no time is
available for unexpected problems even in their own lives, much less in
the lives of their neighbors.  I think that the two biggest
factors at play in
debilitating the gift economy solution are the two-income family and
the desire to
keep children
constantly busy.


Let me share with you a story from my own life. My wife and I have lost
two of our five children to genetic disease. Danny, our first, lived to
be five and a half years old, and Isaac, our last, lived to be three
months old. Isaac spent most of his life in the hospital, and my wife
was right there by his side the whole time. However, we also had three
other children who needed care. Thankfully, our church community (which
extended past just our physical church) came together to help. Our
community came together and not only provided us with an abundance of
meals, but they organized a schedule of care in which several of the
moms in the church took turns watching our kids throughout the day
every day of every week for \textit{two months}. 


Let me tell you that such care simply isn’t available if everyone in the
community is working full-time jobs. Such care is certainly not
available from insurance plans or government assistance. Even if these
third-party entities could provide the childcare, they could not offer
the compassionate,
godly help that our children desperately needed at that time. Going
back to our discussion of value, because our community opted to have
reduced income by having one of the parents not work, there was a large
store of value that was available that far surpassed anything that
money could buy.


When you have a two-income family, your ability to give non-monetarily
to the community is greatly reduced. Our communities, more than
anything else, need us. We need to have the breathing room as a family
to be able to help the community as needs arise, and this won’t happen
if every hour of every day is tied to our paychecks.


Likewise, as business owners, we need to make sure that our employees
aren’t so tightly scheduled that they cannot afford to be away for
emergencies. I have been fortunate to be employed by good people, and I
want to take a moment to give public thanks to both Wolfram Research
and New Medio for giving me such great flexibility when my family was
in trouble. When Danny was still with us, he would have physical
meltdowns almost instantaneously, and after being admitted to the local
hospital, he had to be transferred to another hospital in a different
state. You know your life didn’t turn out like you expected when you
are on a first-name basis with the local life-flight crew. In any case,
I would frequently call my boss from another state and say that I
wouldn’t be in to work for at least another week. This was my life for
many years, and I am thankful that I always had employers who were
willing to give when I needed it.


We need to be in communities where we are able and willing to give to
one another as we have need. But prior to giving to one another as we
have need, we must be in community. When you are in a loving community,
it brings out the sense of responsibility in each of us, including our
children. It helps us understand what need looks like as a concrete
reality, instead of an abstract idea. It is easy to talk about the
percentage of people in the country trying to get by on such-and-such
an income, but the problem with the statistic is that each and every
person with that income has his or her own unique story, unique
problems, unique needs, and unique abilities. By shoving all of these
things into a single category based on how much a person earns is
ludicrous and does injustice to the real situations that people find
themselves in. But in community we can see the needs directly and act
appropriately to the situation. 


The more \textit{value} we generate through gifts to the community, the
less money the community needs to operate. This is not a method of
self-enrichment but rather community enrichment. By living in a gifting
community, we may have a lower income in monetary terms, but the value
that we receive may be greater than is possible even with the largest
paycheck. Why?  Because being in community with people is a value unto
itself, which cannot be purchased at any price.


In fact, there are many intangibles that cannot be purchased with money
at any price. Family, friendship, and community all come from other
sources, but these things often coincide with at least some sort of
economic (even if not monetary) activity. If you have friends over for
dinner, you are both building value that is completely outside any
measurable economic valuation (friendship and community) and
simultaneously adding quantifiable economic value (the meal).

\section{Policy Note -
Silver and Gold as Legal Tender}

Now that a few states are starting to pick up on the idea that silver
and gold should be viewed as money, it would be wise for the federal
government to do likewise. It doesn’t have to alter its policies except
to stop taxing gold as an investment, and rather treat it as a valid
form of currency.


I don’t see us returning to the gold standard, but I do think that
giving people the option of using a more traditional style of money for
payments would be wise. Several states have already started a process
to assist in this at the state level. It would be helpful if more
states adopted this approach, and if the federal government did too.
